\documentclass[12p,numbers=noendperiod,DIV=15]{scrreprt}

\usepackage[ngerman]{babel}
\usepackage[utf8]{inputenc}
\usepackage[T1]{fontenc}

% Schriftart DejaVu
\usepackage{dejavu}
\usepackage[onehalfspacing]{setspace}
%\setstretch{1.5}

\setlength{\parindent}{0pt}
\setlength{\parskip}{2ex plus0.5ex minus0.5ex}

\usepackage{enumitem}
\setlist{noitemsep,topsep=0pt}

% Kopf- und Fußzeile
%\pagestyle{headings}
%\usepackage[singlespacing=true]{scrlayer-scrpage}
%\ifoot{\copyright~Norbert Seulberger, 2021}
%\cfoot{}
%\ofoot{\pagemark}
%\pagestyle{scrheadings}
\pagestyle{plain}
%\setkomafont{pageheadfoot}{\small}

% Koma-Script für Inhaltsverzeichnis
\usepackage{tocbasic}
\DeclareTOCStyleEntries[dynnumwidth=true]{tocline}{chapter,section,subsection}

\usepackage{color}
\usepackage{listings}
\usepackage[fleqn]{amsmath}
\usepackage{fdsymbol}
\usepackage{scrhack}
%\usepackage[top=20mm,left=20mm,right=20mm,bottom=30mm]{geometry}
\mathindent=2em

\usepackage{tikz}
\usetikzlibrary{shapes.geometric, arrows, arrows.meta, positioning}

\sloppy

%% Umgebungen

\newcommand{\seitenumbruch}{\newpage\textcolor{white}{Seitenumbruch}}

\newcounter{regelsatz}[section]

\newenvironment{ncsEnvironmentEins}[1]{\stepcounter{regelsatz} \textbf{\arabic{section}.\arabic{regelsatz} #1} \\}{}
%\newcommand{\aufgabe}[1]{\begin{ncsEnvironmentEins}{Aufgabe: #1} \end{ncsEnvironmentEins}}
\newcommand{\beispiel}[1]{\begin{ncsEnvironmentEins}{Beispiel: #1} \end{ncsEnvironmentEins}}
\newcommand{\hinweis}[1]{\begin{ncsEnvironmentEins}{Hinweis: #1} \end{ncsEnvironmentEins}}
%\newcommand{\loesung}[1]{\begin{ncsEnvironmentEins}{Lösung: #1} \end{ncsEnvironmentEins}}
%\newcommand{\motivation}[1]{\begin{ncsEnvironmentEins}{Motivation: #1} \end{ncsEnvironmentEins}}
\newcommand{\nurTitel}[1]{\begin{ncsEnvironmentEins}{#1} \end{ncsEnvironmentEins}}
%
\newenvironment{ncsEnvironmentZwei}[2]%
{\stepcounter{regelsatz} \textbf{\arabic{chapter}.\arabic{regelsatz} #1} \\ #2}{}
\newcommand{\titelText}[2]{\begin{ncsEnvironmentZwei}{#1}{#2}\end{ncsEnvironmentZwei}}

%\newenvironment{ncsEnvironment}[2]%
%{\stepcounter{regelsatz} \textbf{\arabic{chapter}.\arabic{regelsatz} #1: #2} \\ }{}
%\newcommand{\ncsEnv}[3]{\begin{ncsEnvironment}{#1}{#2} #3 \end{ncsEnvironment}}
%\newcommand{\empfehlung}[2]{\begin{ncsEnvironment}{Empfehlung}{#1} #2 \end{ncsEnvironment}}
%\newcommand{\frage}[2]{\begin{ncsEnvironment}{Frage}{#1} #2 \end{ncsEnvironment}}
%\newcommand{\refactoring}[2]{\begin{ncsEnvironment}{Refactoring}{#1} #2 \end{ncsEnvironment}}
%\newcommand{\synonym}[2]{\begin{ncsEnvironment}{Synonym}{#1} #2 \end{ncsEnvironment}}
%\newcommand{\syntaxPython}[2]{\begin{minipage}{\textwidth}%
%		\begin{ncsEnvironment}{Syntax in Python}{#1} #2 \end{ncsEnvironment}\end{minipage}}
%\newcommand{\test}[2]{\begin{ncsEnvironment}{Unittest}{#1} #2 \end{ncsEnvironment}}
%\newcommand{\tipp}[2]{\begin{ncsEnvironment}{Tipp}{#1} #2 \end{ncsEnvironment}}
%\newcommand{\warnung}[2]{\begin{ncsEnvironment}{Warnung}{#1} #2 \end{ncsEnvironment}}

\newenvironment{ncsItEnvironment}[2]%
{\stepcounter{regelsatz} \textbf{\arabic{chapter}.\arabic{regelsatz} #1: #2} \\ \begin{itshape}}{\end{itshape}}
\newcommand{\begriff}[2]{\begin{ncsItEnvironment}{Begriff}{#1} #2 \end{ncsItEnvironment}}
%\newcommand{\prinzip}[2]{\begin{ncsItEnvironment}{Prinzip}{#1} #2 \end{ncsItEnvironment}}
%\newcommand{\ncsItEnv}[3]{\begin{ncsItEnvironment}{#1}{#2} #3 \end{ncsItEnvironment}}
% Listings Styles

\definecolor{darkred}{rgb}{0.8,0,0}
\definecolor{darkgreen}{rgb}{0.0,0.5,0}

\lstdefinestyle{syntaxPython}{
	language=python, 				        		
	basicstyle=\linespread{1.2}\ttfamily,
	keywordstyle=\color{darkred}\bfseries,	
	identifierstyle=\color{blue},				
	commentstyle=\color{darkgreen},			
	stringstyle=\ttfamily\color{darkgreen}, 			    		
	showstringspaces=false,
	breaklines=true, 			        		
	numbers=none, 				        		
	numberstyle=\small,		        			
	frame=single, 				        		
	tabsize=4,				            		
	float=htbp,
	captionpos=b,
	aboveskip=2ex,
	belowskip=\medskipamount,
	xleftmargin=.02\textwidth}

\lstdefinestyle{syntaxHTML}{%
	language=HTML,%
	basicstyle=\small\ttfamily,%
	keywordstyle=\color{darkred}\bfseries,%
	identifierstyle=\color{blue},%	
	stringstyle=\color{darkgreen},%
	tabsize=4,%
	showstringspaces=false,%
	numbers=none,%			        		
	numberstyle=\small,%	        			
	frame=single,%
	captionpos=b,%
	xleftmargin=.06\textwidth
} 

\lstset{literate=%
	{Ö}{{\"O}}1
	{Ä}{{\"A}}1
	{Ü}{{\"U}}1
	{ß}{{\ss}}2
	{ü}{{\"u}}1
	{ä}{{\"a}}1
	{ö}{{\"o}}1
}

\lstnewenvironment{codePython}{\lstset{style=syntaxPython}}{}
\lstnewenvironment{codeHTML}{\lstset{style=syntaxHTML}}{}

\usepackage[colorlinks=true,linkcolor=blue]{hyperref}

\begin{document}
	
\title{Mathehelfer Bruchrechnen}
\subtitle{Ein Hackathon für Kinder und Jugendliche}
\author{Norbert Seulberger}
\date{Mai 2023}

\maketitle

\tableofcontents

\chapter{Fahrplan für den Kurs}

\section*{Der Fahrplan zur Orientierung}

\subsection*{MathML}

\subsubsection*{MathML: Einführung}

Lesen und verstehen!


\subsection*{Algorithmen}


\subsection*{Variablen}


\subsection*{Funktionen}


\subsection*{Beispiel Kürzen}

\subsection*{Aufgabe Erweitern}

\subsection*{Aufgabe Multiplizieren}

\subsection*{Zeichenketten}

\subsection*{Formel Kürzen}

\subsection*{Formel Erweitern}

\subsection*{Formel Multiplizieren}

\chapter{MathML}

\section{MathML}

MathML ist eine einfache Sprache, die in einem HTML-Dokument dazu verwendet werden kann, um mathematische Formeln darzustellen.

Genauso wie in HTML werden die Daten in sogenannte Tags eingeschlossen, d.h. ein Element beginnt mit einen Tag, dann stehen die Daten und ein schließendes Tag beendet den Ausdruck. 

\subsection*{MathML: Anfang und Ende einer Formel}

Eine Formeln beginnt mit \texttt{<math>} und endet mit \texttt{</math>}.

Für die verschiedenen Teile einer mathematischen Formel gibt es unterschiedliche Elemente.
\begin{itemize}
	\item \texttt{<mn>}42\texttt{</mn>} stellt die Zahl 42 dar.
	\item \texttt{<mo>}=\texttt{</mo>} beschreibt ein Gleichheitszeichen.
	\item \texttt{<mo>}+\texttt{</mo>} schreibt ein Pluszeichen.
\end{itemize}

Die Formel $2 + 3 = 5$ lautet also in MathML:

\begin{codeHTML}
<math>
	<mn>2</mn>
	<mo>+</mo>
	<mn>3</mn>
	<mo>=</mo>
	<mn>5</mn>
</math>
\end{codeHTML}

\textbf{Tipp}

Der Browser ignoriert die Zeilenumbrüche in MathML. Wir können die Formel also auch in eine Zeile schreiben.

\begin{codeHTML}
<math><mn>2</mn><mo>+</mo><mn>3</mn><mo>=</mo><mn>5</mn></math>
\end{codeHTML}

\subsection*{Brüche in MathML}

Ein Bruch beginnt mit \texttt{<mfrac>} und endet mit \texttt{</mfrac>}. Den Zähler und den Nenner schreiben wir zwischen \texttt{<mn>} und \texttt{</mn>}.

Der Bruch $\frac{2}{3}$ sieht also so aus:

\begin{codeHTML}
<math>
	<mfrac>
		<mn>2</mn>
		<mn>3</mn>
	</mfrac>
</math>
\end{codeHTML}

\subsection*{Eine Formelzeile}

Damit jede Formel in eine eigene Zeile geschrieben wird, verwenden wir das Absatz-Tag von HTML: \texttt{<p>} ... \texttt{</p>}

Also zusammen:
\begin{codeHTML}
<p>
	<math><mn>2</mn><mo>+</mo><mn>3</mn><mo>=</mo><mn>5</mn></math>
</p>
\end{codeHTML}

\subsection*{Eine Anleitung zu MathML}

Eine einfache, aber umfassende Anleitung zu MathML findest du auf folgender Internetseite: \href{https://www.math-it.de/Publikationen/MathML_de.html}{MathML-Tutorial}
\section{Aufgabenblatt MathML}

\subsection{Aufgabe: Anzeige der HTML-Datei}

Schau dir im Firefox-Browser die Datei \texttt{manuell.html} an. Die Anzeige sollte etwa so aussehen:

\begin{equation*}
\frac{6}{8} = \frac{3}{4}
\end{equation*}

Hier kannst du die Formeln für das Umrechnen eines Bruchs in eine Dezimalzahl und umgekehrt einfügen.

\subsection{Aufgabe: Ansicht der HTML-Datei im Editor}

Schau dir jetzt dieselbe Datei im Editor der Entwicklungsumgebung an. Der Text sieht etwa so aus:

\begin{codeHTML}{Die Datei manuell.html}
<!doctype html>
<html lang="de">
	<head>
		<meta charset="utf-8">
		<meta name="viewport" content="width=device-width, initial-scale=1.0">
		<title>Bruchrechnen</title>
	</head>
	<body>
		<p>
			<math>
				<mfrac>
					<mi>6</mi>
					<mi>8</mi>
				</mfrac>
				<mo>=</mo>
				<mfrac>
					<mi>3</mi>
					<mi>4</mi>
				</mfrac>
			</math>
		</p>
		<p>
 			Hier kannst du die Formeln für das Umrechnen ...
		</p>
	</body>
</html>
\end{codeHTML}

Kannst du die Element entdecken, die für die Darstellung des Kürzens der beiden Brüche stehen?

\subsection{Aufgabe: Die Beschreibung der beiden Umrechungen}

Füge in die Datei \texttt{manuell.html} den Text in der MathML-Sprache ein, so dass die beiden Umrechungen dargestellt werden (zusätzlich zur Formel für das Kürzen). Der Text „Hier kannst du \dots“ darf überschrieben werden.

\begin{align*}
& \frac{6}{8} = \frac{3}{4} & \\[2ex]
& \frac{3}{4} = 0.75 & \\[2ex]
& 0.8 = \frac{4}{5} & 
\end{align*}

Überprüfe das Ergebnis im Firefox-Browser.

\chapter{Algorithmen und Funktionen}

\section{Algorithmen}
\label{sec:Algorithmus}

Ein Algorithmus ist die eindeutige Beschreibung eines Verfahrens, um eine bestimmte Aufgabe zu lösen.

\subsection*{Beispiel: Das Erweitern eines Bruches}

Ein Bruch wird erweitert, indem der Zähler und der Nenner mit demselben Faktor multipliziert werden. Wird beispielsweise der Bruchs $\frac{2}{5}$ mit dem Faktor $3$ erweitert, so ergibt sich diese Gleichung:
\[
\frac{2}{5} = \frac{2 \cdot 3}{5 \cdot 3} = \frac{6}{15}
\]

\subsection*{Der Algorithmus für das Erweitern}

\begin{itemize}
	\item Nimm einen Bruch und den Faktor, mit dem du den Bruch erweitern willst.
	\item Merke dir den Zähler und den Nenner des Bruchs jeweils in einer Variablen.
	\item Multipliziere die Variable für den Zähler mit dem Erweiterungsfaktor.
	\item Verfahre genauso mit dem Nenner.
	\item Erstelle einen neuen Bruch mit dem neuen Zähler und dem neuen Nenner.
\end{itemize}
\section{Variablen}

\subsection*{Einfache Variable}

Eine Variable dient dazu, sich einen Wert, einen Text oder das Ergebnis einer Rechnung zu merken. Wir geben einer Variablen einen sinnvollen Namen.

\begin{codePython}
wichtige_zahl = 42
summe = 2 + 3
\end{codePython}

Mit Variablen können wir auch rechnen.

\begin{codePython}
summe = summe + 6
\end{codePython}

\subsection*{Strukturierte Variablen}

Wir können Variablen auch komplizierter aufbauen und einen Wert in einer Variablen merken, der sich aus mehreren Teilwerten zusammensetzt. Ein Beispiel ist der Bruch:

\begin{codePython}
class Bruch():
	zaehler: int
	nenner: int
\end{codePython}

Ein Bruch ist ein Wert, der sich aus zwei Teilwerten zusammensetzt, nämlich dem Zähler und dem Nenner. Den Bruchstrich müssen wir uns nicht merken, weil er immer da ist. Zähler und Nenner sind ganze Zahlen (\texttt{int}).
%Wir merken uns den Zähler und Nenner zusammen in einem Bruch. Diesen Bruch können wir dann als einen Wert in einer Variablen aufbewahren.

\lstset{%
	language=python, 				        		
	basicstyle=\linespread{1.2}\ttfamily,
	keywordstyle=\color{darkred}\bfseries,	
	identifierstyle=\color{blue},				
	commentstyle=\color{darkgreen},			
	stringstyle=\ttfamily\color{darkgreen}, 			    		
	showstringspaces=false,
	breaklines=true, 			        		
	numbers=none, 				        		
	numberstyle=\small,		        			
	frame=single, 				        		
	tabsize=4,				            		
	float=htbp,
	captionpos=b,
	aboveskip=2ex,
	belowskip=\medskipamount,
	xleftmargin=.02\textwidth			
}
\begin{lstlisting}
bruch_1 = Bruch(3,4)
\end{lstlisting}

Jetzt können wir die Variable, die ja einen Bruch enthält, nach dem Zähler und dem Nenner fragen. Wir schreiben die beiden Werte in eigene Variablen.

\begin{lstlisting}
zaehler_1 = bruch_1.zaehler
nenner_1 = bruch_1.nenner
\end{lstlisting}

Wir können auch zwei Ganzzahlen jeweils in eine Variable schreiben und damit eine Bruchvariable erzeugen.

\begin{lstlisting}
zaehler_2 = 7
nenner_2 = 8
bruch_2 = Bruch(zaehler_2, nenner_2)
\end{lstlisting}
\section{Funktionen}

Eine Funktion ist ein kleines Teilprogramm, das eine spezifische Aufgabe löst. Für bestimmte Eingabewerte erhält man immer dasselbe Ergebnis.

\beispiel{Eine Funktion zur Mittelwertberechnung zweier Zahlen}
Der Algorithmus lautet:

\section{Aufgabe: Das Kürzen von Brüchen}

\subsection*{Der Algorithmus}

Wie lautet der Algorithmus zum Kürzen eines Bruchs?

{\Large
\begin{itemize}
	\item  
	\item  
	\item  
	\item  
	\item  
	\item  
\end{itemize}
}

\subsection*{Die Funktion zum Kürzen von Brüchen}

Wie lautet die Funktion?

Vermutlich benötigst du eine Funktion zur Berechnung des größten gemeinsamen Teilers von zwei Zahlen. Diese findest du bei den mathematischen Hilfsfunktionen.

\begin{codePython}
def kuerzeBruch(bruch: Bruch) -> Bruch:
\end{codePython}


\subsection*{Überprüfung des Ergebnisses}

Überprüfe das Ergebnis in der Datei \texttt{spielwiese.py}.
\section{Aufgabe: Das Erweitern eines Bruchs}

\subsection*{Der Algorithmus zum Erweitern}

Wie lautet der Algorithmus zur Erweitern eines Bruchs mit einem vorgegebenen Faktor?

{\huge
\begin{itemize}
	\item Nimm einen Bruch und den Faktor, mit dem du den Bruch erweitern willst.
	\item  
	\item  
	\item  
	\item 
	\item  
	\item  
\end{itemize}
}

\subsection*{Die Funktion zum Erweitern}

Programmiere die Funktion zum Erweitern eines Bruchs. Die Funktion erhält als Parameter einen Bruch sowie den Faktor zum Erweitern. Die Signatur lautet:

\begin{codePython}
def erweitereBruch(bruch: Bruch, faktor: int) -> Bruch:
\end{codePython}

Vergiss nicht die letzte Zeile mit dem \texttt{return}-Befehl!

\subsection*{Überprüfung des Ergebnisses}

Überprüfe das Ergebnis. Trage dazu die notwendige Befehle in die Datei
\begin{quote}
	\texttt{uebungsplatz.py}
\end{quote}
ein. Du kannst dich daran orientieren, wie wir es beim Kürzen gemacht haben. Hier möchten wir den Bruch $\frac{3}{4}$ mit dem Faktor 3 erweitern.

Es geht los mit

\begin{codePython}
	bruch = Bruch(3,4)
\end{codePython}

Wie geht es weiter?

Danach führen wir \texttt{uebungsplatz.py} als Python-Skript aus. Das Ergebnis sollte jetzt diese Zeile enthalten:
\begin{quote}
	\texttt{Bruch(zaehler=9, nenner=12)}
\end{quote}

Sieht das Ergebnis bei dir genauso aus?

\section{Aufgabe: Das Multiplizieren von Brüchen}

\subsection*{Der Algorithmus zum Multiplizieren von Brüchen}

Wie lautet der Algorithmus zum Multiplizieren von Brüchen?

\begin{itemize}[itemsep=2ex]
	\item Nimm zwei Brüche entgegen.
	\item 
	\item 
	\item 
	\item 
	\item 
	\item  
	\item 
	\item 
\end{itemize}

Hast du daran gedacht, den neuen Bruch zu kürzen?

\subsection*{Die Funktion zum Multiplizieren von Brüchen}

Wie lautet die Funktion?

\subsection*{Überprüfung der Funktion zum Multiplizieren}

Wir möchten in der Datei
\begin{quote}
	\texttt{uebungsplatz.py}
\end{quote}
die beiden Brüche $\frac{3}{4}$ und $\frac{2}{3}$ miteinander multiplizieren. Es beginnt also mit

\begin{codePython}
erster_faktor = Bruch(3,4)
zweiter_faktor = Bruch(2,3)
\end{codePython}

Wie geht es weiter?

Führe \texttt{uebungsplatz.py} als Python-Skript aus. Die Ausgabe muss folgende Zeile enthalten:
\begin{quote}
	Produkt: Bruch(zaehler=1, nenner=2)
\end{quote}

\chapter{Zeichenketten}
\section{Zeichenketten}

\subsection*{Was ist eine Zeichenkette?}

Eine Zeichenkette enthält ein oder mehrere beliebige Zeichen, also Buchstaben, Zahlen oder Sonderzeichen. Die Zeichen stehen hintereinander. Eine Zeichenkette wird eingeschlossen in Hochkommata (') oder hochgestellte Anführungszeichen (").

\begin{codePython}
eine_zeichenkette = "r2d2"
berg = '8000er-Gipfel'
ein_ganzer_satz = "Python ist toll!"
\end{codePython}

Häufig benutzen wir den englischen Begriff für Zeichenkette: String. Mit Strings kann man viele interessante Dinge anstellen.

\subsection*{Das Zusammenfügen von Zeichenketten}

In Python können zwei Strings mit dem Pluszeichen zu einem langen String zusammengefügt werden:

\begin{codePython}
gruss = "Hallo " + "Welt!"
\end{codePython}

In \texttt{gruss} steht jetzt der String \texttt{"Hallo Welt!"}.

\subsection*{Tipp: Das Pluszeichen setzt Zeichenketten zusammen!}

Das Pluszeichen bedeutet in diesem Fall nicht das Addieren von Zahlen, sondern das Zusammenfügen von Zeichenketten. Daran gewöhnt man sich schnell!

\subsection*{MathML als Zeichenkette}

Wir möchten einen String zusammenbauen, der in MathML eine Rechenformel beschreibt.

Die Formel
\[
2 + 3 = 5
\]
lautet in MathML:

\begin{codeHTML}
<math><mn>2</mn><mo>+</mo><mn>3</mn><mo>=</mo><mn>5</mn></math>
\end{codeHTML}

Diese Formel können wir in Python schrittweise aus kleinen Strings zu einem langen String zusammensetzen. 

\lstset{style=syntaxPython}
\begin{lstlisting}
formel = "<math>"
formel = formel + "<mn>2</mn>"
formel = formel + "<mo>+</mo>"
formel = formel + "<mn>3</mn>"
formel = formel + "<mo>=</mo>"
formel = formel + "<mn>5</mn>"
formel = formel + "</math>"
\end{lstlisting}

Dabei nehmen wir die bereits erstellte Formel, fügen den nächsten kurzen String an und merken uns das Ergebnis wieder in der Variablen \texttt{formel}.
\include{./Kapitel/MathMLSchreibeBruch}

\chapter{Das Erstellen der Formeln}

\section{Beispiel: Die Formel zum Kürzen eines Bruchs in MathML}

\subsection*{Die Vorarbeiten}

\begin{itemize}
	\item Wir verfügen über die Funktion \texttt{kuerzeBruch}. (siehe \ref{sec:FunktionKuerzen})
	\item Außerdem haben wir die Funktion \texttt{schreibeBruch}, die einen Bruch in MathML darstellt. (siehe \ref{sec:MathMLBruch})
\end{itemize}

\subsection*{Der Algorithmus zur Darstellung der Formel zum Kürzen}

\begin{itemize}
	\item Nimm einen Bruch.
	\item Erzeuge aus diesem Bruch einen gekürzten Bruch. Verwende dazu die Funktion, die du gerade programmiert hast.
	\item Erzeuge aus dem ursprünglichen Bruch die Beschreibung in MathML. Verwende dazu die Funktion, die einen Bruch in MathML beschreibt.
	\item Erzeuge aus dem gekürzten Bruch ebenfalls die Beschreibung in MathML. Das geht genauso wie eben.
	\item Setze jetzt den MathML-Text aus den einzelnen Angaben zusammen.
	\item Die Zeichenkette beginnt mit dem MathML-Text für den ungekürzten Bruch.
	\item Dann kommt das Gleichheitszeichen.
	\item Es folge der MathML-Text für den gekürzten Bruch.
\end{itemize}

\subsection*{Die Funktion zur Darstellung der Formel zum Kürzen}

In der Datei \textbf{\texttt{mathehelfer.py}} programmieren wir folgende Funktion:

\begin{codePython}
def schreibeKuerzen(bruch: Bruch) -> str:
	bruchGekuerzt = kuerzeBruch(bruch)
	textUngekuerzt = schreibeBruch(bruch)
	textGekuerzt = schreibeBruch(bruchGekuerzt)
	ergebnis = textUngekuerzt + "<mo>=</mo>" + textGekuerzt
	return ergebnis
\end{codePython}

\subsection*{Überprüfung des Ergebnisses}

Überprüfe das Ergebnis in der Datei \textbf{\texttt{uebungsplatz.py}}.

\lstset{style=syntaxPython}
\begin{lstlisting}
bruch = Bruch(6,8)
formel = mathehelfer.schreibeKuerzen(bruch)
print(formel)
\end{lstlisting}

Der Bruch $\frac{6}{8}$ muss gekürzt den Bruch $\frac{3}{4}$ ergeben. In einer Zeile steht:

\begin{codeHTML}
<mfrac><mn>6</mn><mn>8</mn></mfrac><mo>=</mo>
								<mfrac><mn>3</mn><mn>4</mn></mfrac>
\end{codeHTML}

Jetzt arbeiten wir wieder in der Datei \textbf{\texttt{mathehelfer.py}}.

Ergänze in der Funktion \texttt{schreibeMathML()} die Zeilen für Kürzen für den Bruch $\frac{6}{8}$.

\begin{codePython}
bruch = Bruch(6,8)
inhalt = inhalt + "\n\t\t<p><math>"
inhalt = inhalt + schreibeKuerzen(bruch)
inhalt = inhalt + "</math></p>"
\end{codePython}

Führe die Datei \texttt{htmlErzeugung.py} aus.

Prüfe im Firefox-Browser, wie die Datei \texttt{index.html} aussieht. Dort sollte eine Zeile stehen, die so aussieht:
\[
\frac{6}{8} = \frac{3}{4}
\]

\section{Aufgabe: Die Formel zum Erweitern eines Bruchs in MathML}

\subsection*{Erledigte Vorarbeiten}

Welche bereits erstellten Funktionen können wir sinnvoll nutzen?

\begin{itemize}[itemsep=2ex]
	\item 
	\item
\end{itemize}

\subsection*{Der Algorithmus zur Darstellung der Erweiterungsformel in MathML}

Wie lautet der Algorithmus?

\begin{itemize}[itemsep=5ex]
	\item Nimm einen Bruch und einen Faktor zur Erweiterung entgegen.
	\item 
	\item 
	\item 
	\item 
	\item 
	\item 
	\item 
\end{itemize}

\pagebreak

\subsection*{Die Funktion zur Darstellung der Formel in MathML}

Wie lautet die Funktion? Die Signatur sieht so aus:

\lstset{style=syntaxPython}
\begin{lstlisting}
def schreibeErweitern(bruch: Bruch, faktor: int) -> str:
\end{lstlisting}

Wie geht es weiter?

\subsection*{Überprüfung des Ergebnisses}

Überprüfe das Ergebnis in der Datei
\begin{quote}
	\texttt{uebungsplatz.py}.
\end{quote}
Wir erweitern beispielsweise den Bruch $\frac{3}{4}$ mit dem Faktor 3.

\begin{codePython}
bruch = Bruch(2,3)
faktor = 3
\end{codePython}

Wie geht es weiter? Wie sieht das Ergebnis aus?

\subsection*{Darstellung der Formel im Browser}

Wir können die Formel automatisch in die HTML-Seite übertragen lassen und anschließend im Browser anschauen.

Ergänze in der Funktion
\begin{quote}
	\texttt{schreibeMathML()}
\end{quote}
die entsprechenden Zeilen analog zum Beispiel Kürzen.


Dann führe die Datei
\begin{quote}
\texttt{htmlErzeugung.py}
\end{quote} als Python-Skript aus.

Prüfe im Firefox-Browser, wie die Datei \texttt{index.html} aussieht. Dort sollte eine Zeile stehen, die so aussieht:
\[
\frac{3}{4} = \frac{9}{12}
\]
\section{Aufgabe: Die Formel zum Multiplizieren von Brüchen in MathML}

%\chapter{Weitere Aufgaben}
%
%\section{Aufgabe: Das Dividieren von Brüchen}

Diese Aufgabe ist einfach.

\subsection{Überlege: Wie werden Brüche dividiert?}

Brüche werden dividiert, indem $\dots$

\subsection{Formuliere den Algorithmus!}

\subsection{Programmiere die Funktion, die den Teiler-Bruch umwandelt.}

Das ist eine sehr einfache Funktion.

\subsection{Programmiere jetzt die Division der Brüche.}

Benutze dabei die Funktionen, die du für das Multiplizieren der Brüche programmiert hast.
%\section{Aufgabe: Das Addieren von Brüchen}

Diese Aufgabe ist mittelschwer.

\subsection*{Überlege: Wie werden zwei Brüche addiert?}

Zwei Brüche werden addiert, indem $\dots$

\subsection*{Formuliere den Algorithmus zum Addieren von zwei Brüchen.}

\subsection*{Programmiere die Funktion, die die beiden Brüche für das Addieren vorbereitet.}

Diese Funktion darf sehr einfach sein, weil wir bereits sehr gut Brüche kürzen können.

Könner dürfen eine anspruchsvollere Lösung mit dem kleinsten gemeinsamen Vielfachen (kgV) programmieren!

\subsection*{Programmiere das Addieren von zwei Brüchen.}

Natürlich möchten wir das Ergebnis in MathML im Browser anschauen!

\subsection*{Programmiere das Subtrahieren von zwei Brüchen.}

Wenn wir zwei Brüche addieren können, dann ist das Subtrahieren sehr einfach.
%\section{Aufgabe: Das Umwandeln eines Bruchs in eine Kommmazahl}

Diese Aufgabe ist einfach.

\subsection*{Überlege: Wie wird ein Bruch in eine Kommazahl umgewandelt?}

\subsection*{Formuliere den Algorithmus!}

\subsection*{Programmiere die Funktion zum Umwandeln.}

\subsection*{Zusatz: Eine Datenklasse für Kommazahlen}

Diese Zusatzaufgabe ist mittelschwer.

Grundsätzlich können wir mit dem Datentyp \texttt{float} arbeiten. Schöner ist es, eine eigene Datenklasse für Kommazahlen zu haben, ähnlich zur Datenklasse für Brüche.

Wie sieht eine Datenklasse für eine Kommazahl aus? Bitte nicht wundern - das Ergebnis ist sehr einfach.
%\section{Aufgabe: Die Umwandlung einer Kommazahl in einen Bruch}

Diese Aufgabe ist mittelschwer.

\subsection{Überlege: Wie wird eine Kommazahl in einen Bruch umgewandelt?}

\subsection{Formuliere den Algorithmus.}

\subsection{Programmiere die Funktion zur Umwandlung einer Kommazahl in einen Bruch}

Tipp: Wir haben eine Hilfsfunktion vorbereitet, die den richtigen Nenner berechnet. Vergiss danach das Kürzen nicht.
%\section{Aufgabe: Umwandeln in eine gemischte Zahl}

Diese Aufgabe ist schwer.

Der Wert eines unechten Bruchs, also eines Bruchs, dessen Zähler größer als der Nenner ist, ist oft erst nach etwas Kopfrechnen zu erkennen. Daher schreiben wir unechte Brüche als gemischte Zahl, also als Summe einer ganzen Zahl und einem echten Bruch. Beispiel:
\[
\frac{70}{12} = 5 \frac{5}{6}
\]

Einen echten Bruch können wir ebenfalls als gemischte Zahl schreiben. Dann ist die ganze Zahl einfach gleich Null und wird gar nicht hingeschrieben.

\subsection*{Programmiere die Datenklasse für eine gemischte Zahl}

Tipp: Benutze dafür die Datenklasse \texttt{Bruch}.

\subsection*{Überlege: Wie wird ein Bruch in eine gemischte Zahl umgewandelt?}

\subsection*{Überlege: Wie wird eine gemischte Zahl in einen Bruch umgewandelt?}

\subsection*{Formuliere die beiden Algorithmen zur Umwandlung zwischen Bruch und gemischter Zahl.}

\subsection*{Programmiere die Funktionen zur Umwandlung.}

Sieht deine Darstellung einer gemischten Zahl in MathML schön aus?
%\section{Aufgabe: Objektorientierte Programmierung}

Diese Aufgabe ist sehr schwierig. Es werden gute Kenntnisse der Programmierung in Python vorausgesetzt.

%\chapter{Lösungen}
%
%\section{Lösung: Das Kürzen eines Bruchs}

\subsection*{Der Algorithmus zum Kürzen eines Bruchs}

\begin{itemize}
	\item Nimm einen Bruch.
	\item Merke dir den Zähler des Bruchs in einer Variablen.
	\item Verfahre ebenso mit dem Nenner.
	\item Berechne den größten gemeinsamen Teiler (ggT) von Zähler und Nenner. (Es gibt dafür eine fertige Funktion.)
	\item Berechne den gekürzten Zähler, indem du den alten Zähler durch den ggT teilst. 
	\item Berechne ebenso den gekürzten Nenner.
	\item Erstelle einen neuen Bruch aus dem gekürzten Zähler und dem gekürzten Nenner.
	\item Gib den neuen Bruch zurück.
\end{itemize}

\subsection*{Die Funktion zum Kürzen von Brüchen}

Die Berechnung des ggT erfolgt mittels der Funktion \texttt{mathefunktionen.berechne\_ggT}. In der Funktion zum Kürzen wird also eine Zeile enthalten sein:

\begin{codePython}
ggT = mathefunktionen.berechne_ggT(alterZaehler, alterNenner)
\end{codePython}

Beachte, dass beim Teilen durch den ggT die Division \texttt{//} benutzt wird, damit der neue Zähler und der neue Nenner ganze Zahlen werden!

\begin{codePython}
def kuerzeBruch(bruch: Bruch) -> Bruch:
	alterZaehler = bruch.zaehler
	alterNenner = bruch.nenner
	ggT = mathefunktionen.berechne_ggT(alterZaehler, alterNenner)
	neuerZaehler = alterZaehler // ggT
	neuerNenner = alterNenner // ggT
	neuerBruch = Bruch(neuerZaehler, neuerNenner)
	return neuerBruch
\end{codePython}

\subsection*{Die Algorithums zur Darstellung der Formel zum Kürzen}

\begin{itemize}
	\item Nimm einen Bruch.
	\item Erzeuge aus diesem Bruch einen gekürzten Bruch. Verwende dazu die Funktion, die du gerade programmiert hast.
	\item Erzeuge aus dem ursprünglichen Bruch die Beschreibung in MathML. Verwende dazu die Funktion, die einen Bruch in MathML beschreibt.
	\item Erzeuge aus dem gekürzten Bruch ebenfalls die Beschreibung in MathML. Das geht genauso wie eben.
	\item Setze jetzt den MathML-Text aus den einzelnen Angaben zusammen.
	\item Die Zeichenkette beginnt mit dem MathML-Text für den ungekürzten Bruch.
	\item Dann kommt das Gleichheitszeichen.
	\item Es folge der MathML-Text für den gekürzten Bruch.
\end{itemize}

\subsection*{Die Funktion zur Darstellung der Formel zum Kürzen}

\begin{codePython}
def schreibeKuerzen(bruch: Bruch) -> str:
	bruchGekuerzt = kuerzeBruch(bruch)
	textUngekuerzt = schreibeBruch(bruch)
	textGekuerzt = schreibeBruch(bruchGekuerzt)
	ergebnis = textUngekuerzt + "<mo>=</mo>" + textGekuerzt
	return ergebnis
\end{codePython}

\subsection*{Überprüfung des Ergebnisses}

Überprüfe das Ergebnis in der Datei \texttt{uebungsplatz.py}. 

\begin{codePython}
bruch = Bruch(6,8)
formel = schreibeKuerzen(bruch)
print(formel)
\end{codePython}

Der Bruch $\frac{6}{8}$ muss gekürzt den Bruch $\frac{3}{4}$ ergeben. In einer Zeile steht:

\begin{codeHTML}
<mfrac><mn>6</mn><mn>8</mn></mfrac><mo>=</mo>
	<mfrac><mn>3</mn><mn>4</mn></mfrac>
\end{codeHTML}

Ergänze in der Funktion \texttt{schreibeMathML()} die Zeilen für Kürzen für den Bruch $\frac{6}{8}$.

\begin{codePython}
bruch = Bruch(6,8)
inhalt = inhalt + "\n\t\t<p><math>"
inhalt = inhalt + schreibeKuerzen(bruch)
inhalt = inhalt + "</math></p>"
\end{codePython}

Führe die Datei \texttt{htmlErzeugung.py} aus.

Prüfe im Firefox-Browser, wie die Datei \texttt{index.html} aussieht. Dort sollte eine Zeile stehen, die so aussieht:
\[
\frac{6}{8} = \frac{3}{4}
\]

%\section{Lösung: Das Erweitern eines Bruchs}

\subsection*{Der Algorithmus zum Erweitern}

\begin{itemize}
	\item Nimm einen Bruch und den Faktor, mit dem du den Bruch erweitern willst.
	\item Merke dir den Zähler in einer Variablen.
	\item Merke dir den Nenner des Bruchs in einer zweiten Variablen.
	\item Multipliziere die Variable für den Zähler mit dem Erweiterungsfaktor. Merke dir das Ergebnis in einer Variablen.
	\item Verfahre genauso mit dem Nenner.
	\item Erstelle einen neuen Bruch mit dem neuen Zähler und dem neuen Nenner. Merke dir das Ergebnis in einer Variablen.
	\item Gib die Variable zurück, die den neuen Bruch enthält.
\end{itemize}

\subsection*{Die Funktion zum Erweitern}

\begin{codePython}
def erweitereBruch(bruch: Bruch, faktor: int) -> Bruch:
	alterZaehler = bruch.zaehler
	alterNenner = bruch.nenner
	neuerZaehler = alterZaehler * faktor
	neuerNenner = alterNenner * faktor
	neuerBruch = Bruch(neuerZaehler, neuerNenner)
	return neuerBruch
\end{codePython}

\subsection*{Der Algorithmus zur Darstellung der Erweiterungsformel in MathML}

\begin{itemize}
	\item Nimm einen Bruch und einen Faktor zur Erweiterung entgegen.
	\item Erzeuge aus diesem Bruch einen erweiterten Bruch. Verwende dazu die Funktion \texttt{erweitereBruch}. Merke das Ergebnis in einer Variablen.
	\item Erzeuge aus dem ursprünglichen Bruch die Beschreibung in MathML. Verwende dazu die Funktion \texttt{schreibeBruch}. Merke das Ergebnis in einer Variablen.
	\item Erzeuge aus dem erweiterten Bruch die Beschreibung in MathML. Verwende dazu ebenfalls die Funktion \texttt{schreibeBruch}. Merke das Ergebnis in einer Variablen.
	\item Setze jetzt den MathML-Text aus den einzelnen Angaben zusammen wie in den folgenden Schritten angegeben. Verknüpfe die einzelnen Strings mit dem Pluszeichen.
	\item Es beginnt mit dem MathML-Text für den ungekürzten Bruch.
	\item Dann kommt das Gleichheitszeichen.
	\item Es folgt der MathML-Text für den gekürzten Bruch.
\end{itemize}

\subsection*{Die Funktion zur Darstellung der Formel in MathML}

\lstset{style=syntaxPython}
\begin{lstlisting}
def schreibeErweitern(bruch: Bruch, faktor: int) -> str:
	bruchErweitert = erweitereBruch(bruch, faktor)
	textAlterBruch = schreibeBruch(bruch)
	textNeuerBruch = schreibeBruch(bruchErweitert)
	ergebnis = textAlterBruch + "<mo>=</mo>" + textNeuerBruch
	return ergebnis
\end{lstlisting}

\subsection*{Überprüfung des Ergebnisses}

Überprüfe das Ergebnis in der Datei \texttt{uebungsplatz.py}. Wir erweitern beispielsweise den Bruch $\frac{2}{3}$ mit dem Faktor 3.

\begin{codePython}
bruch = Bruch(2,3)
formel = schreibeErweitern(bruch, 3)
print(formel)
\end{codePython}

Die resultierende Formel lautet (allerdings in einer Zeile):

\begin{codeHTML}
<mfrac><mn>2</mn><mn>3</mn></mfrac><mo>=</mo>
									<mfrac><mn>6</mn><mn>9</mn></mfrac>
\end{codeHTML}

\subsection*{Darstellung der Formel im Browser}

Wir können die Formel automatisch in die HTML-Seite übertragen lassen und anschließend im Browser anschauen.

Ergänze dazu in der Funktion \texttt{schreibeMathML()} folgende Zeilen:

\begin{codePython}
inhalt = inhalt + "\n\t\t<p><math>"
inhalt = inhalt + schreibeErweitern(Bruch(2,3), 3)
inhalt = inhalt + "</math></p>"
\end{codePython}

Jetzt führen wir die Datei \texttt{htmlErzeugung.py} aus.

Prüfe im Firefox-Browser, wie die Datei \texttt{index.html} aussieht. Dort sollte eine Zeile stehen, die so aussieht:
\[
\frac{2}{3} = \frac{6}{9}
\]
%\section{Lösung: Das Multiplizieren von Brüchen}


\end{document}