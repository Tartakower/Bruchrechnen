\documentclass[12p,numbers=noendperiod,DIV=15]{scrreprt}

\usepackage[ngerman]{babel}
\usepackage[utf8]{inputenc}
\usepackage[T1]{fontenc}

% Schriftart DejaVu
\usepackage{dejavu}
\usepackage[onehalfspacing]{setspace}
%\setstretch{1.5}

\setlength{\parindent}{0pt}
\setlength{\parskip}{2ex plus0.5ex minus0.5ex}

\usepackage{enumitem}
\setlist{noitemsep,topsep=0pt}

% Kopf- und Fußzeile
%\pagestyle{headings}
%\usepackage[singlespacing=true]{scrlayer-scrpage}
%\ifoot{\copyright~Norbert Seulberger, 2021}
%\cfoot{}
%\ofoot{\pagemark}
%\pagestyle{scrheadings}
\pagestyle{plain}
%\setkomafont{pageheadfoot}{\small}

% Koma-Script für Inhaltsverzeichnis
\usepackage{tocbasic}
\DeclareTOCStyleEntries[dynnumwidth=true]{tocline}{chapter,section,subsection}

\usepackage{color}
\usepackage{listings}
\usepackage[fleqn]{amsmath}
\usepackage{fdsymbol}
\usepackage{scrhack}
%\usepackage[top=20mm,left=20mm,right=20mm,bottom=30mm]{geometry}
\mathindent=2em

\usepackage{tikz}
\usetikzlibrary{shapes.geometric, arrows, arrows.meta}

\sloppy

%% Umgebungen

\newcounter{regelsatz}[chapter]

\newenvironment{ncsEnvironmentEins}[1]{\stepcounter{regelsatz} \textbf{\arabic{chapter}.\arabic{section}.\arabic{regelsatz} #1} \\}{}
%\newcommand{\aufgabe}[1]{\begin{ncsEnvironmentEins}{Aufgabe: #1} \end{ncsEnvironmentEins}}
\newcommand{\beispiel}[1]{\begin{ncsEnvironmentEins}{Beispiel: #1} \end{ncsEnvironmentEins}}
\newcommand{\hinweis}[1]{\begin{ncsEnvironmentEins}{Hinweis: #1} \end{ncsEnvironmentEins}}
%\newcommand{\loesung}[1]{\begin{ncsEnvironmentEins}{Lösung: #1} \end{ncsEnvironmentEins}}
%\newcommand{\motivation}[1]{\begin{ncsEnvironmentEins}{Motivation: #1} \end{ncsEnvironmentEins}}
\newcommand{\nurTitel}[1]{\begin{ncsEnvironmentEins}{#1} \end{ncsEnvironmentEins}}
%

\newenvironment{ncsEnvironmentZwei}[2]%
{\stepcounter{regelsatz} \textbf{\arabic{chapter}.\arabic{regelsatz} #1} \\ #2}{}
\newcommand{\titelText}[2]{\begin{ncsEnvironmentZwei}{#1}{#2}\end{ncsEnvironmentZwei}}

\newenvironment{ncsItEnvironment}[2]%
{\stepcounter{regelsatz} \textbf{\arabic{chapter}.\arabic{regelsatz} #1: #2} \\ \begin{itshape}}{\end{itshape}}
\newcommand{\begriff}[2]{\begin{ncsItEnvironment}{Begriff}{#1} #2 \end{ncsItEnvironment}}
% Listings Styles

\definecolor{darkred}{rgb}{0.8,0,0}
\definecolor{darkgreen}{rgb}{0.0,0.5,0}

\lstdefinestyle{syntaxZeile}{
	language=python, 				        		
	basicstyle=\ttfamily,
	keywordstyle=\color{darkred}\bfseries,	
	identifierstyle=\color{blue},				
	commentstyle=\color{darkgreen},			
	stringstyle=\ttfamily\color{darkgreen}, 			    		
	showstringspaces=false,	        		
	numbers=none, 	        			
	frame=none, 				        		
	tabsize=2,				            		
	float=htbp,
	captionpos=b}

\lstdefinestyle{syntaxPython}{
	language=python, 				        		
	basicstyle=\linespread{1.2}\ttfamily,
	keywordstyle=\color{darkred}\bfseries,	
	identifierstyle=\color{blue},				
	commentstyle=\color{darkgreen},			
	stringstyle=\ttfamily\color{darkgreen}, 			    		
	showstringspaces=false,
	breaklines=true, 			        		
	numbers=left, 				        		
	numberstyle=\small,		        			
	frame=single, 				        		
	tabsize=2,				            		
	float=htbp,
	captionpos=b,
	aboveskip=2ex,
	belowskip=\medskipamount,
	xleftmargin=.06\textwidth}

\lstdefinestyle{syntaxHTML}{%
	language=HTML,%
	basicstyle=\small\ttfamily,%
	keywordstyle=\color{darkred}\bfseries,%
	identifierstyle=\color{blue},%	
	stringstyle=\color{darkgreen},%
	tabsize=2,%
	showstringspaces=false,%
	numbers=left,%			        		
	numberstyle=\small,%	        			
	frame=single,%
	captionpos=b,%
	xleftmargin=.06\textwidth
} 

\lstset{literate=%
	{Ö}{{\"O}}1
	{Ä}{{\"A}}1
	{Ü}{{\"U}}1
	{ß}{{\ss}}2
	{ü}{{\"u}}1
	{ä}{{\"a}}1
	{ö}{{\"o}}1
}

\lstnewenvironment{codePython}[1]{\lstset{style=syntaxPython,caption=#1}}{}
\lstnewenvironment{codeHTML}[1]{\lstset{style=syntaxHTML,caption=#1}}{}

\usepackage[colorlinks=true,linkcolor=blue]{hyperref}

\begin{document}
	
\title{Mathehelfer Bruchrechnen}
\subtitle{Ein Hackathon für Kinder und Jugendliche}
\author{Norbert Seulberger}
\date{Mai 2023}

\maketitle

\tableofcontents

\chapter{Fahrplan für den Kurs}

%\section*{Der Fahrplan zur Orientierung}

\subsection*{MathML}

\subsubsection*{MathML: Einführung}

Lesen und verstehen!


\subsection*{Algorithmen}


\subsection*{Variablen}


\subsection*{Funktionen}


\subsection*{Beispiel Kürzen}

\subsection*{Aufgabe Erweitern}

\subsection*{Aufgabe Multiplizieren}

\subsection*{Zeichenketten}

\subsection*{Formel Kürzen}

\subsection*{Formel Erweitern}

\subsection*{Formel Multiplizieren}

\chapter{MathML}

\section{MathML}

MathML ist eine einfache Markup-Sprache, die in einem HTML-Dokument dazu verwendet werden kann, um mathematische Formeln darzustellen.

Genauso wie in HTML werden die Daten in sogenannte Tags eingeschlossen, d.h. ein Element beginnt mit einen Tag, dann stehen die Daten und ein schließendes Tag beendet den Ausdruck. 

\subsection{Operanden und Operatoren}

Für die verschiedenen Teile einer mathematischen Formel gibt es unterschiedliche Elemente.
\begin{itemize}
	\item \texttt{<mn>}42\texttt{</mn>} stellt die Zahl 42 dar.
	\item \texttt{<mo>}=\texttt{</mo>} beschreibt ein Gleichheitszeichen.
	\item \texttt{<mo>}+\texttt{</mo>} schreibt ein Pluszeichen.
\end{itemize}

Die Formel $2 + 3 = 5$ lautet also in MathML:
\begin{codeHTML}{Eine einfache Addition}
<mn>2</mn>
<mo>+</mo>
<mn>3</mn>
<mo>=</mo>
<mn>5</mn>
\end{codeHTML}

\nurTitel{Tipp}
Der Browser ignoriert die Zeilenumbrüche in MathML. Wir können die Formel also auch in eine Zeile schreiben.

\begin{codeHTML}{Die Formel in einer Zeile}
<mn>2</mn><mo>+</mo><mn>3</mn><mo>=</mo><mn>5</mn>
\end{codeHTML}

\subsection{Brüche}

\subsection{Eine Formelzeile}

Eine Formel wird wie folgt geschrieben: \texttt{<math>} Hier steht die Formel. \texttt{</math>}

Damit die Formel in eine eigene Zeile geschrieben wird, verwenden wir das Absatz-Tag von HTML: \texttt{<p>} ... \texttt{</p>}

Also zusammen:
\begin{codeHTML}{Eine Formelzeile}
<p>
	<math>
		Hier steht die Formel.
	</math>
</p>
\end{codeHTML}
\section{Aufgabenblatt MathML}

\subsection*{Aufgabe: Anzeige der HTML-Datei}

Schau dir im Firefox-Browser die Datei \texttt{manuell.html} an. Die Anzeige sollte etwa so aussehen:
\[
\frac{6}{8} = \frac{3}{4}
\]

„Hier kannst du die Formeln für das Umrechnen eines Bruchs in eine Dezimalzahl und umgekehrt einfügen.“

\subsection*{Aufgabe: Ansicht der HTML-Datei im Editor}

Schau dir jetzt dieselbe Datei im Editor der Entwicklungsumgebung an. Der Text sieht etwa so aus:

\begin{codeHTML}
<!doctype html>
<html lang="de">
	<head>
		<meta charset="utf-8">
		<meta name="viewport" content="width=device-width, initial-scale=1.0">
		<title>Bruchrechnen</title>
	</head>
	<body>
		<p>
			<math>
				<mfrac>
					<mn>6</mn>
					<mn>8</mn>
				</mfrac>
				<mo>=</mo>
				<mfrac>
					<mn>3</mn>
					<mn>4</mn>
				</mfrac>
			</math>
		</p>
		<p>
 			Hier kannst du die Formeln für das Umrechnen ...
		</p>
	</body>
</html>
\end{codeHTML}

Kannst du die Element entdecken, die für die Darstellung des Kürzens der beiden Brüche stehen?

\subsection*{Aufgabe: Die Umrechung eines Bruchs in eine Dezimalzahl}

Füge in die Datei \texttt{manuell.html} den Text in der MathML-Sprache ein, so dass die Umrechnung eines Bruchs in eine Dezimalzahl dargestellt wird. Der Text „Hier kannst du \dots“ darf überschrieben werden.

Konkret soll die letzte Zeile so aussehen.
\[
\frac{3}{4} = 0.75
\]

Überprüfe das Ergebnis im Firefox-Browser. Die Seite sollte jetzt ungefähr so aussehen:
\begin{align*}
& \frac{6}{8} = \frac{3}{4} & \\[2ex]
& \frac{3}{4} = 0.75 &
\end{align*}

\subsection*{Aufgabe: Die Umrechnung einer Dezimalzahl in einen Bruch}

Füge in die Datei \texttt{manuell.html} den Text in der MathML-Sprache ein, so dass diese Zeile als letzte Zeile angezeigt wird:
\[
0.8 = \frac{4}{5}
\]

Überprüfe das Ergebnis im Firefox-Browser. Die Seite sollte jetzt ungefähr so aussehen:
\begin{align*}
& \frac{6}{8} = \frac{3}{4} & \\[2ex]
& \frac{3}{4} = 0.75 & \\[2ex]
& 0.8 = \frac{4}{5} & 
\end{align*}





\chapter{Algorithmen und Funktionen}

\section{Algorithmen}
\label{sec:Algorithmus}

Ein Algorithmus ist die eindeutige Beschreibung eines Verfahrens, um eine bestimmte Aufgabe zu lösen.

\subsection*{Beispiel: Das Erweitern eines Bruches}

Ein Bruch wird erweitert, indem der Zähler und der Nenner mit demselben Faktor multipliziert werden. Wird beispielsweise der Bruchs $\frac{2}{5}$ mit dem Faktor $3$ erweitert, so ergibt sich diese Gleichung:
\[
\frac{2}{5} = \frac{2 \cdot 3}{5 \cdot 3} = \frac{6}{15}
\]

\subsection*{Der Algorithmus für das Erweitern}

\begin{itemize}
	\item Nimm einen Bruch und den Faktor, mit dem du den Bruch erweitern willst.
	\item Merke dir den Zähler und den Nenner des Bruchs jeweils in einer Variablen.
	\item Multipliziere die Variable für den Zähler mit dem Erweiterungsfaktor.
	\item Verfahre genauso mit dem Nenner.
	\item Erstelle einen neuen Bruch mit dem neuen Zähler und dem neuen Nenner.
\end{itemize}
\section{Variablen}

\subsection*{Einfache Variable}

Eine Variable dient dazu, sich einen Wert, einen Text oder das Ergebnis einer Rechnung zu merken. Wir geben einer Variablen einen sinnvollen Namen.

\begin{codePython}
wichtige_zahl = 42
summe = 2 + 3
\end{codePython}

Mit Variablen können wir auch rechnen.

\begin{codePython}
summe = summe + 6
\end{codePython}

\subsection*{Strukturierte Variablen}

Wir können Variablen auch komplizierter aufbauen und einen Wert in einer Variablen merken, der sich aus mehreren Teilwerten zusammensetzt. Ein Beispiel ist der Bruch:

\begin{codePython}
class Bruch():
	zaehler: int
	nenner: int
\end{codePython}

Ein Bruch ist ein Wert, der sich aus zwei Teilwerten zusammensetzt, nämlich dem Zähler und dem Nenner. Den Bruchstrich müssen wir uns nicht merken, weil er immer da ist. Zähler und Nenner sind ganze Zahlen (\texttt{int}).
%Wir merken uns den Zähler und Nenner zusammen in einem Bruch. Diesen Bruch können wir dann als einen Wert in einer Variablen aufbewahren.

\lstset{style=syntaxPython}
\begin{lstlisting}
bruch = Bruch(3,4)
\end{lstlisting}

Jetzt können wir die Variable, die ja einen Bruch enthält, nach dem Zähler und dem Nenner fragen. Wir schreiben die beiden Werte in eigene Variablen.

\begin{lstlisting}
zaehler = bruch.zaehler
nenner = bruch.nenner
\end{lstlisting}

Wir können auch zwei Ganzzahlen jeweils in eine Variable schreiben und damit eine Bruchvariable erzeugen.

\begin{lstlisting}
neuer_zaehler = 7
neuer_nenner = 8
bruch = Bruch(neuer_zaehler, neuer_nenner)
\end{lstlisting}
\section{Funktionen}

Eine Funktion ist ein kleines Teilprogramm, das eine spezifische Aufgabe löst. Eine Funktion arbeitet mit folgenden Schritten:
\begin{itemize}
	\item Die Funktion nimmt einen oder mehrere Eingabewerte entgegen.
	\item Die Funktion führt eine Berechnung oder andere Aktivität durch.
	\item Die Funktion gibt das Ergebnis der Berechnung zurück.
\end{itemize}

Für die gleichen Eingabewerte erhält man immer dasselbe Ergebnis. Die Eingabewerte heißen Parameter. Es kann einen oder mehrere Parameter geben, manchmal sogar gar keinen.

\subsection*{Der Aufbau einer Funktion in Python}

Der Aufbau einer Funktion orientiert sich an den drei Schritte, die wir eben kennen gelernt haben:
\begin{itemize}
	\item die Signaturzeile
	\item der Rumpf der Funktion 
	\item der Rückgabewert
\end{itemize}

\subsection*{Die Signaturzeile}

Die Signaturzeile nennt den Namen und beschreibt die Parameter.
\begin{codePython}
def kuerzeBruch(bruch):
\end{codePython}

\begin{itemize}
	\item Zuerst steht immer das Wort \texttt{def}.
	\item Dann folgt der Name der Funktion. Der Name sollte ausdrücken, welche Berechnung die Funktion durchführt.
	\item In den runden Klammer stehen die Eingabewerte für die Funktion. Hier ist es ein Parameter mit dem Namen \texttt{bruch}. Die Funktion kann den Wert des Parameters für die Berechnung nutzen.
\end{itemize}

\subsection*{Tipp: Datentypen angeben}

Wer möchte, kann die Typen der Parameter und des Ergebnisses angeben. Es hilft, Fehler zu vermeiden. Programmierprofis tun das.

Die Signaturzeile für die Funktion zum Kürzen lautet dann:

\lstset{style=syntaxPython}
\begin{lstlisting}
def kuerzeBruch(bruch: Bruch) -> Bruch:
\end{lstlisting}

Die Funktion nimmt also einen Parameter entgegen, dessen Name \texttt{bruch} lautet (klein geschrieben!) und dessen Datentyp ein \texttt{Bruch} ist (groß geschrieben!).

Die Funktion führt eine Berechnung durch, bei der ein Rückgabewert berechnet wird, der vom Datentyp \texttt{Bruch} ist. Das steht hier: \texttt{-> Bruch}.

\subsection*{Der Rumpf einer Funktion}

Der Rumpf der Funktion ist eingerückt. Wir benutzen dafür vier Leerzeichen.

Hier wird die Berechnung durchgeführt. Das erklären wir ausführlich im nächsten Beispiel.


\subsection*{Der Rückgabewert}

Die letzte Zeile im Rumpf gibt das Ergebnis zurück. Sie beginnt mit dem Wort
\begin{quote}
\texttt{return}
\end{quote}


\section{Beispiel: Das Kürzen eines Bruchs}

Im folgenden Beispiel lernst du das Programmieren einer Funktion in Python.

\subsection*{Aufgabe: Der Algorithmus zum Kürzen eines Bruchs}

Lies den Algorithmus zum Kürzen eines Bruchs durch:

\begin{itemize}
	\item Nimm einen Bruch.
	\item Merke dir den Zähler des Bruchs in einer Variablen.
	\item Merke dir den Nenner des Bruchs in einer anderen Variablen.
	\item Berechne den größten gemeinsamen Teiler (ggT) von Zähler und Nenner. Es gibt dafür eine fertige Funktion.
	\item Berechne den gekürzten Zähler, indem du den alten Zähler durch den ggT teilst. 
	\item Berechne ebenso den gekürzten Nenner.
	\item Erstelle einen neuen Bruch aus dem gekürzten Zähler und dem gekürzten Nenner.
	\item Gib den neuen Bruch zurück.
\end{itemize}

Verstehst du das Vorgehen? Andernfalls stelle deine Fragen dem Betreuer.

\subsection*{Der größte gemeinsame Teiler zweier Zahlen (ggT)}
\label{sec:FunktionGgT}

Die Berechnung des ggT erfolgt mittels der Funktion
\begin{quote}
\texttt{mathefunktionen.berechne\_ggT}.
\end{quote}

In der Funktion zum Kürzen wird also eine Zeile enthalten sein:

\begin{codePython}
ggT = mathefunktionen.berechne_ggT(alterZaehler, alterNenner)
\end{codePython}

\pagebreak

\subsection*{Die Funktion zum Kürzen von Brüchen}
\label{sec:FunktionKuerzen}

Beachte, dass beim Teilen durch den ggT die Division \texttt{//} benutzt wird, damit der neue Zähler und der neue Nenner ganze Zahlen werden!

\begin{codePython}
def kuerzeBruch(alterBruch: Bruch) -> Bruch:
	alterZaehler = alterBruch.zaehler
	alterNenner = alterBruch.nenner
	ggT = mathefunktionen.berechne_ggT(alterZaehler, alterNenner)
	neuerZaehler = alterZaehler // ggT
	neuerNenner = alterNenner // ggT
	neuerBruch = Bruch(neuerZaehler, neuerNenner)
	return neuerBruch
\end{codePython}

\subsection*{Aufgabe: Programmieren der Funktion}

Wir bearbeiten jetzt die Datei

\begin{quote}
\textbf{\texttt{mathehelfer.py}}
\end{quote}

hinter der Zeile

\begin{quote}
\textbf{\texttt{\# Ab hier dürft ihr Änderungen vornehmen.}}
\end{quote}

Gib die Funktion in den Editor der Entwicklungsumgebung (IDE) ein. Zeigt die IDE noch Fehler an? Vielleicht hast du dich vertippt? Falls du den Fehler nicht finden kannst, sprich den Betreuer an.

\subsection*{Überprüfung des Ergebnisses}

Wir überprüfen das Ergebnis mit Hilfe folgender Datei:

\begin{quote}
\textbf{\texttt{uebungsplatz.py}}
\end{quote}

Wir tragen dort die folgenden Zeilen ein:

\begin{codePython}
bruch = Bruch(6,8)
gekuerzter_bruch = mathehelfer.kuerzeBruch(bruch)
print(gekuerzter_bruch)
\end{codePython}

\subsection*{Ausführen eines Python-Programms}

Jetzt führen wir \texttt{uebungsplatz.py} als Python-Programm aus. Das geht in VS Code auf zwei verschiedene Weisen:

\begin{itemize}
	\item Im Terminal: \texttt{python3 uebungsplatz.py}
	\item Klicke mit der Maus rechts oben in der Ecke auf $\medtriangleright$ und wähle \textit{Führen Sie die Python-Datei aus}.
\end{itemize}


Das Ergebnis sollte jetzt diese Zeile enthalten:
\begin{quote}
\texttt{Bruch(zaehler=3, nenner=4)}
\end{quote}

Sieht das Ergebnis bei dir genauso aus?


\section{Aufgabe: Das Erweitern eines Bruchs}

\subsection*{Der Algorithmus zum Erweitern}

Wie lautet der Algorithmus zur Erweitern eines Bruchs mit einem vorgegebenen Faktor?

{\huge
\begin{itemize}
	\item Nimm einen Bruch und den Faktor, mit dem du den Bruch erweitern willst.
	\item  
	\item  
	\item  
	\item 
	\item  
	\item  
\end{itemize}
}

\subsection*{Die Funktion zum Erweitern}

Programmiere die Funktion zum Erweitern eines Bruchs. Die Funktion erhält als Parameter einen Bruch sowie den Faktor zum Erweitern. Die Signatur lautet:

\begin{codePython}
def erweitereBruch(bruch: Bruch, faktor: int) -> Bruch:
\end{codePython}

Vergiss nicht die letzte Zeile mit dem \texttt{return}-Befehl!

\subsection*{Überprüfung des Ergebnisses}

Überprüfe das Ergebnis. Trage dazu die notwendige Befehle in die Datei
\begin{quote}
	\texttt{uebungsplatz.py}
\end{quote}
ein. Du kannst dich daran orientieren, wie wir es beim Kürzen gemacht haben. Hier möchten wir den Bruch $\frac{3}{4}$ mit dem Faktor 3 erweitern.

Es geht los mit

\begin{codePython}
	bruch = Bruch(3,4)
\end{codePython}

Wie geht es weiter?

Danach führen wir \texttt{uebungsplatz.py} als Python-Skript aus. Das Ergebnis sollte jetzt diese Zeile enthalten:
\begin{quote}
	\texttt{Bruch(zaehler=9, nenner=12)}
\end{quote}

Sieht das Ergebnis bei dir genauso aus?

\section{Aufgabe: Das Multiplizieren von Brüchen}

\subsection*{Der Algorithmus zum Multiplizieren von Brüchen}

Wie lautet der Algorithmus zum Multiplizieren von Brüchen?

\begin{itemize}[itemsep=2ex]
	\item Nimm zwei Brüche entgegen.
	\item 
	\item 
	\item 
	\item 
	\item 
	\item  
	\item 
	\item 
\end{itemize}

Hast du daran gedacht, den neuen Bruch zu kürzen?

\subsection*{Die Funktion zum Multiplizieren von Brüchen}

Wie lautet die Funktion?

\subsection*{Überprüfung der Funktion zum Multiplizieren}

Wir möchten in der Datei
\begin{quote}
	\texttt{uebungsplatz.py}
\end{quote}
die beiden Brüche $\frac{3}{4}$ und $\frac{2}{3}$ miteinander multiplizieren. Es beginnt also mit

\begin{codePython}
erster_faktor = Bruch(3,4)
zweiter_faktor = Bruch(2,3)
\end{codePython}

Wie geht es weiter?

Führe \texttt{uebungsplatz.py} als Python-Skript aus. Die Ausgabe muss folgende Zeile enthalten:
\begin{quote}
	Produkt: Bruch(zaehler=1, nenner=2)
\end{quote}

\chapter{Zeichenketten}
\section{Zeichenketten}

\subsection*{Was ist eine Zeichenkette?}

Eine Zeichenkette enthält ein oder mehrere beliebige Zeichen, also Buchstaben, Zahlen oder Sonderzeichen. Die Zeichen stehen hintereinander. Eine Zeichenkette wird eingeschlossen in Hochkommata (') oder hochgestellte Anführungszeichen (").

\begin{codePython}
eine_zeichenkette = "r2d2"
berg = '8000er-Gipfel'
ein_ganzer_satz = "Python ist toll!"
\end{codePython}

Häufig benutzen wir den englischen Begriff für Zeichenkette: String. Mit Strings kann man viele interessante Dinge anstellen.

\subsection*{Das Zusammenfügen von Zeichenketten}

In Python können zwei Strings mit dem Pluszeichen zu einem langen String zusammengefügt werden:

\begin{codePython}
gruss = "Hallo " + "Welt!"
\end{codePython}

In \texttt{gruss} steht jetzt der String \texttt{"Hallo Welt!"}.

\subsection*{Tipp: Das Pluszeichen setzt Zeichenketten zusammen!}

Das Pluszeichen bedeutet in diesem Fall nicht das Addieren von Zahlen, sondern das Zusammenfügen von Zeichenketten. Daran gewöhnt man sich schnell!

\subsection*{MathML als Zeichenkette}

Wir möchten einen String zusammenbauen, der in MathML eine Rechenformel beschreibt.

Die Formel
\[
2 + 3 = 5
\]
lautet in MathML:

\begin{codeHTML}
<math><mn>2</mn><mo>+</mo><mn>3</mn><mo>=</mo><mn>5</mn></math>
\end{codeHTML}

Diese Formel können wir in Python schrittweise aus kleinen Strings zu einem langen String zusammensetzen. 

\lstset{style=syntaxPython}
\begin{lstlisting}
formel = "<math>"
formel = formel + "<mn>2</mn>"
formel = formel + "<mo>+</mo>"
formel = formel + "<mn>3</mn>"
formel = formel + "<mo>=</mo>"
formel = formel + "<mn>5</mn>"
formel = formel + "</math>"
\end{lstlisting}

Dabei nehmen wir die bereits erstellte Formel, fügen den nächsten kurzen String an und merken uns das Ergebnis wieder in der Variablen \texttt{formel}.
\section{Beispiel: Die Darstellung eines Bruchs in MathML}

Einen Bruch in MathML zu schreiben, ist nicht schwierig, aber mühsam. Daher möchten wir eine Funktion ein Python programmieren, die das automatisch erledigt.

\subsection*{Der Algorithmus zur Darstellung eines Bruchs in MathML}

\begin{itemize}
	\item Nimm einen Bruch entgegen.
	\item Wandle den Zähler des Bruchs in eine Zeichenkette und speichere diese Zeichenkette in einer Variablen.
	\item Verfahre genauso mit dem Nenner.
	\item Baue die Zeichenkette für das Ergebnis aus Einzelteilen zusammen, wie in den nächsten Schritten beschrieben.
	\item Beginne die Zeichenkette mit \texttt{<mfrac>}.
	\item Rahme die Zeichenkette, die den Zähler beschreibt, mit den Tags \texttt{<mn>} und \texttt{</mn>} ein.
	\item Verfahre ebenso mit der Zeichenkette für den Nenner.
	\item Beende die Zeichenkette mit \texttt{</mfrac>}.
	\item Gib die Zeichenkette zurück.
\end{itemize}

\subsection*{Die Funktion zur Darstellung eines Bruchs in MathML}
\label{sec:MathMLBruch}

\begin{codePython}
def schreibeBruch(bruch: Bruch) -> str:	
	zaehlerAlsString = str(bruch.zaehler)
	nennerAlsString = str(bruch.nenner)
	ergebnis = "<mfrac>"
	ergebnis = ergebnis + "<mn>" + zaehlerAlsString + "</mn>"
	ergebnis = ergebnis + "<mn>" + nennerAlsString + "</mn>"
	ergebnis = ergebnis + "</mfrac>"
	return ergebnis
\end{codePython}

\chapter{Das Erstellen der Formeln}

\section{Aufgabe: Die Formel zum Kürzen eines Bruchs in MathML}

\subsection*{Die Vorarbeiten}

Hast du die Funktion zum Kürzen eines Bruchs bereit? Wie heißt sie?

Außerdem benötigst du wieder die Funktion zur Darstellung eines Bruchs in MathML.

\subsection*{Der Algorithmus}

Wie lautet der Algorithmus für die Formeldarstellung des Kürzens?

{\huge
\begin{itemize}
	\item Nimm einen Bruch entgegen. 
	\item  
	\item  
	\item  
	\item  
	\item  
\end{itemize}
}


\subsection*{Die Funktion}

Die Signatur der Funktion lautet:

\begin{codePython}
def schreibeKuerzen(bruch: Bruch) -> str:
\end{codePython}

Wie geht es weiter?

\subsection*{Überprüfung des Ergebnisses}

Überprüfe das Ergebnis in der Datei \texttt{uebungsplatz.py}. Der Bruch $\frac{6}{8}$ muss gekürzt den Bruch $\frac{3}{4}$ ergeben.

\begin{codeHTML}
<mfrac><mn>6</mn><mn>8</mn></mfrac><mo>=</mo>
	<mfrac><mn>3</mn><mn>4</mn></mfrac>
\end{codeHTML}

\subsection*{Darstellung der Formel im Browser}

Ergänze in der Funktion \texttt{schreibeMathML()} die Zeilen für Kürzen für den Bruch $\frac{6}{8}$ analog zum Erweitern.

Führe die Datei \texttt{htmlErzeugung.py} aus.

Prüfe im Firefox-Browser, wie die Datei \texttt{index.html} aussieht. Dort sollte eine Zeile stehen, die so aussieht:
\[
\frac{6}{8} = \frac{3}{4}
\]
\section{Beispiel: Die Formel zum Erweitern eines Bruchs in MathML}

\subsection*{Erledigte Vorarbeiten}

\begin{itemize}
	\item Wir verfügen über die Funktion \texttt{erweitereBruch}, die zu einem übergebenen Bruch und einem Faktor den erweiterten Bruch berechnet. (siehe \ref{sec:FunktionErweitern})
	\item Außerdem haben wir die Funktion \texttt{schreibeBruch}, die einen Bruch in MathML darstellt. (siehe \ref{sec:MathMLBruch})
\end{itemize}

\subsection*{Der Algorithmus zur Darstellung der Erweiterungsformel in MathML}

\begin{itemize}
	\item Nimm einen Bruch und einen Faktor zur Erweiterung entgegen.
	\item Erzeuge aus diesem Bruch einen erweiterten Bruch. Verwende dazu die Funktion \texttt{erweitereBruch}. Merke das Ergebnis in einer Variablen.
	\item Erzeuge aus dem ursprünglichen Bruch die Beschreibung in MathML. Verwende dazu die Funktion \texttt{schreibeBruch}. Merke das Ergebnis in einer Variablen.
	\item Erzeuge aus dem erweiterten Bruch die Beschreibung in MathML. Verwende dazu ebenfalls die Funktion \texttt{schreibeBruch}. Merke das Ergebnis in einer Variablen.
	\item Setze jetzt den MathML-Text aus den einzelnen Angaben zusammen wie in den folgenden Schritten angegeben:
	\item Es beginnt mit dem MathML-Text für den ungekürzten Bruch.
	\item Dann kommt das Gleichheitszeichen.
	\item Es folgt der MathML-Text für den gekürzten Bruch.
\end{itemize}

\subsection*{Die Funktion zur Darstellung der Formel in MathML}

\begin{codePython}
def schreibeErweitern(bruch: Bruch, faktor: int) -> str:
	bruchErweitert = erweitereBruch(bruch, faktor)
	textAlterBruch = schreibeBruch(bruch)
	textNeuerBruch = schreibeBruch(bruchErweitert)
	ergebnis = textAlterBruch + "<mo>=</mo>" + textNeuerBruch
	return ergebnis
\end{codePython}

\subsection*{Überprüfung des Ergebnisses}

Überprüfe das Ergebnis in der Datei \texttt{spielwiese.py}. Wir erweitern beispielsweise den Bruch $\frac{2}{3}$ mit dem Faktor 3.

\begin{codePython}
bruch = Bruch(2,3)
formel = schreibeErweitern(bruch, 3)
print(formel)
\end{codePython}

Die resultierende Formel lautet (allerdings in einer Zeile):

\begin{codeHTML}
<mfrac><mn>2</mn><mn>3</mn></mfrac><mo>=</mo>
									<mfrac><mn>6</mn><mn>9</mn></mfrac>
\end{codeHTML}

\subsection*{Darstellung der Formel im Browser}

Wir können die Formel automatisch in die HTML-Seite übertragen lassen und anschließend im Browser anschauen.

Ergänze dazu in der Funktion \texttt{schreibeMathML()} folgende Zeilen:

\begin{codePython}
inhalt = inhalt + "\n\t\t<p><math>"
inhalt = inhalt + schreibeErweitern(Bruch(2,3), 3)
inhalt = inhalt + "</math></p>"
\end{codePython}

Jetzt führen wir die Datei \texttt{htmlErzeugung.py} aus.

Prüfe im Firefox-Browser, wie die Datei \texttt{index.html} aussieht. Dort sollte eine Zeile stehen, die so aussieht:
\[
\frac{2}{3} = \frac{6}{9}
\]
\section{Aufgabe: Die Formel zum Multiplizieren von Brüchen in MathML}

%\chapter{Weitere Aufgaben}
%
%\section{Aufgabe: Das Dividieren von Brüchen}

Diese Aufgabe ist einfach.

\subsection*{Überlege: Wie werden Brüche dividiert?}

Brüche werden dividiert, indem $\dots$

\subsection*{Formuliere den Algorithmus!}

\subsection*{Programmiere die Funktion, die den Teiler-Bruch umwandelt.}

Das ist eine sehr einfache Funktion.

\subsection*{Programmiere jetzt die Division der Brüche.}

Benutze dabei die Funktionen, die du für das Multiplizieren der Brüche programmiert hast.
%\section{Aufgabe: Das Addieren von Brüchen}
%\section{Aufgabe: Das Umwandeln eines Bruchs in eine Kommmazahl}

Diese Aufgabe ist einfach.

\subsection{Überlege: Wie wird ein Bruch in eine Kommazahl umgewandelt?}

\subsection{Formuliere den Algorithmus!}

\subsection{Programmiere die Funktion zum Umwandeln.}

\subsection{Zusatz: Eine Datenklasse für Kommazahlen}

Diese Zusatzaufgabe ist mittelschwer.

Grundsätzlich können wir mit dem Datentyp \texttt{float} arbeiten. Schöner ist es, eine eigene Datenklasse für Kommazahlen zu haben, ähnlich zur Datenklasse für Brüche.

Wie sieht eine Datenklasse für eine Kommazahl aus? Bitte nicht wundern - das Ergebnis ist sehr einfach.
%\section{Aufgabe: Die Umwandlung einer Kommazahl in einen Bruch}

Diese Aufgabe ist mittelschwer.

\subsection{Überlege: Wie wird eine Kommazahl in einen Bruch umgewandelt?}

\subsection{Formuliere den Algorithmus.}

\subsection{Programmiere die Funktion zur Umwandlung einer Kommazahl in einen Bruch}

Tipp: Wir haben eine Hilfsfunktion vorbereitet, die den richtigen Nenner berechnet. Vergiss danach das Kürzen nicht.
%\section{Aufgabe: Umwandeln in eine gemischte Zahl}

Diese Aufgabe ist schwer.

Der Wert eines unechten Bruchs, also eines Bruchs, dessen Zähler größer als der Nenner ist, ist oft erst nach etwas Kopfrechnen zu erkennen. Daher schreiben wir unechte Brüche als gemischte Zahl, also als Summe einer ganzen Zahl und einem echten Bruch. Beispiel:
\[
\frac{70}{12} = 5 \frac{5}{6}
\]

Einen echten Bruch können wir ebenfalls als gemischte Zahl schreiben. Dann ist die ganze Zahl einfach gleich Null und wird gar nicht hingeschrieben.

\subsection{Programmiere die Datenklasse für eine gemischte Zahl}

Tipp: Benutze dafür die Datenklasse \texttt{Bruch}.

\subsection{Überlege: Wie wird ein Bruch in eine gemischte Zahl umgewandelt?}

\subsection{Überlege: Wie wird eine gemischte Zahl in einen Bruch umgewandelt?}

\subsection{Formuliere die beiden Algorithmen zur Umwandlung zwischen Bruch und gemischter Zahl.}

\subsection{Programmiere die Funktionen zur Umwandlung.}

Sieht deine Darstellung einer gemischten Zahl in MathML schön aus?
%\section{Aufgabe: Objektorientierte Programmierung}

Diese Aufgabe ist sehr schwierig. Es werden gute Kenntnisse der Programmierung in Python vorausgesetzt.

\subsection*{Ziel der Aufgabe}

Wir möchten nur noch \textbf{eine} Funktion haben, die das Ergebnis einer Berechnung von zwei Zahlen liefert, unabhängig davon, ob es Brüche, Kommazahlen oder gemischte Zahlen sind, auch wild durcheinander. Das ist sehr anspruchsvoll!

Tipp: Intern rechnen wir alle Zahlen in Brüche um und rechen dann mit Brüchen.

\subsection*{Die Datenklassen für Bruch, Kommazahl, gemischte Zahl}

Betrachte die Datenklassen für Bruch, Kommazahl, gemischte Zahl. Welche Funktionen, die wir bisher programmiert haben, können wir als Methoden in die Klassen ziehen?

\subsection*{Der Operator}

Wie können wir den Operator als Klasse beschreiben? Tipp: Es gibt den Aufzählungstyp \texttt{enum}.

\subsection*{Die Superklasse}

Programmiere eine Superklasse für Bruch, Kommazahl und gemischte Zahl. Diese Superklasse sollte eine abstrakte Methode deklarieren, die ein Objekt in einen Bruch umwandelt.

Implementiere in den drei Subklassen diese Methode.

%\chapter{Lösungen}
%
%\section{Lösung: Das Kürzen von Brüchen}

Lies dieses Beispiel sorgfältig durch. Es dient als Vorlage für die Aufgaben, die du danach selbstständig lösen sollst.

\subsection{Der Algorithmus}

Das Kürzen eines Bruches funktioniert so:
\begin{itemize}
	\item Nimm einen Bruch.
	\item Merke dir den Zähler und den Nenner des Bruchs.
	\item Berechne den größten gemeinsamen Teiler von Zähler und Bruch. (Es gibt dafür eine fertige Funktion.)
	\item Berechne den gekürzten Zähler.
	\item Berechne den gekürzten Nenner.
	\item Erstelle einen neuen Bruch aus dem gekürzten Zähler und dem gekürzten Nenner.
\end{itemize}

\subsection{Die Funktion zum Kürzen von Brüchen}

Die Funktion lautet dann:

\begin{codePython}{Funktion zum Kürzen von Brüchen}
def kuerzeBruch(bruch: Bruch) -> Bruch:
	alterZaehler = bruch.zaehler
	alterNenner = bruch.nenner
	kuerzungsFaktor = ggT(alterZaehler, alterNenner)
	neuerZaehler = alterZaehler // kuerzungsFaktor
	neuerNenner = alterNenner // kuerzungsFaktor
	neuerBruch = Bruch(neuerZaehler, neuerNenner)
	return neuerBruch
\end{codePython}

\subsection{Die Funktion zur Darstellung der Rechnung zum Kürzen}

Der Algorithmus lautet:
\begin{itemize}
	\item Nimm einen Bruch.
	\item Erzeuge aus diesem Bruch einen gekürzten Bruch. Verwende dazu die Funktion, die du eben gesehen hast.
	\item Erzeuge aus dem ursprünglichen Bruch die Beschreibung in MathML. Verwende dazu die Funktion, die du im vorherigen Beispiel gesehen hast.
	\item Erzeuge aus dem gekürzten Bruch die Beschreibung in MathML. Verwende dazu ebenfalls die Funktion aus dem vorherigen Beispiel.
	\item Setze jetzt den MathML-Text aus den einzelnen Angaben zusammen.
	\item Die Zeichenkette beginnt mit <math>.
	\item Es folgt der MathML-Text für den ungekürzten Bruch.
	\item Dann kommt das Gleichheitszeichen.
	\item Es folge der MathML-Text für den gekürzten Bruch.
	\item Die Zeichenkette beginnt mit </math>
\end{itemize}

Die Funktion lautet dann:
\begin{codePython}{MathML für das Kürzen eines Bruchs}{code:schreibeKuerzen}
def schreibeKuerzen(bruch: Bruch) -> str:
	bruchGekuerzt = kuerzeBruch(bruch)
	textUngekuerzt = schreibeBruch(bruch)
	textGekuerzt = schreibeBruch(bruchGekuerzt)
	return "<math>" + textUngekuerzt + "<mo>=</mo>" + textGekuerzt + "</math>"
\end{codePython}

\subsection{Überprüfung des Ergebnisses}

Wir überprüfen das Ergebnis analog zum Vorgehen im Beispiel für die Darstellung eines Bruchs. Dazu ergänzen wir in der Funktion \texttt{schreibeMathML()} die
Zeilen

\begin{codePython}{Integration des Kürzens}
bruchUngekuerzt: Bruch = Bruch(6,8)
inhalt += "\n\t\t<p>"
inhalt += schreibeKuerzen(bruchUngekuerzt)
inhalt += "</p>"
\end{codePython}
%\section{Lösung: Das Erweitern eines Bruchs}

\subsection*{Der Algorithmus zum Erweitern}

\begin{itemize}
	\item Nimm einen Bruch und den Faktor, mit dem du den Bruch erweitern willst.
	\item Merke dir den Zähler in einer Variablen.
	\item Merke dir den Nenner des Bruchs in einer zweiten Variablen.
	\item Multipliziere die Variable für den Zähler mit dem Erweiterungsfaktor. Merke dir das Ergebnis in einer Variablen.
	\item Verfahre genauso mit dem Nenner.
	\item Erstelle einen neuen Bruch mit dem neuen Zähler und dem neuen Nenner. Merke dir das Ergebnis in einer Variablen.
	\item Gib die Variable zurück, die den neuen Bruch enthält.
\end{itemize}

\subsection*{Die Funktion zum Erweitern}

\begin{codePython}
def erweitereBruch(bruch, faktor):
	alterZaehler = bruch.zaehler
	alterNenner = bruch.nenner
	neuerZaehler = alterZaehler * faktor
	neuerNenner = alterNenner * faktor
	neuerBruch = Bruch(neuerZaehler, neuerNenner)
	return neuerBruch
\end{codePython}

\subsection*{Der Algorithmus zur Darstellung der Erweiterungsformel in MathML}

\begin{itemize}
	\item Nimm einen Bruch und einen Faktor zur Erweiterung entgegen.
	\item Erzeuge aus diesem Bruch einen erweiterten Bruch. Verwende dazu die Funktion \texttt{erweitereBruch}. Merke das Ergebnis in einer Variablen.
	\item Erzeuge aus dem ursprünglichen Bruch die Beschreibung in MathML. Verwende dazu die Funktion \texttt{schreibeBruch}. Merke das Ergebnis in einer Variablen.
	\item Erzeuge aus dem erweiterten Bruch die Beschreibung in MathML. Verwende dazu ebenfalls die Funktion \texttt{schreibeBruch}. Merke das Ergebnis in einer Variablen.
	\item Setze jetzt den MathML-Text aus den einzelnen Angaben zusammen wie in den folgenden Schritten angegeben. Verknüpfe die einzelnen Strings mit dem Pluszeichen.
	\item Es beginnt mit dem MathML-Text für den ungekürzten Bruch.
	\item Dann kommt das Gleichheitszeichen.
	\item Es folgt der MathML-Text für den gekürzten Bruch.
\end{itemize}

\subsection*{Die Funktion zur Darstellung der Formel in MathML}

\lstset{style=syntaxPython}
\begin{lstlisting}
def schreibeErweitern(bruch: Bruch, faktor: int) -> str:
	bruchErweitert = erweitereBruch(bruch, faktor)
	textAlterBruch = schreibeBruch(bruch)
	textNeuerBruch = schreibeBruch(bruchErweitert)
	ergebnis = textAlterBruch + "<mo>=</mo>" + textNeuerBruch
	return ergebnis
\end{lstlisting}

\subsection*{Überprüfung des Ergebnisses}

Überprüfe das Ergebnis in der Datei \texttt{uebungsplatz.py}. Wir erweitern beispielsweise den Bruch $\frac{2}{3}$ mit dem Faktor 3.

\begin{codePython}
bruch = Bruch(2,3)
formel = schreibeErweitern(bruch, 3)
print(formel)
\end{codePython}

Die resultierende Formel lautet (allerdings in einer Zeile):

\begin{codeHTML}
<mfrac><mn>2</mn><mn>3</mn></mfrac><mo>=</mo>
									<mfrac><mn>6</mn><mn>9</mn></mfrac>
\end{codeHTML}

\subsection*{Darstellung der Formel im Browser}

Wir können die Formel automatisch in die HTML-Seite übertragen lassen und anschließend im Browser anschauen.

Ergänze dazu in der Funktion \texttt{schreibeMathML()} folgende Zeilen:

\begin{codePython}
inhalt = inhalt + "\n\t\t<p><math>"
inhalt = inhalt + schreibeErweitern(Bruch(2,3), 3)
inhalt = inhalt + "</math></p>"
\end{codePython}

Jetzt führen wir die Datei \texttt{htmlErzeugung.py} aus.

Prüfe im Firefox-Browser, wie die Datei \texttt{index.html} aussieht. Dort sollte eine Zeile stehen, die so aussieht:
\[
\frac{2}{3} = \frac{6}{9}
\]
%\section{Lösung: Das Multiplizieren von Brüchen}

\subsection*{Der Algorithmus zum Multiplizieren von Brüchen}

\begin{itemize}
	\item Nimm zwei Brüche entgegen.
	\item Merke dir den Zähler des ersten Bruchs in einer Variablen.
	\item Merke dir den Nenner des ersten Bruchs in einer zeiten Variablen.
	\item Verfahre ebenso mit dem zweiten Bruch. Bisher hast du vier Variablen.
	\item Multipliziere die beiden Zähler-Variablen und merke dir das Ergebnis in einer Variablen.
	\item Multipliziere die beiden Nenner-Variablen.
	\item Erzeuge einen Bruch aus den Produkten der Zähler und der Nenner und merke dir das Ergebnis in einer Variablen. 
	\item Wichtig! Kürze den soeben erstellten Bruch.
	\item Gib den neuen, gekürzten Bruch zurück.
\end{itemize}

\subsection*{Die Funktion zum zum Multiplizieren von Brüchen}

\begin{codePython}
def multipliziereBrueche(erster_faktor: Bruch, zweiter_faktor: Bruch) -> Bruch:
	erster_zaehler = erster_faktor.zaehler
	erster_nenner = erster_faktor.nenner
	zweiter_zaehler = zweiter_faktor.zaehler
	zweiter_nenner = zweiter_faktor.nenner
	produkt_zaehler = erster_zaehler * zweiter_zaehler
	produkt_nenner = erster_nenner * zweiter_nenner
	produkt = Bruch(produkt_zaehler, produkt_nenner)
	produkt = kuerzeBruch(produkt)
	return produkt
\end{codePython}

\subsection*{Überprüfung der Funktion}

Ergänze in der Datei \texttt{uebungsplatz.py}:

\begin{codePython}
	erster_faktor = Bruch(3,4)
	zweiter_faktor = Bruch(2,3)
	produkt = mathehelfer.multipliziereBrueche(erster_faktor, zweiter_faktor)
	print("Produkt: " + str(produkt))
\end{codePython}

Führe \texttt{uebungsplatz.py} als Python-Skript aus. Die Ausgabe muss folgende Zeile enthalten:
\begin{quote}
	Produkt: Bruch(zaehler=1, nenner=2)
\end{quote}

\subsection*{Der Algorithmus zur Darstellung der Multiplikationsformel}

\begin{itemize}
	\item Nimm zwei Brüche entgegen.
	\item Berechne das Produkt der beiden Brüche mit der Funktion \texttt{multipliziereBrueche}.
	\item Erstelle für den ersten Bruch die MathML-Darstellung.
	\item Erstelle für den zweiten Bruch die MathML-Darstellung.
	\item Erstelle für das Produkt die MathML-Darstellung.
	\item Setze jetzt den MathML-Text aus den drei MathML-Texten der Brüche zusammen.
	\item Zwischen dem ersten und dem zweiten Bruch steht noch das Multiplikationszeichen.
	\item Zwischen dem zweiten Bruch und dem Produkt steht das Gleichheitszeichen.
	\item Gib den gesamten MathML-Text zurück.
\end{itemize}

\subsection*{Die Funktion zur Darstellung der Multiplikationsformel}

\begin{codePython}
def schreibeMultiplikation(erster_faktor: Bruch, zweiter_faktor: Bruch) -> str:
	text_erster_faktor = schreibeBruch(erster_faktor)
	text_zweiter_faktor = schreibeBruch(zweiter_faktor)
	produkt = multipliziereBrueche(erster_faktor, zweiter_faktor)
	text_produkt = schreibeBruch(produkt)
	ergebnis = text_erster_faktor + "<mo>&middot;</mo>" + text_zweiter_faktor + "<mo>=</mo>" + text_produkt
	return ergebnis
\end{codePython}

\subsection*{Überprüfung der Darstellung}

Ergänze in der Funktion \texttt{mathehelfer.schreibeMathML()} die Zeilen für das Multiplizieren:

\begin{codePython}
erster_faktor = Bruch(3,4)
zweiter_faktor = Bruch(2,3)
inhalt = inhalt + "\n\t\t<p><math>"
inhalt = inhalt + schreibeMultiplikation(erster_faktor, zweiter_faktor)
inhalt = inhalt + "</math></p>"
\end{codePython}

Führe die Datei \texttt{htmlErzeugung.py} aus.

Prüfe im Firefox-Browser, wie die Datei \texttt{index.html} aussieht. Dort sollte eine Zeile stehen, die so aussieht:
\[
\frac{3}{4} \cdot \frac{2}{3} = \frac{1}{2}
\]



\end{document}