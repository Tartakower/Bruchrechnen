\section{Lösung: Der Kehrwert eines Bruches}

\subsection*{Der Algorithmus zur Berechnung des Kehrwerts}

\begin{itemize}
	\item Nimm einen Bruch.
	\item Schreibe den Zähler des Bruchs in eine Variable.
	\item Schreibe den Nenner des Bruchs in eine andere Variable.
	\item Definiere eine Variable für den neuen Zähler und fülle sie mit der zweiten Variablen von eben.
	\item Verfahre mit dem neuen Nenner ebenso.
	\item Erzeuge einen neuen Bruch mit dem neuen Zähler und dem neuen Nenner. Merke dir den neuen Bruch in einer Variablen.
	\item Gib die Variable zurück, die den neuen Bruch enthält.
\end{itemize}

\subsection*{Die Funktion zur Berechung des Kehrwerts}

\begin{codePython}
def berechneKehrwert(bruch: Bruch) -> Bruch:
	alterZaehler = bruch.zaehler
	alterNenner = bruch.nenner
	neuerZaehler = alterNenner
	neuerNenner = alterZaehler
	kehrwert = Bruch(neuerZaehler, neuerNenner)
	return kehrwert
\end{codePython}

\subsection*{Überprüfung des Ergebnisses}

Erzeuge in der Datei \texttt{uebungsplatz.py} einen Bruch, z.B. $\frac{2}{5}$. Berechne den Kehrwertbruch und lasse diesen anzeigen.

\begin{codePython}
bruch = Bruch(2,5)
kehrwert = berechneKehrwert(bruch)
print(kehrwert)
\end{codePython}

Die Ausgabe lautet: \texttt{Bruch(zaehler=5, nenner=2)}