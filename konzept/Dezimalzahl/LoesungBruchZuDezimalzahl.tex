\section{Lösung: Die Umwandlung eines Bruchs in eine Dezimalzahl}

\subsection*{Der Algorithmus zur Umwandlung}

\begin{itemize}
	\item Nimm einen Bruch entgegen.
	\item Merke dir den Zähler des Bruchs in einer Variablen.
	\item Merke dir den Nenner des Bruchs in einer zweiten Variablen.
	\item Berechne den Quotienten aus Zähler und Nenner und merke dir das Ergebnis in einer neuen Variablen. 
	\item Erzeuge mit dem Quotienten eine Dezimalzahl und merke dir die Dezimalzahl in einer Variablen.
	\item Gib die Variable, welche die Dezimalzahl enthält, zurück.
\end{itemize}

\subsection*{Die Funktion zur Umwandlung}

\begin{codePython}
def wandleBruchZuDezimalzahl(bruch: Bruch) -> Dezimalzahl:
	zaehler = bruch.zaehler
	nenner = bruch.nenner
	kommazahl = zaehler / nenner
	dezimalzahl = Dezimalzahl(kommazahl)
	return dezimalzahl
\end{codePython}

\subsection*{Der Algorithmus zur Darstellung in MathML}

\begin{itemize}
	\item Nimm einen Bruch entgegen.
	\item Wandle den Bruch in eine Dezimalzahl und merke dir das Ergebnis in einer Variablen. Verwende dazu die Funktion \texttt{wandleBruchZuDezimalzahl}.
	\item Erstelle den MathML-String für den Bruch. Verwende dazu die Funktion \texttt{schreibeBruch}.
	\item Erstelle den MathML-String für die Dezimalzahl. Verwende dazu die Funktion \texttt{schreibeDezimalzahl}
	\item Baue den MathML-String aus den Einzelteilen zusammen.
	\item Gib die Variable für den Ergebnisstring zurück.
\end{itemize}

\subsection*{Die Funktion zur Darstellung in MathML}

\begin{codePython}
def schreibeBruchZuDezimalzahl(bruch: Bruch) -> str:
	dezimalzahl = wandleBruchZuDezimalzahl(bruch)
	textBruch = schreibeBruch(bruch)
	textDezimalzahl = schreibeDezimalzahl(dezimalzahl)
	ergebnis = textBruch + "<mo>=</mo>" + textDezimalzahl
	return ergebnis
\end{codePython}

\subsection*{Überprüfung des Ergebnisses}

Überprüfe das Ergebnis in der Datei \texttt{uebungsplatz.py}. Ergänze dazu folgende Zeilen:

\begin{codePython}
bruch = Bruch(3,4)
dezimalzahl = mathehelfer.wandleBruchZuDezimalzahl(bruch)
print(dezimalzahl)
\end{codePython}

Führe die Datei
\begin{quote}
\texttt{uebungsplatz.py}
\end{quote}
als Python-Skript aus. Das Ergebnis sollte diese Zeile enthalten:
\begin{quote}
\texttt{Dezimalzahl(kommazahl=0.75)}
\end{quote} 

\subsection*{Darstellung der Formel im Browser}

Ergänze in der Funktion \texttt{schreibeMathML()} folgende Zeilen:

\begin{codePython}    
bruch = Bruch(3,4)
inhalt = inhalt + "\n\t\t<p><math>"
inhalt = inhalt + schreibeBruchZuDezimalzahl(bruch)
inhalt = inhalt +  "</math></p>"
\end{codePython}

Führe die Datei
\begin{quote}
\texttt{htmlErzeugung.py}
\end{quote}
als Python-Skript aus. Schaue dir die Datei
\begin{quote}
\texttt{index.html}
\end{quote} im Firefox-Browser an. Dort sollte folgende Zeile zu lesen sein:
\[
\frac{3}{4} = 0.75
\]

