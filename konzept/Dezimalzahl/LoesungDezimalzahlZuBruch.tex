\section{Lösung: Die Umwandlung einer Dezimalzahl in einen Bruch}

\subsection*{Der Algorithmus zur Umwandlung}

\begin{itemize}
	\item Nimm einen Bruch entgegen.
	\item Merke dir den Zähler des Bruchs in einer Variablen.
	\item Merke dir den Nenner des Bruchs in einer zweiten Variablen.
	\item Berechne den Quotienten aus Zähler und Nenner und merke dir das Ergebnis in einer neuen Variablen. 
	\item Erzeuge mit dem Quotienten eine Dezimalzahl und merke dir die Dezimalzahl in einer Variablen.
	\item Gib die Variable, welche die Dezimalzahl enthält, zurück.
\end{itemize}

\subsection*{Die Funktion zur Umwandlung}

\begin{codePython}
\end{codePython}

\subsection*{Überprüfung des Ergebnisses}

Überprüfe das Ergebnis in der Datei \texttt{uebungsplatz.py}. Ergänze dazu folgende Zeilen:

\begin{codePython}
\end{codePython}

Führe die Datei
\begin{quote}
	\texttt{uebungsplatz.py}
\end{quote}
als Python-Skript aus. Das Ergebnis sollte diese Zeile enthalten:
\begin{quote}
\end{quote} 

\subsection*{Darstellung der Formel im Browser}

Ergänze in der Funktion \texttt{schreibeMathML()} folgende Zeilen:

\begin{codePython} 
\end{codePython}

Führe die Datei
\begin{quote}
	\texttt{htmlErzeugung.py}
\end{quote}
als Python-Skript aus. Schaue dir die Datei
\begin{quote}
	\texttt{index.html}
\end{quote} im Firefox-Browser an. Dort sollte folgende Zeile zu lesen sein:
\[
0.8 = \frac{4}{5}
\]

