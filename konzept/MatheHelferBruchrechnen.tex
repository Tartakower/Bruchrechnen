\documentclass[12p,numbers=noendperiod,DIV=15]{scrreprt}

\usepackage[ngerman]{babel}
\usepackage[utf8]{inputenc}
\usepackage[T1]{fontenc}

% Schriftart DejaVu
\usepackage{dejavu}
\usepackage[onehalfspacing]{setspace}
%\setstretch{1.5}

\setlength{\parindent}{0pt}
\setlength{\parskip}{2ex plus0.5ex minus0.5ex}

\usepackage{enumitem}
\setlist{noitemsep,topsep=0pt}

% Kopf- und Fußzeile
%\pagestyle{headings}
%\usepackage[singlespacing=true]{scrlayer-scrpage}
%\ifoot{\copyright~Norbert Seulberger, 2021}
%\cfoot{}
%\ofoot{\pagemark}
%\pagestyle{scrheadings}
\pagestyle{plain}
%\setkomafont{pageheadfoot}{\small}

% Koma-Script für Inhaltsverzeichnis
\usepackage{tocbasic}
\DeclareTOCStyleEntries[dynnumwidth=true]{tocline}{chapter,section,subsection}

\usepackage{color}
\usepackage{listings}
\usepackage[fleqn]{amsmath}
\usepackage{scrhack}
%\usepackage[top=20mm,left=20mm,right=20mm,bottom=30mm]{geometry}
\mathindent=2em

%\usepackage{tikz}
%\usetikzlibrary{shapes.geometric, arrows}

\sloppy

%% Umgebungen

\newcommand{\seitenumbruch}{\newpage\textcolor{white}{Seitenumbruch}}

\newcounter{regelsatz}[section]

\newenvironment{ncsEnvironmentEins}[1]{\stepcounter{regelsatz} \textbf{\arabic{section}.\arabic{regelsatz} #1} \\}{}
%\newcommand{\aufgabe}[1]{\begin{ncsEnvironmentEins}{Aufgabe: #1} \end{ncsEnvironmentEins}}
\newcommand{\beispiel}[1]{\begin{ncsEnvironmentEins}{Beispiel: #1} \end{ncsEnvironmentEins}}
\newcommand{\hinweis}[1]{\begin{ncsEnvironmentEins}{Hinweis: #1} \end{ncsEnvironmentEins}}
%\newcommand{\loesung}[1]{\begin{ncsEnvironmentEins}{Lösung: #1} \end{ncsEnvironmentEins}}
%\newcommand{\motivation}[1]{\begin{ncsEnvironmentEins}{Motivation: #1} \end{ncsEnvironmentEins}}
\newcommand{\nurTitel}[1]{\begin{ncsEnvironmentEins}{#1} \end{ncsEnvironmentEins}}
%
\newenvironment{ncsEnvironmentZwei}[2]%
{\stepcounter{regelsatz} \textbf{\arabic{chapter}.\arabic{regelsatz} #1} \\ #2}{}
\newcommand{\titelText}[2]{\begin{ncsEnvironmentZwei}{#1}{#2}\end{ncsEnvironmentZwei}}

%\newenvironment{ncsEnvironment}[2]%
%{\stepcounter{regelsatz} \textbf{\arabic{chapter}.\arabic{regelsatz} #1: #2} \\ }{}
%\newcommand{\ncsEnv}[3]{\begin{ncsEnvironment}{#1}{#2} #3 \end{ncsEnvironment}}
%\newcommand{\empfehlung}[2]{\begin{ncsEnvironment}{Empfehlung}{#1} #2 \end{ncsEnvironment}}
%\newcommand{\frage}[2]{\begin{ncsEnvironment}{Frage}{#1} #2 \end{ncsEnvironment}}
%\newcommand{\refactoring}[2]{\begin{ncsEnvironment}{Refactoring}{#1} #2 \end{ncsEnvironment}}
%\newcommand{\synonym}[2]{\begin{ncsEnvironment}{Synonym}{#1} #2 \end{ncsEnvironment}}
%\newcommand{\syntaxPython}[2]{\begin{minipage}{\textwidth}%
%		\begin{ncsEnvironment}{Syntax in Python}{#1} #2 \end{ncsEnvironment}\end{minipage}}
%\newcommand{\test}[2]{\begin{ncsEnvironment}{Unittest}{#1} #2 \end{ncsEnvironment}}
%\newcommand{\tipp}[2]{\begin{ncsEnvironment}{Tipp}{#1} #2 \end{ncsEnvironment}}
%\newcommand{\warnung}[2]{\begin{ncsEnvironment}{Warnung}{#1} #2 \end{ncsEnvironment}}

\newenvironment{ncsItEnvironment}[2]%
{\stepcounter{regelsatz} \textbf{\arabic{chapter}.\arabic{regelsatz} #1: #2} \\ \begin{itshape}}{\end{itshape}}
\newcommand{\begriff}[2]{\begin{ncsItEnvironment}{Begriff}{#1} #2 \end{ncsItEnvironment}}
%\newcommand{\prinzip}[2]{\begin{ncsItEnvironment}{Prinzip}{#1} #2 \end{ncsItEnvironment}}
%\newcommand{\ncsItEnv}[3]{\begin{ncsItEnvironment}{#1}{#2} #3 \end{ncsItEnvironment}}
% Listings Styles

\definecolor{darkred}{rgb}{0.8,0,0}
\definecolor{darkgreen}{rgb}{0.0,0.5,0}

\lstdefinestyle{syntaxPython}{
	language=python, 				        		
	basicstyle=\linespread{1.2}\ttfamily,
	keywordstyle=\color{darkred}\bfseries,	
	identifierstyle=\color{blue},				
	commentstyle=\color{darkgreen},			
	stringstyle=\ttfamily\color{darkgreen}, 			    		
	showstringspaces=false,
	breaklines=true, 			        		
	numbers=none, 				        		
	numberstyle=\small,		        			
	frame=single, 				        		
	tabsize=4,				            		
	float=htbp,
	captionpos=b,
	aboveskip=2ex,
	belowskip=\medskipamount,
	xleftmargin=.02\textwidth}

\lstdefinestyle{syntaxHTML}{%
	language=HTML,%
	basicstyle=\small\ttfamily,%
	keywordstyle=\color{darkred}\bfseries,%
	identifierstyle=\color{blue},%	
	stringstyle=\color{darkgreen},%
	tabsize=4,%
	showstringspaces=false,%
	numbers=none,%			        		
	numberstyle=\small,%	        			
	frame=single,%
	captionpos=b,%
	xleftmargin=.06\textwidth
} 

\lstset{literate=%
	{Ö}{{\"O}}1
	{Ä}{{\"A}}1
	{Ü}{{\"U}}1
	{ß}{{\ss}}2
	{ü}{{\"u}}1
	{ä}{{\"a}}1
	{ö}{{\"o}}1
}

\lstnewenvironment{codePython}{\lstset{style=syntaxPython}}{}
\lstnewenvironment{codeHTML}{\lstset{style=syntaxHTML}}{}

\usepackage[colorlinks=true,linkcolor=blue]{hyperref}

%\includeonly{%
%	./Kapitel/MathML,%
%	./Kapitel/AufgabeMathML%
%}

\begin{document}
	
\title{Mathehelfer Bruchrechnen}
\subtitle{Ein Hackathon für Kinder und Jugendliche}
\author{Norbert Seulberger}
\date{September 2022}

\maketitle

\tableofcontents

\chapter{Thema und Planung}

\section{Die Aufgabenstellung}

Der Mathehelfer Bruchrechnen kann in der Grundversion Brüche und Kommazahlen ineinander umrechnen. Außerdem können Brüche gekürzt werden.

Die Darstellung der Umrechnungen erfolgt in einem Browser. Dabei werden die Brüche grafisch schön mit Zähler, Nenner und Bruchstrich dargestellt wie in einem Mathematikbuch.

Es sind also beispielsweise folgende Operationen möglich:
\begin{align*}
& \frac{6}{8} = \frac{3}{4} & \\[2ex]
& \frac{3}{4} = 0.75 & \\[2ex]
& 0.8 = \frac{4}{5} & 
\end{align*}


Dabei geben wir die linke Seite fest vor und berechnen die rechte Seite jeweils mit einem Programm. Wir schreiben jeweils eine eigene Funktion, also drei Funktionen, die

\begin{itemize}
	\item einen Bruch kürzt,
	\item einen Bruch in eine Kommazahl umrechnet oder 
	\item eine Kommazahl in einen Bruch umwandelt.
\end{itemize}

Als Programmiersprache verwenden wir Python.

Die Formelzeile tragen wir automatisch in eine kleine Web-Seite ein, die wir uns im Browser ansehen. Dabei beschreiben wir die Formel in einer speziellen HTML-Sprache für mathematische Formeln: MathML.

Die Sprache MathML ist sehr einfach. Allerdings muss man sehr viel Text eingeben, um eine Formel mit MathML zu beschreiben. Daher schreiben wir ein Programm, das die Formel automatisch in MathML ausdrückt.
\section{Technische Rahmenbedingungen}

\subsection*{Hardware und Betriebssystem}

Die Beispiele wurden auf einem Raspberry Pi, Version 4, mit 8 GB Ram entwickelt. Damit lässt sich stabil und flüssig arbeiten.

Vermutlich reicht ein Raspi 4 mit mindestens 2 GB Ram aus.

Ein Raspi 3 mit 1 GB Ram läuft nicht mehr performant mit einer anspruchsvollen IDE wie z.B. VS Code.

Als Betriebssystem wurden verschiedene Debian-/Ubuntu-Derivate für arm64-Prozessoren genutzt. Diese arbeiten stabil und flüssig. Unter Windows oder MacOS sollte alles genauso gut klappen.

\subsection*{Programmiersprache}

Der Mathehelfer Bruchrechnen wird in Python programmiert. Da die Klasse für Brüche als Dataclass definiert werden, ist mindestens Version 3.7 erforderlich. Jede aktuelle Linux-Distribution enthält diese oder eine neuere Python-Version.

\subsection*{IDE}

Für Python existieren viele verschiedene Entwicklungsumgebungen. Um eine sinnvolle Unterstützung durch Code-Vervollständigung zu erhalten, sind folgende IDEs geeignet:
\begin{itemize}
	\item Visual Studio Code (Python-Plugin, Pylance-Plugin von Microsoft)
	\item Pycharm
	\item eclipse mit dem Pydev-Plugin
\end{itemize}

Die Online-IDE repl.it funktioniert ebenfalls gut. Allerdings ist der Vorteil der fehlenden Installation gegenüber der Online-Abhängigkeit abzuwägen.

\subsection*{Browser}

MathML wird aktuell unmittelbar nur von Firefox dargestellt.

Die Chromium-/Chrome-Familie stellt MathML nicht (mehr)  dar.
\section{Zeitliche Planung}

Der zeitliche Rahmen besteht aus einer zweitägigen Veranstaltung mit jeweils vier Stunden Dauer.

Die TeilnehmerInnen arbeiten in Zweier-Teams an einem Rechner. Sie dürfen entscheiden, wie selbstständig sie arbeiten möchten. Ein möglichst eigenständiges Vorgehen ist wünschenswert.

Es sind ausreichend Pausen vorzusehen. Die TeilnehmerInnen dürfen die Pausen selbst bestimmen, werden von den Betreuern daran erinnert.

\subsection*{Tag 1}

\begin{itemize}
	\item Begrüßung durch einen Inspirer der Hacker School (15 min, 15/240)
	\item Begrüßung durch das Veranstalterteam / die Inspirer des Veranstalters. Vorstellung der TeilnehmerInnen einschließlich Vorkenntnisse und Erwartungen (30 min, 45/240)
	\item MathML: Durchlesen der Unterlagen zu MathML, Lösen der Aufgabe (45 min, 90/240)
	\item Pause (15 min, 105/240)
	\item Algorithmen und Funktionen: Durchlesen der Unterlagen (20 min, 125/240)
	\item Nachvollziehen und Umsetzen des Beispiels Kürzen (20 min, 145/240)
	\item Pause (15 min, 160/240)
	\item Bearbeiten der Aufgaben zum Erweitern und der Multiplikation (40 min, 200/240)
	\item Vorstellen von Teamlösungen, Code-Reviews (30 min, 230/240)
	\item Feedback-Runde (10 min, 240/240)
\end{itemize}

\subsection*{Tag 2}

\begin{itemize}
	\item Ankommen, Feedback zu Vortag (10 min, 10/240)
	\item Optional: Fertigstellen der Aufgaben vom Vortag (30 min, 40/240)
	\item Zeichenketten: Durchlesen der Unterlagen, Nachvollziehen und Umsetzen des Beispiels (30 min, 40/240)
	
\end{itemize}
\section{Vorkenntnisse}

Das Thema setzt nur wenige Vorkenntnisse zwingend voraus. Allerdings wächst der Spaß mit steigendem Schwierigkeitsgrad der Aufgaben.

\subsection*{Notwendige Vorkenntnisse}

Unverzichtbar ist eine gewisse Fertigkeit im Umgang mit einem PC oder Laptop, u.a.
\begin{itemize}
	\item flüssige Bedienung von Maus und Tastatur,
	\item Aufruf und Nutzung des Dateimanagers,
	\item das Betrachten von HTML-Dateien im Browser,
	\item das Editieren von Text in einem Editor.
\end{itemize}

Unbedingt bekannt sollte das Bruchrechnen sein. Für das Formulieren der Algorithmen ist die Kenntnis der Rechenregeln und von einschlägigen Merksätzen wichtig.

\subsection*{Hilfreiche Vorkenntnisse}

Grundsätzlich wird in Form der Funktion ein einfaches Sprachelement von Python genutzt, das allerdings in Lehrtexte häufig sehr spät eingeführt wird. Daher erklärt der Kurs, wie Funktionen in Python aufgebaut sind. Vorerfahrung in der Programmierung mit Python ist natürlich hilfreich, weil dann die anspruchsvolleren Aufgaben bewältigt werden können.

Bei der Generierung von MathML sind Vorkenntnisse in HTML hilfreich, etwa der grundsätzliche Aufbau mit öffnenden und schließenden Tags.

\subsection*{Abfrage der Vorkenntnisse und Erwartungen}

Bei der Vorstellung der TeilnehmerInnen werden die Vorkenntnisse und Erwartungen besprochen. Mögliche Fragen sind:

\begin{itemize}
	\item Nutzt du regelmäßig den Computer zum Schreiben von Texten, beispielsweise für die Schule?
	\item Hast du schon einmal eine Web-Seite erstellt? Kennst du dich mit HTML aus?
	\item Hast du schon einmal mit Python programmiert? Was hast du dort kennengelernt?
	\item Was wünscht du dir von diesem Workshop?
\end{itemize}

\chapter{Zum Warmwerden: MathML}

\section{MathML}

MathML ist eine einfache Sprache, die in einem HTML-Dokument dazu verwendet werden kann, um mathematische Formeln darzustellen.

Genauso wie in HTML werden die Daten in sogenannte Tags eingeschlossen, d.h. ein Element beginnt mit einen Tag, dann stehen die Daten und ein schließendes Tag beendet den Ausdruck. 

\subsection*{MathML: Anfang und Ende einer Formel}

Eine Formeln beginnt mit \texttt{<math>} und endet mit \texttt{</math>}.

Für die verschiedenen Teile einer mathematischen Formel gibt es unterschiedliche Elemente.
\begin{itemize}
	\item \texttt{<mn>}42\texttt{</mn>} stellt die Zahl 42 dar.
	\item \texttt{<mo>}=\texttt{</mo>} beschreibt ein Gleichheitszeichen.
	\item \texttt{<mo>}+\texttt{</mo>} schreibt ein Pluszeichen.
\end{itemize}

Die Formel $2 + 3 = 5$ lautet also in MathML:

\begin{codeHTML}
<math>
	<mn>2</mn>
	<mo>+</mo>
	<mn>3</mn>
	<mo>=</mo>
	<mn>5</mn>
</math>
\end{codeHTML}

\textbf{Tipp}

Der Browser ignoriert die Zeilenumbrüche in MathML. Wir können die Formel also auch in eine Zeile schreiben.

\begin{codeHTML}
<math><mn>2</mn><mo>+</mo><mn>3</mn><mo>=</mo><mn>5</mn></math>
\end{codeHTML}

\subsection*{Brüche in MathML}

Ein Bruch beginnt mit \texttt{<mfrac>} und endet mit \texttt{</mfrac>}. Den Zähler und den Nenner schreiben wir zwischen \texttt{<mn>} und \texttt{</mn>}.

Der Bruch $\frac{2}{3}$ sieht also so aus:

\begin{codeHTML}
<math>
	<mfrac>
		<mn>2</mn>
		<mn>3</mn>
	</mfrac>
</math>
\end{codeHTML}

\subsection*{Eine Formelzeile}

Damit jede Formel in eine eigene Zeile geschrieben wird, verwenden wir das Absatz-Tag von HTML: \texttt{<p>} ... \texttt{</p>}

Also zusammen:
\begin{codeHTML}
<p>
	<math><mn>2</mn><mo>+</mo><mn>3</mn><mo>=</mo><mn>5</mn></math>
</p>
\end{codeHTML}

\subsection*{Eine Anleitung zu MathML}

Eine einfache, aber umfassende Anleitung zu MathML findest du auf folgender Internetseite: \href{https://www.math-it.de/Publikationen/MathML_de.html}{MathML-Tutorial}
\section{Aufgabenblatt MathML}

\subsection{Aufgabe: Anzeige der HTML-Datei}

Schau dir im Firefox-Browser die Datei \texttt{manuell.html} an. Die Anzeige sollte etwa so aussehen:

\begin{equation*}
\frac{6}{8} = \frac{3}{4}
\end{equation*}

Hier kannst du die Formeln für das Umrechnen eines Bruchs in eine Dezimalzahl und umgekehrt einfügen.

\subsection{Aufgabe: Ansicht der HTML-Datei im Editor}

Schau dir jetzt dieselbe Datei im Editor der Entwicklungsumgebung an. Der Text sieht etwa so aus:

\begin{codeHTML}{Die Datei manuell.html}
<!doctype html>
<html lang="de">
	<head>
		<meta charset="utf-8">
		<meta name="viewport" content="width=device-width, initial-scale=1.0">
		<title>Bruchrechnen</title>
	</head>
	<body>
		<p>
			<math>
				<mfrac>
					<mi>6</mi>
					<mi>8</mi>
				</mfrac>
				<mo>=</mo>
				<mfrac>
					<mi>3</mi>
					<mi>4</mi>
				</mfrac>
			</math>
		</p>
		<p>
 			Hier kannst du die Formeln für das Umrechnen ...
		</p>
	</body>
</html>
\end{codeHTML}

Kannst du die Element entdecken, die für die Darstellung des Kürzens der beiden Brüche stehen?

\subsection{Aufgabe: Die Beschreibung der beiden Umrechungen}

Füge in die Datei \texttt{manuell.html} den Text in der MathML-Sprache ein, so dass die beiden Umrechungen dargestellt werden (zusätzlich zur Formel für das Kürzen). Der Text „Hier kannst du \dots“ darf überschrieben werden.

\begin{align*}
& \frac{6}{8} = \frac{3}{4} & \\[2ex]
& \frac{3}{4} = 0.75 & \\[2ex]
& 0.8 = \frac{4}{5} & 
\end{align*}

Überprüfe das Ergebnis im Firefox-Browser.

\chapter{Algorithmen und Funktionen}

\section{Algorithmen}
\label{sec:Algorithmus}

Ein Algorithmus ist die eindeutige Beschreibung eines Verfahrens, um eine bestimmte Aufgabe zu lösen.

\subsection*{Beispiel: Das Erweitern eines Bruches}

Ein Bruch wird erweitert, indem der Zähler und der Nenner mit demselben Faktor multipliziert werden. Wird beispielsweise der Bruchs $\frac{2}{5}$ mit dem Faktor $3$ erweitert, so ergibt sich diese Gleichung:
\[
\frac{2}{5} = \frac{2 \cdot 3}{5 \cdot 3} = \frac{6}{15}
\]

\subsection*{Der Algorithmus für das Erweitern}

\begin{itemize}
	\item Nimm einen Bruch und den Faktor, mit dem du den Bruch erweitern willst.
	\item Merke dir den Zähler und den Nenner des Bruchs jeweils in einer Variablen.
	\item Multipliziere die Variable für den Zähler mit dem Erweiterungsfaktor.
	\item Verfahre genauso mit dem Nenner.
	\item Erstelle einen neuen Bruch mit dem neuen Zähler und dem neuen Nenner.
\end{itemize}
\section{Variablen}

\subsection*{Einfache Variable}

Eine Variable dient dazu, sich einen Wert, einen Text oder das Ergebnis einer Rechnung zu merken. Wir geben einer Variablen einen sinnvollen Namen.

\begin{codePython}
wichtige_zahl = 42
summe = 2 + 3
\end{codePython}

Mit Variablen können wir auch rechnen.

\begin{codePython}
summe = summe + 6
\end{codePython}

\subsection*{Strukturierte Variablen}

Wir können Variablen auch komplizierter aufbauen und einen Wert in einer Variablen merken, der sich aus mehreren Teilwerten zusammensetzt. Ein Beispiel ist der Bruch:

\begin{codePython}
class Bruch():
	zaehler: int
	nenner: int
\end{codePython}

Ein Bruch ist ein Wert, der sich aus zwei Teilwerten zusammensetzt, nämlich dem Zähler und dem Nenner. Den Bruchstrich müssen wir uns nicht merken, weil er immer da ist. Zähler und Nenner sind ganze Zahlen (\texttt{int}).
%Wir merken uns den Zähler und Nenner zusammen in einem Bruch. Diesen Bruch können wir dann als einen Wert in einer Variablen aufbewahren.

\lstset{%
	language=python, 				        		
	basicstyle=\linespread{1.2}\ttfamily,
	keywordstyle=\color{darkred}\bfseries,	
	identifierstyle=\color{blue},				
	commentstyle=\color{darkgreen},			
	stringstyle=\ttfamily\color{darkgreen}, 			    		
	showstringspaces=false,
	breaklines=true, 			        		
	numbers=none, 				        		
	numberstyle=\small,		        			
	frame=single, 				        		
	tabsize=4,				            		
	float=htbp,
	captionpos=b,
	aboveskip=2ex,
	belowskip=\medskipamount,
	xleftmargin=.02\textwidth			
}
\begin{lstlisting}
bruch_1 = Bruch(3,4)
\end{lstlisting}

Jetzt können wir die Variable, die ja einen Bruch enthält, nach dem Zähler und dem Nenner fragen. Wir schreiben die beiden Werte in eigene Variablen.

\begin{lstlisting}
zaehler_1 = bruch_1.zaehler
nenner_1 = bruch_1.nenner
\end{lstlisting}

Wir können auch zwei Ganzzahlen jeweils in eine Variable schreiben und damit eine Bruchvariable erzeugen.

\begin{lstlisting}
zaehler_2 = 7
nenner_2 = 8
bruch_2 = Bruch(zaehler_2, nenner_2)
\end{lstlisting}
\section{Funktionen}

Eine Funktion ist ein kleines Teilprogramm, das eine spezifische Aufgabe löst. Für bestimmte Eingabewerte erhält man immer dasselbe Ergebnis.

\beispiel{Eine Funktion zur Mittelwertberechnung zweier Zahlen}
Der Algorithmus lautet:

\include{./Kapitel/AufgabeKehrwert}

\chapter{Zeichenketten}
\section{Zeichenketten}

\subsection*{Was ist eine Zeichenkette?}

Eine Zeichenkette enthält ein oder mehrere beliebige Zeichen, also Buchstaben, Zahlen oder Sonderzeichen. Die Zeichen stehen hintereinander. Eine Zeichenkette wird eingeschlossen in Hochkommata (') oder hochgestellte Anführungszeichen (").

\begin{codePython}
eine_zeichenkette = "r2d2"
berg = '8000er-Gipfel'
ein_ganzer_satz = "Python ist toll!"
\end{codePython}

Häufig benutzen wir den englischen Begriff für Zeichenkette: String. Mit Strings kann man viele interessante Dinge anstellen.

\subsection*{Das Zusammenfügen von Zeichenketten}

In Python können zwei Strings mit dem Pluszeichen zu einem langen String zusammengefügt werden:

\begin{codePython}
gruss = "Hallo " + "Welt!"
\end{codePython}

In \texttt{gruss} steht jetzt der String \texttt{"Hallo Welt!"}.

\subsection*{Tipp: Das Pluszeichen setzt Zeichenketten zusammen!}

Das Pluszeichen bedeutet in diesem Fall nicht das Addieren von Zahlen, sondern das Zusammenfügen von Zeichenketten. Daran gewöhnt man sich schnell!

\subsection*{MathML als Zeichenkette}

Wir möchten einen String zusammenbauen, der in MathML eine Rechenformel beschreibt.

Die Formel
\[
2 + 3 = 5
\]
lautet in MathML:

\begin{codeHTML}
<math><mn>2</mn><mo>+</mo><mn>3</mn><mo>=</mo><mn>5</mn></math>
\end{codeHTML}

Diese Formel können wir in Python schrittweise aus kleinen Strings zu einem langen String zusammensetzen. 

\lstset{style=syntaxPython}
\begin{lstlisting}
formel = "<math>"
formel = formel + "<mn>2</mn>"
formel = formel + "<mo>+</mo>"
formel = formel + "<mn>3</mn>"
formel = formel + "<mo>=</mo>"
formel = formel + "<mn>5</mn>"
formel = formel + "</math>"
\end{lstlisting}

Dabei nehmen wir die bereits erstellte Formel, fügen den nächsten kurzen String an und merken uns das Ergebnis wieder in der Variablen \texttt{formel}.

\chapter{Das Beispiel}

\section{Beispiel: Das Erweitern eines Bruchs}

Das folgende Beispiel fasst nochmals alle zusammen, was in den Kapitel über MathML, Algorithmen, Funktionen und Zeichenketten vorgestellt wurde. Lies dieses Beispiel sorgfältig durch. Es dient als Vorlage für die Aufgaben, die du danach selbstständig lösen sollst.

\subsection{Der Algorithmus zum Erweitern}

\begin{itemize}
	\item Nimm einen Bruch und den Faktor, mit dem du den Bruch erweitern willst.
	\item Merke dir den Zähler und den Nenner des Bruchs jeweils in einer Variablen.
	\item Multipliziere diese Variablen jeweils mit dem Erweiterungsfaktor.
	\item Erstelle einen neuen Bruch aus dem neuen Zähler und dem neuen Nenner.
\end{itemize}

\subsection{Die Funktion zum Erweitern}
\label{sec:FunktionErweitern}

\begin{codePython}
def erweitereBruch(bruch: Bruch, faktor: int) -> Bruch:
	alterZaehler = bruch.zaehler
	alterNenner = bruch.nenner
	neuerZaehler = alterZaehler * faktor
	neuerNenner = alterNenner * faktor
	neuerBruch = Bruch(neuerZaehler, neuerNenner)
	return neuerBruch
\end{codePython}

\subsection{Die Funktion zur Darstellung eines Bruchs in MathML}
\label{sec:FunktionSchreibeBruch}

Der Algorithmus lautet:
\begin{itemize}
	\item Nimm einen Bruch entgegen.
	\item Wandle den Zähler des Bruchs in eine Zeichenkette und speichere diese Zeichenkette in einer Variablen.
	\item Verfahre genauso mit dem Nenner.
	\item Baue die Zeichenkette für das Ergebnis aus Einzelteilen zusammen, wie in den nächsten Schritten beschrieben.
	\item Beginne die Zeichenkette mit \texttt{<mfrac>}.
	\item Rahme die Zeichenkette, die den Zähler beschreibt, mit den Tags \texttt{<mn>} und \texttt{</mn>} ein.
	\item Verfahre ebenso mit der Zeichenkette für den Nenner.
	\item Beende die Zeichenkette mit \texttt{</mfrac>}.
	\item Gib die Zeichenkette zurück.
\end{itemize}

\begin{codePython}
def schreibeBruch(bruch: Bruch) -> str:
	zaehlerAlsString = str(bruch.zaehler)
	nennerAlsString = str(bruch.nenner)
	ergebnis = "<mfrac>"
	ergebnis = ergebnis + "<mn>" + zaehlerAlsString + "</mn>"
	ergebnis = ergebnis + "<mn>" + nennerAlsString + "</mn>"
	ergebnis = ergebnis + "</mfrac>"
	return ergebnis
\end{codePython}

\subsection{Die Funktion zur Darstellung der Formel in MathML}

Der Algorithmus lautet:
\begin{itemize}
	\item Nimm einen Bruch.
	\item Erzeuge aus diesem Bruch einen erweiterten Bruch. Verwende dazu die Funktion \ref{sec:FunktionErweitern}, die du eben gesehen hast.
	\item Erzeuge aus dem ursprünglichen Bruch die Beschreibung in MathML. Verwende dazu die Funktion \ref{sec:FunktionSchreibeBruch} von oben.
	\item Erzeuge aus dem erweiterten Bruch die Beschreibung in MathML. Verwende dazu ebenfalls die Funktion \ref{sec:FunktionSchreibeBruch}.
	\item Setze jetzt den MathML-Text aus den einzelnen Angaben zusammen wie in den folgenden Schritten angegeben:
	\item Es beginnt mit dem MathML-Text für den ungekürzten Bruch.
	\item Dann kommt das Gleichheitszeichen.
	\item Es folgt der MathML-Text für den gekürzten Bruch.
\end{itemize}

Die Funktion lautet dann:
\begin{codePython}
def schreibeErweitern(bruch: Bruch, faktor: int) -> str:
	bruchErweitert = erweitereBruch(bruch, faktor)
	textAlterBruch = schreibeBruch(bruch)
	textNeuerBruch = schreibeBruch(bruchErweitert)
	ergebnis = textAlterBruch + "<mo>=</mo>" + textNeuerBruch
	return ergebnis
\end{codePython}

\subsection{Überprüfung des Ergebnisses}

Wir überprüfen das Ergebnis, indem wir in der Funktion \texttt{schreibeMathML()} folgende Zeilen ergänzen:

\begin{codePython}
inhalt = inhalt + "\n\t\t<p><math>"
inhalt = inhalt + schreibeErweitern(Bruch(3,4), 3)
inhalt = inhalt + "</math></p>"
\end{codePython}

Jetzt führen wir die Datei \texttt{htmlErzeugung.py} aus.

Prüfe im Firefox-Browser, wie die Datei \texttt{index.html} aussieht. Dort sollte eine Zeile stehen, die so aussieht:
\[
\frac{3}{4} = \frac{9}{12}
\]

\chapter{Die Aufgaben}

\include{./Kapitel/AufgabeKuerzen}
\section{Aufgabenblatt: Umwandlung eines Bruchs in eine Dezimalzahl}
\section{Aufgabe: Wandle eine Dezimalzahl in einen Bruch}

\chapter{Aufgaben für Könner}

\section{Aufgabe: Programmiere die Grundrechenarten für Brüche}

Diese Aufgabe ist etwas schwieriger.
\section{Aufgabe: Gemischte Zahlen}

Diese Aufgabe ist schwierig.
\include{./Kapitel/ObjektorientierteProgrammierung}

\chapter{Lösungen}

\section{Lösung: Der Kehrwert eines Bruches}

\subsection*{Der Algorithmus zur Berechnung des Kehrwerts}

\begin{itemize}
	\item Nimm einen Bruch.
	\item Schreibe den Zähler des Bruchs in eine Variable.
	\item Schreibe den Nenner des Bruchs in eine andere Variable.
	\item Definiere eine Variable für den neuen Zähler und fülle sie mit der zweiten Variablen von eben.
	\item Verfahre mit dem neuen Nenner ebenso.
	\item Erzeuge einen neuen Bruch mit dem neuen Zähler und dem neuen Nenner. Merke dir den neuen Bruch in einer Variablen.
	\item Gib die Variable zurück, die den neuen Bruch enthält.
\end{itemize}

\subsection*{Die Funktion zur Berechung des Kehrwerts}

\begin{codePython}
def berechneKehrwert(bruch: Bruch) -> Bruch:
	alterZaehler = bruch.zaehler
	alterNenner = bruch.nenner
	neuerZaehler = alterNenner
	neuerNenner = alterZaehler
	kehrwert = Bruch(neuerZaehler, neuerNenner)
	return kehrwert
\end{codePython}

\subsection*{Überprüfung des Ergebnisses}

Erzeuge in der Datei \texttt{uebungsplatz.py} einen Bruch, z.B. $\frac{2}{5}$. Berechne den Kehrwertbruch und lasse diesen anzeigen.

\begin{codePython}
bruch = Bruch(2,5)
kehrwert = berechneKehrwert(bruch)
print(kehrwert)
\end{codePython}

Die Ausgabe lautet: \texttt{Bruch(zaehler=5, nenner=2)}
\section{Lösung: Das Kürzen eines Bruchs}

\subsection*{Der Algorithmus zum Kürzen eines Bruchs}

\begin{itemize}
	\item Nimm einen Bruch.
	\item Merke dir den Zähler des Bruchs in einer Variablen.
	\item Verfahre ebenso mit dem Nenner.
	\item Berechne den größten gemeinsamen Teiler (ggT) von Zähler und Nenner. (Es gibt dafür eine fertige Funktion.)
	\item Berechne den gekürzten Zähler, indem du den alten Zähler durch den ggT teilst. 
	\item Berechne ebenso den gekürzten Nenner.
	\item Erstelle einen neuen Bruch aus dem gekürzten Zähler und dem gekürzten Nenner.
	\item Gib den neuen Bruch zurück.
\end{itemize}

\subsection*{Die Funktion zum Kürzen von Brüchen}

Die Berechnung des ggT erfolgt mittels der Funktion \texttt{mathefunktionen.berechne\_ggT}. In der Funktion zum Kürzen wird also eine Zeile enthalten sein:

\begin{codePython}
ggT = mathefunktionen.berechne_ggT(alterZaehler, alterNenner)
\end{codePython}

Beachte, dass beim Teilen durch den ggT die Division \texttt{//} benutzt wird, damit der neue Zähler und der neue Nenner ganze Zahlen werden!

\begin{codePython}
def kuerzeBruch(bruch: Bruch) -> Bruch:
	alterZaehler = bruch.zaehler
	alterNenner = bruch.nenner
	ggT = mathefunktionen.berechne_ggT(alterZaehler, alterNenner)
	neuerZaehler = alterZaehler // ggT
	neuerNenner = alterNenner // ggT
	neuerBruch = Bruch(neuerZaehler, neuerNenner)
	return neuerBruch
\end{codePython}

\subsection*{Die Algorithums zur Darstellung der Formel zum Kürzen}

\begin{itemize}
	\item Nimm einen Bruch.
	\item Erzeuge aus diesem Bruch einen gekürzten Bruch. Verwende dazu die Funktion, die du gerade programmiert hast.
	\item Erzeuge aus dem ursprünglichen Bruch die Beschreibung in MathML. Verwende dazu die Funktion, die einen Bruch in MathML beschreibt.
	\item Erzeuge aus dem gekürzten Bruch ebenfalls die Beschreibung in MathML. Das geht genauso wie eben.
	\item Setze jetzt den MathML-Text aus den einzelnen Angaben zusammen.
	\item Die Zeichenkette beginnt mit dem MathML-Text für den ungekürzten Bruch.
	\item Dann kommt das Gleichheitszeichen.
	\item Es folge der MathML-Text für den gekürzten Bruch.
\end{itemize}

\subsection*{Die Funktion zur Darstellung der Formel zum Kürzen}

\begin{codePython}
def schreibeKuerzen(bruch: Bruch) -> str:
	bruchGekuerzt = kuerzeBruch(bruch)
	textUngekuerzt = schreibeBruch(bruch)
	textGekuerzt = schreibeBruch(bruchGekuerzt)
	ergebnis = textUngekuerzt + "<mo>=</mo>" + textGekuerzt
	return ergebnis
\end{codePython}

\subsection*{Überprüfung des Ergebnisses}

Überprüfe das Ergebnis in der Datei \texttt{uebungsplatz.py}. 

\begin{codePython}
bruch = Bruch(6,8)
formel = schreibeKuerzen(bruch)
print(formel)
\end{codePython}

Der Bruch $\frac{6}{8}$ muss gekürzt den Bruch $\frac{3}{4}$ ergeben. In einer Zeile steht:

\begin{codeHTML}
<mfrac><mn>6</mn><mn>8</mn></mfrac><mo>=</mo>
	<mfrac><mn>3</mn><mn>4</mn></mfrac>
\end{codeHTML}

Ergänze in der Funktion \texttt{schreibeMathML()} die Zeilen für Kürzen für den Bruch $\frac{6}{8}$.

\begin{codePython}
bruch = Bruch(6,8)
inhalt = inhalt + "\n\t\t<p><math>"
inhalt = inhalt + schreibeKuerzen(bruch)
inhalt = inhalt + "</math></p>"
\end{codePython}

Führe die Datei \texttt{htmlErzeugung.py} aus.

Prüfe im Firefox-Browser, wie die Datei \texttt{index.html} aussieht. Dort sollte eine Zeile stehen, die so aussieht:
\[
\frac{6}{8} = \frac{3}{4}
\]



\end{document}