\documentclass[12p,numbers=noendperiod,DIV=15]{scrartcl}

\usepackage[ngerman]{babel}
\usepackage[utf8]{inputenc}
\usepackage[T1]{fontenc}

% Schriftart DejaVu
\usepackage{dejavu}
\usepackage[onehalfspacing]{setspace}
%\setstretch{1.5}

\setlength{\parindent}{0pt}
\setlength{\parskip}{2ex plus0.5ex minus0.5ex}

\usepackage{enumitem}
\setlist{noitemsep,topsep=0pt}

% Kopf- und Fußzeile
%\pagestyle{headings}
%\usepackage[singlespacing=true]{scrlayer-scrpage}
%\ifoot{\copyright~Norbert Seulberger, 2021}
%\cfoot{}
%\ofoot{\pagemark}
%\pagestyle{scrheadings}
\pagestyle{plain}
%\setkomafont{pageheadfoot}{\small}

% Koma-Script für Inhaltsverzeichnis
\usepackage{tocbasic}
%\DeclareTOCStyleEntries[dynnumwidth=true]{tocline}{chapter,section,subsection}

\usepackage{color}
\usepackage{listings}
\usepackage[fleqn]{amsmath}
%\usepackage{eurosym}
\usepackage{scrhack}
%\usepackage[top=20mm,left=20mm,right=20mm,bottom=30mm]{geometry}
\mathindent=2em

%\usepackage{tikz}
%\usetikzlibrary{shapes.geometric, arrows}

\sloppy

% Umgebungen

\newcommand{\seitenumbruch}{\newpage\textcolor{white}{Seitenumbruch}}

\newcounter{regelsatz}[section]

\newenvironment{ncsEnvironmentEins}[1]{\stepcounter{regelsatz} \textbf{\arabic{section}.\arabic{regelsatz} #1} \\}{}
%\newcommand{\aufgabe}[1]{\begin{ncsEnvironmentEins}{Aufgabe: #1} \end{ncsEnvironmentEins}}
\newcommand{\beispiel}[1]{\begin{ncsEnvironmentEins}{Beispiel: #1} \end{ncsEnvironmentEins}}
\newcommand{\hinweis}[1]{\begin{ncsEnvironmentEins}{Hinweis: #1} \end{ncsEnvironmentEins}}
%\newcommand{\loesung}[1]{\begin{ncsEnvironmentEins}{Lösung: #1} \end{ncsEnvironmentEins}}
%\newcommand{\motivation}[1]{\begin{ncsEnvironmentEins}{Motivation: #1} \end{ncsEnvironmentEins}}
\newcommand{\nurTitel}[1]{\begin{ncsEnvironmentEins}{#1} \end{ncsEnvironmentEins}}
%
\newenvironment{ncsEnvironmentZwei}[2]%
{\stepcounter{regelsatz} \textbf{\arabic{chapter}.\arabic{regelsatz} #1} \\ #2}{}
\newcommand{\titelText}[2]{\begin{ncsEnvironmentZwei}{#1}{#2}\end{ncsEnvironmentZwei}}

%\newenvironment{ncsEnvironment}[2]%
%{\stepcounter{regelsatz} \textbf{\arabic{chapter}.\arabic{regelsatz} #1: #2} \\ }{}
%\newcommand{\ncsEnv}[3]{\begin{ncsEnvironment}{#1}{#2} #3 \end{ncsEnvironment}}
%\newcommand{\empfehlung}[2]{\begin{ncsEnvironment}{Empfehlung}{#1} #2 \end{ncsEnvironment}}
%\newcommand{\frage}[2]{\begin{ncsEnvironment}{Frage}{#1} #2 \end{ncsEnvironment}}
%\newcommand{\refactoring}[2]{\begin{ncsEnvironment}{Refactoring}{#1} #2 \end{ncsEnvironment}}
%\newcommand{\synonym}[2]{\begin{ncsEnvironment}{Synonym}{#1} #2 \end{ncsEnvironment}}
%\newcommand{\syntaxPython}[2]{\begin{minipage}{\textwidth}%
%		\begin{ncsEnvironment}{Syntax in Python}{#1} #2 \end{ncsEnvironment}\end{minipage}}
%\newcommand{\test}[2]{\begin{ncsEnvironment}{Unittest}{#1} #2 \end{ncsEnvironment}}
%\newcommand{\tipp}[2]{\begin{ncsEnvironment}{Tipp}{#1} #2 \end{ncsEnvironment}}
%\newcommand{\warnung}[2]{\begin{ncsEnvironment}{Warnung}{#1} #2 \end{ncsEnvironment}}

\newenvironment{ncsItEnvironment}[2]%
{\stepcounter{regelsatz} \textbf{\arabic{chapter}.\arabic{regelsatz} #1: #2} \\ \begin{itshape}}{\end{itshape}}
\newcommand{\begriff}[2]{\begin{ncsItEnvironment}{Begriff}{#1} #2 \end{ncsItEnvironment}}
%\newcommand{\prinzip}[2]{\begin{ncsItEnvironment}{Prinzip}{#1} #2 \end{ncsItEnvironment}}
%\newcommand{\ncsItEnv}[3]{\begin{ncsItEnvironment}{#1}{#2} #3 \end{ncsItEnvironment}}
% Listings Styles

\definecolor{darkred}{rgb}{0.8,0,0}
\definecolor{darkgreen}{rgb}{0.0,0.5,0}

\lstdefinestyle{syntaxPython}{
	language=python, 				        		
	basicstyle=\linespread{1.2}\ttfamily,
	keywordstyle=\color{darkred}\bfseries,	
	identifierstyle=\color{blue},				
	commentstyle=\color{darkgreen},			
	stringstyle=\ttfamily\color{darkgreen}, 			    		
	showstringspaces=false,
	breaklines=true, 			        		
	numbers=none, 				        		
	numberstyle=\small,		        			
	frame=single, 				        		
	tabsize=4,				            		
	float=htbp,
	captionpos=b,
	aboveskip=2ex,
	belowskip=\medskipamount,
	xleftmargin=.02\textwidth}

\lstdefinestyle{syntaxHTML}{%
	language=HTML,%
	basicstyle=\small\ttfamily,%
	keywordstyle=\color{darkred}\bfseries,%
	identifierstyle=\color{blue},%	
	stringstyle=\color{darkgreen},%
	tabsize=4,%
	showstringspaces=false,%
	numbers=none,%			        		
	numberstyle=\small,%	        			
	frame=single,%
	captionpos=b,%
	xleftmargin=.06\textwidth
} 

\lstset{literate=%
	{Ö}{{\"O}}1
	{Ä}{{\"A}}1
	{Ü}{{\"U}}1
	{ß}{{\ss}}2
	{ü}{{\"u}}1
	{ä}{{\"a}}1
	{ö}{{\"o}}1
}

\lstnewenvironment{codePython}{\lstset{style=syntaxPython}}{}
\lstnewenvironment{codeHTML}{\lstset{style=syntaxHTML}}{}

\usepackage[colorlinks=true,linkcolor=blue]{hyperref}

%\listfiles % Gibt die Versionsnummern der geladenen Pakete in der log-Datei aus.

\begin{document}
	
\title{Mathehelfer Bruchrechnen}
\subtitle{Ein Hackathon für Kinder und Jugendliche}
\author{Norbert Seulberger}
\date{September 2022}

\maketitle

\tableofcontents

\section{Die Aufgabenstellung}

Der Mathehelfer Bruchrechnen kann in der Grundversion Brüche und Kommazahlen ineinander umrechnen. Außerdem können Brüche gekürzt werden.

Die Darstellung der Umrechnungen erfolgt in einem Browser. Dabei werden die Brüche grafisch schön mit Zähler, Nenner und Bruchstrich dargestellt wie in einem Mathematikbuch.

Es sind also beispielsweise folgende Operationen möglich:
\begin{align*}
& \frac{6}{8} = \frac{3}{4} & \\[2ex]
& \frac{3}{4} = 0.75 & \\[2ex]
& 0.8 = \frac{4}{5} & 
\end{align*}


Dabei geben wir die linke Seite fest vor und berechnen die rechte Seite jeweils mit einem Programm. Wir schreiben jeweils eine eigene Funktion, also drei Funktionen, die

\begin{itemize}
	\item einen Bruch kürzt,
	\item einen Bruch in eine Kommazahl umrechnet oder 
	\item eine Kommazahl in einen Bruch umwandelt.
\end{itemize}

Als Programmiersprache verwenden wir Python.

Die Formelzeile tragen wir automatisch in eine kleine Web-Seite ein, die wir uns im Browser ansehen. Dabei beschreiben wir die Formel in einer speziellen HTML-Sprache für mathematische Formeln: MathML.

Die Sprache MathML ist sehr einfach. Allerdings muss man sehr viel Text eingeben, um eine Formel mit MathML zu beschreiben. Daher schreiben wir ein Programm, das die Formel automatisch in MathML ausdrückt.

\section{Algorithmen}
\label{sec:Algorithmus}

Ein Algorithmus ist die eindeutige Beschreibung eines Verfahrens, um eine bestimmte Aufgabe zu lösen.

\subsection*{Beispiel: Das Erweitern eines Bruches}

Ein Bruch wird erweitert, indem der Zähler und der Nenner mit demselben Faktor multipliziert werden. Wird beispielsweise der Bruchs $\frac{2}{5}$ mit dem Faktor $3$ erweitert, so ergibt sich diese Gleichung:
\[
\frac{2}{5} = \frac{2 \cdot 3}{5 \cdot 3} = \frac{6}{15}
\]

\subsection*{Der Algorithmus für das Erweitern}

\begin{itemize}
	\item Nimm einen Bruch und den Faktor, mit dem du den Bruch erweitern willst.
	\item Merke dir den Zähler und den Nenner des Bruchs jeweils in einer Variablen.
	\item Multipliziere die Variable für den Zähler mit dem Erweiterungsfaktor.
	\item Verfahre genauso mit dem Nenner.
	\item Erstelle einen neuen Bruch mit dem neuen Zähler und dem neuen Nenner.
\end{itemize}

%\section{Die fachlichen Themen}
\section{Funktionen}

Eine Funktion ist ein kleines Teilprogramm, das eine spezifische Aufgabe löst. Für bestimmte Eingabewerte erhält man immer dasselbe Ergebnis.

\beispiel{Eine Funktion zur Mittelwertberechnung zweier Zahlen}
Der Algorithmus lautet:

%\include{./Kapitel/Bedingungen}
%\include{./Kapitel/Schleifen}

\section{MathML}

MathML ist eine einfache Sprache, die in einem HTML-Dokument dazu verwendet werden kann, um mathematische Formeln darzustellen.

Genauso wie in HTML werden die Daten in sogenannte Tags eingeschlossen, d.h. ein Element beginnt mit einen Tag, dann stehen die Daten und ein schließendes Tag beendet den Ausdruck. 

\subsection*{MathML: Anfang und Ende einer Formel}

Eine Formeln beginnt mit \texttt{<math>} und endet mit \texttt{</math>}.

Für die verschiedenen Teile einer mathematischen Formel gibt es unterschiedliche Elemente.
\begin{itemize}
	\item \texttt{<mn>}42\texttt{</mn>} stellt die Zahl 42 dar.
	\item \texttt{<mo>}=\texttt{</mo>} beschreibt ein Gleichheitszeichen.
	\item \texttt{<mo>}+\texttt{</mo>} schreibt ein Pluszeichen.
\end{itemize}

Die Formel $2 + 3 = 5$ lautet also in MathML:

\begin{codeHTML}
<math>
	<mn>2</mn>
	<mo>+</mo>
	<mn>3</mn>
	<mo>=</mo>
	<mn>5</mn>
</math>
\end{codeHTML}

\textbf{Tipp}

Der Browser ignoriert die Zeilenumbrüche in MathML. Wir können die Formel also auch in eine Zeile schreiben.

\begin{codeHTML}
<math><mn>2</mn><mo>+</mo><mn>3</mn><mo>=</mo><mn>5</mn></math>
\end{codeHTML}

\subsection*{Brüche in MathML}

Ein Bruch beginnt mit \texttt{<mfrac>} und endet mit \texttt{</mfrac>}. Den Zähler und den Nenner schreiben wir zwischen \texttt{<mn>} und \texttt{</mn>}.

Der Bruch $\frac{2}{3}$ sieht also so aus:

\begin{codeHTML}
<math>
	<mfrac>
		<mn>2</mn>
		<mn>3</mn>
	</mfrac>
</math>
\end{codeHTML}

\subsection*{Eine Formelzeile}

Damit jede Formel in eine eigene Zeile geschrieben wird, verwenden wir das Absatz-Tag von HTML: \texttt{<p>} ... \texttt{</p>}

Also zusammen:
\begin{codeHTML}
<p>
	<math><mn>2</mn><mo>+</mo><mn>3</mn><mo>=</mo><mn>5</mn></math>
</p>
\end{codeHTML}

\subsection*{Eine Anleitung zu MathML}

Eine einfache, aber umfassende Anleitung zu MathML findest du auf folgender Internetseite: \href{https://www.math-it.de/Publikationen/MathML_de.html}{MathML-Tutorial}
%\include{./Kapitel/HTMLDatei}

\section{Aufgabenblatt MathML}

\subsection{Aufgabe: Anzeige der HTML-Datei}

Schau dir im Firefox-Browser die Datei \texttt{manuell.html} an. Die Anzeige sollte etwa so aussehen:

\begin{equation*}
\frac{6}{8} = \frac{3}{4}
\end{equation*}

Hier kannst du die Formeln für das Umrechnen eines Bruchs in eine Dezimalzahl und umgekehrt einfügen.

\subsection{Aufgabe: Ansicht der HTML-Datei im Editor}

Schau dir jetzt dieselbe Datei im Editor der Entwicklungsumgebung an. Der Text sieht etwa so aus:

\begin{codeHTML}{Die Datei manuell.html}
<!doctype html>
<html lang="de">
	<head>
		<meta charset="utf-8">
		<meta name="viewport" content="width=device-width, initial-scale=1.0">
		<title>Bruchrechnen</title>
	</head>
	<body>
		<p>
			<math>
				<mfrac>
					<mi>6</mi>
					<mi>8</mi>
				</mfrac>
				<mo>=</mo>
				<mfrac>
					<mi>3</mi>
					<mi>4</mi>
				</mfrac>
			</math>
		</p>
		<p>
 			Hier kannst du die Formeln für das Umrechnen ...
		</p>
	</body>
</html>
\end{codeHTML}

Kannst du die Element entdecken, die für die Darstellung des Kürzens der beiden Brüche stehen?

\subsection{Aufgabe: Die Beschreibung der beiden Umrechungen}

Füge in die Datei \texttt{manuell.html} den Text in der MathML-Sprache ein, so dass die beiden Umrechungen dargestellt werden (zusätzlich zur Formel für das Kürzen). Der Text „Hier kannst du \dots“ darf überschrieben werden.

\begin{align*}
& \frac{6}{8} = \frac{3}{4} & \\[2ex]
& \frac{3}{4} = 0.75 & \\[2ex]
& 0.8 = \frac{4}{5} & 
\end{align*}

Überprüfe das Ergebnis im Firefox-Browser.
\section{Beispiel: Die Darstellung eines Bruches in MathML}
\label{sec:BruchMathML}

Lies dieses Beispiel sorgfältig durch. Es dient als Vorlage für die Aufgaben, die du danach selbstständig lösen sollst.

\subsection{Der Algorithmus}

Einen Bruch in MathML zu schreiben, ist nicht schwierig, aber mühsam. Daher möchten wir eine Funktion ein Python programmieren, die das automatisch erledigt. Der Algorithmus lautet so:
\begin{itemize}
	\item Nimm einen Bruch entgegen.
	\item Wandle den Zähler des Bruchs in eine Zeichenkette.
	\item Wandle den Nenner des Bruchs in eine Zeichenkette.
	\item Gib der Zeichenkette, die den Bruch in MathML beschreibt, einen Namen.
	\item Baue die Zeichenketteaus Einzelteilen zusammen.
	\item Beginne die Zeichenkette mit \texttt{<mfrac>}.
	\item Rahme die Zeichenkette, die den Zähler beschreibt, mit den Tags \texttt{<mi>} und \texttt{</mi>} ein.
	\item Verfahre ebenso mit der Zeichenkette für den Nenner.
	\item Beende die Zeichenkette mit \texttt{</mfrac>}.
	\item Gib die Zeichenkette zurück.
\end{itemize}

\subsection{Die Funktion}

Die Funktion sieht dann so aus:

\begin{codePython}{Darstellung eines Bruchs in MathML}
def schreibeBruch(bruch: Bruch) -> str:
	zaehlerAlsString = str(bruch.zaehler)
	nennerAlsString = str(bruch.nenner)
	ergebnis = "<mfrac>"
	ergebnis += "<mi>" + zaehlerAlsString + "</mi>"
	ergebnis += "<mi>" + nennerAlsString + "</mi>"
	ergebnis += "</mfrac>"
	return ergebnis
\end{codePython}

\subsection{Das Überprüfen des Ergebnisses}

In der Funktion \texttt{schreibeMathML()} ergänzen wir nun die Zeilen für die Darstellung des Bruchs.

\begin{codePython}{Schreibe die HTML-Datei}
def schreibeMathML() -> str:
	inhalt = "<p><math><mi>2</mi><mo>+</mo><mi>3</mi>"
				+ "<mo>=</mo><mi>5</mi></math></p>"
	bruch: Bruch = Bruch(3,4)
	inhalt += "\n\t\t<p>"
	inhalt += "<math>" + schreibeBruch(bruch) + "</math>"
	inhalt += "</p>"
	return inhalt
\end{codePython}

Danach führen wir die Python-Datei \texttt{htmlErzeugung.py} aus und betrachten im Firefox-Browser die Datei \texttt{index.html}

\section{Beispiel: Das Kürzen von Brüchen}

Lies dieses Beispiel sorgfältig durch. Es dient als Vorlage für die Aufgaben, die du danach selbstständig lösen sollst.

\subsection{Der Algorithmus}

Das Kürzen eines Bruches funktioniert so:
\begin{itemize}
	\item Nimm einen Bruch.
	\item Merke dir den Zähler und den Nenner des Bruchs.
	\item Berechne den größten gemeinsamen Teiler von Zähler und Bruch. (Es gibt dafür eine fertige Funktion.)
	\item Berechne den gekürzten Zähler.
	\item Berechne den gekürzten Nenner.
	\item Erstelle einen neuen Bruch aus dem gekürzten Zähler und dem gekürzten Nenner.
\end{itemize}

\subsection{Die Funktion zum Kürzen von Brüchen}

Die Funktion lautet dann:

\begin{codePython}{Funktion zum Kürzen von Brüchen}
def kuerzeBruch(bruch: Bruch) -> Bruch:
	alterZaehler = bruch.zaehler
	alterNenner = bruch.nenner
	kuerzungsFaktor = ggT(alterZaehler, alterNenner)
	neuerZaehler = alterZaehler // kuerzungsFaktor
	neuerNenner = alterNenner // kuerzungsFaktor
	neuerBruch = Bruch(neuerZaehler, neuerNenner)
	return neuerBruch
\end{codePython}

\subsection{Die Funktion zur Darstellung der Rechnung zum Kürzen}

Der Algorithmus lautet:
\begin{itemize}
	\item Nimm einen Bruch.
	\item Erzeuge aus diesem Bruch einen gekürzten Bruch. Verwende dazu die Funktion, die du eben gesehen hast.
	\item Erzeuge aus dem ursprünglichen Bruch die Beschreibung in MathML. Verwende dazu die Funktion, die du im vorherigen Beispiel gesehen hast.
	\item Erzeuge aus dem gekürzten Bruch die Beschreibung in MathML. Verwende dazu ebenfalls die Funktion aus dem vorherigen Beispiel.
	\item Setze jetzt den MathML-Text aus den einzelnen Angaben zusammen.
	\item Die Zeichenkette beginnt mit <math>.
	\item Es folgt der MathML-Text für den ungekürzten Bruch.
	\item Dann kommt das Gleichheitszeichen.
	\item Es folge der MathML-Text für den gekürzten Bruch.
	\item Die Zeichenkette beginnt mit </math>
\end{itemize}

Die Funktion lautet dann:
\begin{codePython}{MathML für das Kürzen eines Bruchs}{code:schreibeKuerzen}
def schreibeKuerzen(bruch: Bruch) -> str:
	bruchGekuerzt = kuerzeBruch(bruch)
	textUngekuerzt = schreibeBruch(bruch)
	textGekuerzt = schreibeBruch(bruchGekuerzt)
	return "<math>" + textUngekuerzt + "<mo>=</mo>" + textGekuerzt + "</math>"
\end{codePython}

\subsection{Überprüfung des Ergebnisses}

Wir überprüfen das Ergebnis analog zum Vorgehen im Beispiel für die Darstellung eines Bruchs. Dazu ergänzen wir in der Funktion \texttt{schreibeMathML()} die
Zeilen

\begin{codePython}{Integration des Kürzens}
bruchUngekuerzt: Bruch = Bruch(6,8)
inhalt += "\n\t\t<p>"
inhalt += schreibeKuerzen(bruchUngekuerzt)
inhalt += "</p>"
\end{codePython}
\section{Aufgabenblatt: Umwandlung eines Bruchs in eine Dezimalzahl}
\section{Aufgabe: Wandle eine Dezimalzahl in einen Bruch}
\section{Aufgabe: Programmiere die Grundrechenarten für Brüche}

Diese Aufgabe ist etwas schwieriger.
\section{Aufgabe: Gemischte Zahlen}

Diese Aufgabe ist schwierig.
\include{./Kapitel/ObjektorientierteProgrammierung}


\end{document}