\documentclass[12p,numbers=noendperiod,DIV=15]{scrartcl}

\usepackage[ngerman]{babel}
\usepackage[utf8]{inputenc}
\usepackage[T1]{fontenc}

% Schriftart DejaVu
\usepackage{dejavu}
\usepackage[onehalfspacing]{setspace}
%\setstretch{1.5}

\setlength{\parindent}{0pt}
\setlength{\parskip}{2ex plus0.5ex minus0.5ex}

\usepackage{enumitem}
\setlist{noitemsep,topsep=0pt}

% Kopf- und Fußzeile
%\pagestyle{headings}
%\usepackage[singlespacing=true]{scrlayer-scrpage}
%\ifoot{\copyright~Norbert Seulberger, 2021}
%\cfoot{}
%\ofoot{\pagemark}
%\pagestyle{scrheadings}
\pagestyle{plain}
%\setkomafont{pageheadfoot}{\small}

% Koma-Script für Inhaltsverzeichnis
\usepackage{tocbasic}
%\DeclareTOCStyleEntries[dynnumwidth=true]{tocline}{chapter,section,subsection}

\usepackage{color}
\usepackage{listings}
\usepackage[fleqn]{amsmath}
%\usepackage{eurosym}
\usepackage{scrhack}
%\usepackage[top=20mm,left=20mm,right=20mm,bottom=30mm]{geometry}
\mathindent=2em

%\usepackage{tikz}
%\usetikzlibrary{shapes.geometric, arrows}

\sloppy

% Umgebungen

\newcounter{regelsatz}[chapter]

\newenvironment{ncsEnvironmentEins}[1]{\stepcounter{regelsatz} \textbf{\arabic{chapter}.\arabic{section}.\arabic{regelsatz} #1} \\}{}
%\newcommand{\aufgabe}[1]{\begin{ncsEnvironmentEins}{Aufgabe: #1} \end{ncsEnvironmentEins}}
\newcommand{\beispiel}[1]{\begin{ncsEnvironmentEins}{Beispiel: #1} \end{ncsEnvironmentEins}}
\newcommand{\hinweis}[1]{\begin{ncsEnvironmentEins}{Hinweis: #1} \end{ncsEnvironmentEins}}
%\newcommand{\loesung}[1]{\begin{ncsEnvironmentEins}{Lösung: #1} \end{ncsEnvironmentEins}}
%\newcommand{\motivation}[1]{\begin{ncsEnvironmentEins}{Motivation: #1} \end{ncsEnvironmentEins}}
\newcommand{\nurTitel}[1]{\begin{ncsEnvironmentEins}{#1} \end{ncsEnvironmentEins}}
%

\newenvironment{ncsEnvironmentZwei}[2]%
{\stepcounter{regelsatz} \textbf{\arabic{chapter}.\arabic{regelsatz} #1} \\ #2}{}
\newcommand{\titelText}[2]{\begin{ncsEnvironmentZwei}{#1}{#2}\end{ncsEnvironmentZwei}}

\newenvironment{ncsItEnvironment}[2]%
{\stepcounter{regelsatz} \textbf{\arabic{chapter}.\arabic{regelsatz} #1: #2} \\ \begin{itshape}}{\end{itshape}}
\newcommand{\begriff}[2]{\begin{ncsItEnvironment}{Begriff}{#1} #2 \end{ncsItEnvironment}}
% Listings Styles

\definecolor{darkred}{rgb}{0.8,0,0}
\definecolor{darkgreen}{rgb}{0.0,0.5,0}

\lstdefinestyle{syntaxZeile}{
	language=python, 				        		
	basicstyle=\ttfamily,
	keywordstyle=\color{darkred}\bfseries,	
	identifierstyle=\color{blue},				
	commentstyle=\color{darkgreen},			
	stringstyle=\ttfamily\color{darkgreen}, 			    		
	showstringspaces=false,	        		
	numbers=none, 	        			
	frame=none, 				        		
	tabsize=2,				            		
	float=htbp,
	captionpos=b}

\lstdefinestyle{syntaxPython}{
	language=python, 				        		
	basicstyle=\linespread{1.2}\ttfamily,
	keywordstyle=\color{darkred}\bfseries,	
	identifierstyle=\color{blue},				
	commentstyle=\color{darkgreen},			
	stringstyle=\ttfamily\color{darkgreen}, 			    		
	showstringspaces=false,
	breaklines=true, 			        		
	numbers=left, 				        		
	numberstyle=\small,		        			
	frame=single, 				        		
	tabsize=2,				            		
	float=htbp,
	captionpos=b,
	aboveskip=2ex,
	belowskip=\medskipamount,
	xleftmargin=.06\textwidth}

\lstdefinestyle{syntaxHTML}{%
	language=HTML,%
	basicstyle=\small\ttfamily,%
	keywordstyle=\color{darkred}\bfseries,%
	identifierstyle=\color{blue},%	
	stringstyle=\color{darkgreen},%
	tabsize=2,%
	showstringspaces=false,%
	numbers=left,%			        		
	numberstyle=\small,%	        			
	frame=single,%
	captionpos=b,%
	xleftmargin=.06\textwidth
} 

\lstset{literate=%
	{Ö}{{\"O}}1
	{Ä}{{\"A}}1
	{Ü}{{\"U}}1
	{ß}{{\ss}}2
	{ü}{{\"u}}1
	{ä}{{\"a}}1
	{ö}{{\"o}}1
}

\lstnewenvironment{codePython}[1]{\lstset{style=syntaxPython,caption=#1}}{}
\lstnewenvironment{codeHTML}[1]{\lstset{style=syntaxHTML,caption=#1}}{}
%\input{listings-python.prf}
%\input{TikzStyles}

\usepackage[colorlinks=true,linkcolor=blue]{hyperref}

%\listfiles % Gibt die Versionsnummern der geladenen Pakete in der log-Datei aus.

\begin{document}
	
\title{Mathehelfer Bruchrechnen}
\subtitle{Ein Hackathon für Kinder und Jugendliche}
\author{Norbert Seulberger, Dataport AöR}
\date{September 2022}

\maketitle[-1]

\tableofcontents

\section{Die Aufgabenstellung}

Der Mathehelfer Bruchrechnen kann in der Grundversion Brüche kürzen, erweitern und miteinander multiplizieren.

Die Darstellung der Umrechnungen erfolgt in einem Browser. Dabei werden die Brüche grafisch schön mit Zähler, Nenner und Bruchstrich dargestellt wie in einem Mathematikbuch.

\begin{align*}
& \frac{6}{8} = \frac{3}{4} & \\[2ex]
& \frac{3}{4} = \frac{9}{12} & \\[2ex]
& \frac{3}{4} \cdot \frac{2}{3} = \frac{1}{2} & 
\end{align*}


Dabei geben wir die linke Seite vor und berechnen die rechte Seite jeweils mit einem Programm. Wir schreiben jeweils eine eigene Funktion, also drei Funktionen, die

\begin{itemize}
	\item einen Bruch kürzt,
	\item einen Bruch mit einem beliebigen Faktor erweitert oder
	\item zwei Brüche miteinander multipliziert.
\end{itemize}

Als Programmiersprache verwenden wir Python.

Die Formelzeile tragen wir automatisch in eine kleine Web-Seite ein, die wir uns im Browser ansehen. Dabei beschreiben wir die Formel in einer speziellen HTML-Sprache für mathematische Formeln: MathML.

Die Sprache MathML ist sehr einfach. Allerdings muss man sehr viel Text eingeben, um eine Formel mit MathML zu beschreiben. Daher schreiben wir ein Programm, das die Formel automatisch in MathML ausdrückt.

\section{Algorithmen}
\label{sec:Algorithmus}

Ein Algorithmus ist die eindeutige Beschreibung eines Verfahrens, um eine bestimmte Aufgabe zu lösen.

\subsection*{Beispiel: Das Erweitern eines Bruches}

Ein Bruch wird erweitert, indem der Zähler und der Nenner mit demselben Faktor multipliziert werden. Wird beispielsweise der Bruchs $\frac{2}{5}$ mit dem Faktor $3$ erweitert, so ergibt sich diese Gleichung:
\[
\frac{2}{5} = \frac{2 \cdot 3}{5 \cdot 3} = \frac{6}{15}
\]

\subsection*{Der Algorithmus für das Erweitern}

\begin{itemize}
	\item Nimm einen Bruch und den Faktor, mit dem du den Bruch erweitern willst.
	\item Merke dir den Zähler und den Nenner des Bruchs jeweils in einer Variablen.
	\item Multipliziere die Variable für den Zähler mit dem Erweiterungsfaktor.
	\item Verfahre genauso mit dem Nenner.
	\item Erstelle einen neuen Bruch mit dem neuen Zähler und dem neuen Nenner.
\end{itemize}

%\section{Die fachlichen Themen}
\section{Funktionen}

Eine Funktion ist ein kleines Teilprogramm, das eine spezifische Aufgabe löst. Eine Funktion arbeitet mit folgenden Schritten:
\begin{itemize}
	\item Die Funktion nimmt einen oder mehrere Eingabewerte entgegen.
	\item Die Funktion führt eine Berechnung oder andere Aktivität durch.
	\item Die Funktion gibt das Ergebnis der Berechnung zurück.
\end{itemize}

Für die gleichen Eingabewerte erhält man immer dasselbe Ergebnis. Die Eingabewerte heißen Parameter. Es kann einen oder mehrere Parameter geben, manchmal sogar gar keinen.

\subsection*{Der Aufbau einer Funktion in Python}

Der Aufbau einer Funktion orientiert sich an den drei Schritte, die wir eben kennen gelernt haben:
\begin{itemize}
	\item die Signaturzeile
	\item der Rumpf der Funktion 
	\item der Rückgabewert
\end{itemize}

\subsection*{Die Signaturzeile}

Die Signaturzeile nennt den Namen und beschreibt die Parameter.
\begin{codePython}
def kuerzeBruch(bruch):
\end{codePython}

\begin{itemize}
	\item Zuerst steht immer das Wort \texttt{def}.
	\item Dann folgt der Name der Funktion. Der Name sollte ausdrücken, welche Berechnung die Funktion durchführt.
	\item In den runden Klammer stehen die Eingabewerte für die Funktion. Hier ist es ein Parameter mit dem Namen \texttt{bruch}. Die Funktion kann den Wert des Parameters für die Berechnung nutzen.
\end{itemize}

\subsection*{Tipp: Datentypen angeben}

Wer möchte, kann die Typen der Parameter und des Ergebnisses angeben. Es hilft, Fehler zu vermeiden. Programmierprofis tun das.

Die Signaturzeile für die Funktion zum Kürzen lautet dann:

\lstset{style=syntaxPython}
\begin{lstlisting}
def kuerzeBruch(bruch: Bruch) -> Bruch:
\end{lstlisting}

Die Funktion nimmt also einen Parameter entgegen, dessen Name \texttt{bruch} lautet (klein geschrieben!) und dessen Datentyp ein \texttt{Bruch} ist (groß geschrieben!).

Die Funktion führt eine Berechnung durch, bei der ein Rückgabewert berechnet wird, der vom Datentyp \texttt{Bruch} ist. Das steht hier: \texttt{-> Bruch}.

\subsection*{Der Rumpf einer Funktion}

Der Rumpf der Funktion ist eingerückt. Wir benutzen dafür vier Leerzeichen.

Hier wird die Berechnung durchgeführt. Das erklären wir ausführlich im nächsten Beispiel.


\subsection*{Der Rückgabewert}

Die letzte Zeile im Rumpf gibt das Ergebnis zurück. Sie beginnt mit dem Wort
\begin{quote}
\texttt{return}
\end{quote}


%\include{./Kapitel/Bedingungen}
%\include{./Kapitel/Schleifen}

\section{MathML}

MathML ist eine einfache Markup-Sprache, die in einem HTML-Dokument dazu verwendet werden kann, um mathematische Formeln darzustellen.

Genauso wie in HTML werden die Daten in sogenannte Tags eingeschlossen, d.h. ein Element beginnt mit einen Tag, dann stehen die Daten und ein schließendes Tag beendet den Ausdruck. 

\subsection{Operanden und Operatoren}

Für die verschiedenen Teile einer mathematischen Formel gibt es unterschiedliche Elemente.
\begin{itemize}
	\item \texttt{<mn>}42\texttt{</mn>} stellt die Zahl 42 dar.
	\item \texttt{<mo>}=\texttt{</mo>} beschreibt ein Gleichheitszeichen.
	\item \texttt{<mo>}+\texttt{</mo>} schreibt ein Pluszeichen.
\end{itemize}

Die Formel $2 + 3 = 5$ lautet also in MathML:
\begin{codeHTML}{Eine einfache Addition}
<mn>2</mn>
<mo>+</mo>
<mn>3</mn>
<mo>=</mo>
<mn>5</mn>
\end{codeHTML}

\nurTitel{Tipp}
Der Browser ignoriert die Zeilenumbrüche in MathML. Wir können die Formel also auch in eine Zeile schreiben.

\begin{codeHTML}{Die Formel in einer Zeile}
<mn>2</mn><mo>+</mo><mn>3</mn><mo>=</mo><mn>5</mn>
\end{codeHTML}

\subsection{Brüche}

\subsection{Eine Formelzeile}

Eine Formel wird wie folgt geschrieben: \texttt{<math>} Hier steht die Formel. \texttt{</math>}

Damit die Formel in eine eigene Zeile geschrieben wird, verwenden wir das Absatz-Tag von HTML: \texttt{<p>} ... \texttt{</p>}

Also zusammen:
\begin{codeHTML}{Eine Formelzeile}
<p>
	<math>
		Hier steht die Formel.
	</math>
</p>
\end{codeHTML}
%\include{./Kapitel/HTMLDatei}

\section{Aufgabenblatt MathML}

\subsection*{Aufgabe: Anzeige der HTML-Datei}

Schau dir im Firefox-Browser die Datei \texttt{manuell.html} an. Die Anzeige sollte etwa so aussehen:
\[
\frac{6}{8} = \frac{3}{4}
\]

„Hier kannst du die Formeln für das Umrechnen eines Bruchs in eine Dezimalzahl und umgekehrt einfügen.“

\subsection*{Aufgabe: Ansicht der HTML-Datei im Editor}

Schau dir jetzt dieselbe Datei im Editor der Entwicklungsumgebung an. Der Text sieht etwa so aus:

\begin{codeHTML}
<!doctype html>
<html lang="de">
	<head>
		<meta charset="utf-8">
		<meta name="viewport" content="width=device-width, initial-scale=1.0">
		<title>Bruchrechnen</title>
	</head>
	<body>
		<p>
			<math>
				<mfrac>
					<mn>6</mn>
					<mn>8</mn>
				</mfrac>
				<mo>=</mo>
				<mfrac>
					<mn>3</mn>
					<mn>4</mn>
				</mfrac>
			</math>
		</p>
		<p>
 			Hier kannst du die Formeln für das Umrechnen ...
		</p>
	</body>
</html>
\end{codeHTML}

Kannst du die Element entdecken, die für die Darstellung des Kürzens der beiden Brüche stehen?

\subsection*{Aufgabe: Die Umrechung eines Bruchs in eine Dezimalzahl}

Füge in die Datei \texttt{manuell.html} den Text in der MathML-Sprache ein, so dass die Umrechnung eines Bruchs in eine Dezimalzahl dargestellt wird. Der Text „Hier kannst du \dots“ darf überschrieben werden.

Konkret soll die letzte Zeile so aussehen.
\[
\frac{3}{4} = 0.75
\]

Überprüfe das Ergebnis im Firefox-Browser. Die Seite sollte jetzt ungefähr so aussehen:
\begin{align*}
& \frac{6}{8} = \frac{3}{4} & \\[2ex]
& \frac{3}{4} = 0.75 &
\end{align*}

\subsection*{Aufgabe: Die Umrechnung einer Dezimalzahl in einen Bruch}

Füge in die Datei \texttt{manuell.html} den Text in der MathML-Sprache ein, so dass diese Zeile als letzte Zeile angezeigt wird:
\[
0.8 = \frac{4}{5}
\]

Überprüfe das Ergebnis im Firefox-Browser. Die Seite sollte jetzt ungefähr so aussehen:
\begin{align*}
& \frac{6}{8} = \frac{3}{4} & \\[2ex]
& \frac{3}{4} = 0.75 & \\[2ex]
& 0.8 = \frac{4}{5} & 
\end{align*}






\end{document}