\section{Beispiel: Das Kürzen von Brüchen}

Lies dieses Beispiel sorgfältig durch. Es dient als Vorlage für die Aufgaben, die du danach selbstständig lösen sollst.

\subsection{Der Algorithmus}

Das Kürzen eines Bruches funktioniert so:
\begin{itemize}
	\item Nimm einen Bruch.
	\item Merke dir den Zähler und den Nenner des Bruchs.
	\item Berechne den größten gemeinsamen Teiler von Zähler und Bruch. (Es gibt dafür eine fertige Funktion.)
	\item Berechne den gekürzten Zähler.
	\item Berechne den gekürzten Nenner.
	\item Erstelle einen neuen Bruch aus dem gekürzten Zähler und dem gekürzten Nenner.
\end{itemize}

\subsection{Die Funktion zum Kürzen von Brüchen}

\subsection{Die Funktion zur Darstellung eines Bruches in MathML}