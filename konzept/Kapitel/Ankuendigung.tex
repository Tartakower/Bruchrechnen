\section{Programmiere den Mathehelfer für das Bruchrechnen!}

Zukünftig wird der Computer deine Hausaufgaben im Bruchrechnen kontrollieren. Na ja, fast $\dots$

Wir bringen dem Computer die ersten Schritte im Bruchrechnen bei, nämlich das Kürzen, Erweitern und Multiplizieren von Brüchen. Dazu formulieren wir die Algorithmen (also das Kochrezept, wie es geht) und programmieren Funktionen in der Programmiersprache Python, die diese Berechnungen durchführen.

Damit die Rechnungen schön aussehen, schreiben wir die Gleichungen mit MathML (das ist eine Erweiterung von HTML) und prüfen das Ergebnis im Browser.

\subsection{Schwierigkeitsstufen}

Wir bearbeiten Aufgaben in verschiedenen Schwierigkeitsstufen, von einfach bis sehr schwierig. Du musst also keine Programmierkenntnisse haben, um mit Spaß mitmachen zu können.

Richtige Programmierfreaks kommen auch auf ihre Kosten! Wir werden euch Dinge zeigen, die ihr vermutlich noch nicht gesehen habt. Die Wette gilt.

\subsection{Vorkenntnisse}

Du solltest relativ flüssig Text in einem Editor auf dem Computer bearbeiten können.

Von Vorteil ist es, wenn du bestimmte Vorkenntnisse mitbringst:
\begin{itemize}
	\item Du weißt, wie eine HTML-Datei aufgebaut ist und hast schon Text in HTML geschrieben.
	\item Du hast schon kleine Programme in Python programmiert.
\end{itemize}

Wenn du diese Vorkenntnisse besitzt, kannst du schwierigere Aufgaben lösen!

