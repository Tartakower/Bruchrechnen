\section{MathML}

MathML ist eine einfache Markup-Sprache, die in einem HTML-Dokument dazu verwendet werden kann, um mathematische Formeln darzustellen.

Genauso wie in HTML werden die Daten in sogenannte Tags eingeschlossen, d.h. ein Element beginnt mit einen Tag, dann stehen die Daten und ein schließendes Tag beendet den Ausdruck. 

\subsection{Operanden und Operatoren}

Für die verschiedenen Teile einer mathematischen Formel gibt es unterschiedliche Elemente.
\begin{itemize}
	\item \texttt{<mn>}42\texttt{</mn>} stellt die Zahl 42 dar.
	\item \texttt{<mo>}=\texttt{</mo>} beschreibt ein Gleichheitszeichen.
	\item \texttt{<mo>}+\texttt{</mo>} schreibt ein Pluszeichen.
\end{itemize}

Die Formel $2 + 3 = 5$ lautet also in MathML:
\begin{codeHTML}{Eine einfache Addition}
<mn>2</mn>
<mo>+</mo>
<mn>3</mn>
<mo>=</mo>
<mn>5</mn>
\end{codeHTML}

\nurTitel{Tipp}
Der Browser ignoriert die Zeilenumbrüche in MathML. Wir können die Formel also auch in eine Zeile schreiben.

\begin{codeHTML}{Die Formel in einer Zeile}
<mn>2</mn><mo>+</mo><mn>3</mn><mo>=</mo><mn>5</mn>
\end{codeHTML}

\subsection{Brüche}

\subsection{Eine Formelzeile}

Eine Formel wird wie folgt geschrieben: \texttt{<math>} Hier steht die Formel. \texttt{</math>}

Damit die Formel in eine eigene Zeile geschrieben wird, verwenden wir das Absatz-Tag von HTML: \texttt{<p>} ... \texttt{</p>}

Also zusammen:
\begin{codeHTML}{Eine Formelzeile}
<p>
	<math>
		Hier steht die Formel.
	</math>
</p>
\end{codeHTML}