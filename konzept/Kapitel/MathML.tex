\section{MathML}

MathML ist eine einfache Sprache, die in einem HTML-Dokument dazu verwendet werden kann, um mathematische Formeln darzustellen.

Genauso wie in HTML werden die Daten in sogenannte Tags eingeschlossen, d.h. ein Element beginnt mit einen Tag, dann stehen die Daten und ein schließendes Tag beendet den Ausdruck. 

\subsection*{MathML: Anfang und Ende einer Formel}

Eine Formeln beginnt mit \texttt{<math>} und endet mit \texttt{</math>}.

Für die verschiedenen Teile einer mathematischen Formel gibt es unterschiedliche Elemente.
\begin{itemize}
	\item \texttt{<mn>42</mn>} stellt die Zahl 42 dar.
	\item \texttt{<mo>}=\texttt{</mo>} beschreibt ein Gleichheitszeichen.
	\item \texttt{<mo>}+\texttt{</mo>} schreibt ein Pluszeichen.
	\item \texttt{<mo>\&middot;</mo>} schreibt das Multiplikationszeichen.
\end{itemize}

Die Formel $2 + 3 = 5$ lautet also in MathML:

\begin{codeHTML}
<math>
	<mn>2</mn>
	<mo>+</mo>
	<mn>3</mn>
	<mo>=</mo>
	<mn>5</mn>
</math>
\end{codeHTML}

\textbf{Tipp}

Der Browser ignoriert die Zeilenumbrüche in MathML. Wir können die Formel also auch in eine Zeile schreiben.

\begin{codeHTML}
<math><mn>2</mn><mo>+</mo><mn>3</mn><mo>=</mo><mn>5</mn></math>
\end{codeHTML}

\subsection*{Brüche in MathML}

Ein Bruch beginnt mit \texttt{<mfrac>} und endet mit \texttt{</mfrac>}. Den Zähler und den Nenner schreiben wir zwischen \texttt{<mn>} und \texttt{</mn>}.

Der Bruch $\frac{2}{3}$ sieht also so aus:

\begin{codeHTML}
<math>
	<mfrac>
		<mn>2</mn>
		<mn>3</mn>
	</mfrac>
</math>
\end{codeHTML}

\subsection*{Eine Formelzeile}

Damit jede Formel in eine eigene Zeile geschrieben wird, verwenden wir das Absatz-Tag von HTML: \texttt{<p>} ... \texttt{</p>}

Also zusammen:
\begin{codeHTML}
<p>
	<math><mn>2</mn><mo>+</mo><mn>3</mn><mo>=</mo><mn>5</mn></math>
</p>
\end{codeHTML}

\subsection*{Eine Anleitung zu MathML}

Eine einfache, aber umfassende Anleitung zu MathML findest du auf folgender Internetseite: \href{https://www.math-it.de/Publikationen/MathML_de.html}{MathML-Tutorial}