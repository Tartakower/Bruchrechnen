\section{Aufgabe: Der Kehrwert eines Bruches}

\subsection*{Der Kehrwert eines Bruches}

Der Kehrwert eines Bruches ist ein anderer Bruch, dessen Zähler der Nenner des ursprünglichen Bruchs und dessen Nenner der ursprüngliche Zähler ist. Beispielsweise lautet der Kehrwert des Bruchs $\frac{2}{3}$ demzufolge $\frac{3}{2}$.

\subsection*{Der Algorithmus}

Wie lautet der Algorithmus?
{\Large
\begin{itemize}
	\item Nimm einen Bruch.
	\item $\dots$
	\item $\dots$
	\item $\dots$
	\item $\dots$
\end{itemize}
}

\subsection*{Die Funktion zur Berechung des Kehrwerts}

Die Signatur der Funktion lautet:

\begin{codePython}
def berechneKehrwert(bruch: Bruch) -> Bruch:
\end{codePython}

Wie geht es weiter?

\subsection*{Überprüfung des Ergebnisses}

Erzeuge in der Datei \texttt{spielwiese.py} einen Bruch, z.B. $\frac{2}{5}$. Berechne den Kehrwertbruch und lasse diesen anzeigen.

\begin{codePython}
bruch_1 = Bruch(2,5)
kehrwert = berechneKehrwert(bruch_1)
print(kehrwert)
\end{codePython}

\subsection*{Weshalb ist der Kehrwert wichtig?}

Wir benötigen den Kehrwert beim Dividieren von Brüchen. Es gibt eine bekannte Rechenregel: Ein Bruch wird durch einen anderen Bruch dividiert, indem man den ersten Bruch mit dem Kehrwert des zweiten Bruchs multipliziert:
\[
\frac{2}{5} : \frac{1}{2} = \frac{2}{5} \cdot \frac{2}{1} = \frac{4}{5}
\]

