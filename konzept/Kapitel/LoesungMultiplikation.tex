\section{Lösung: Das Multiplizieren von Brüchen}

\subsection*{Der Algorithmus zum Multiplizieren von Brüchen}

\begin{itemize}
	\item Nimm zwei Brüche entgegen.
	\item Merke dir den Zähler des ersten Bruchs in einer Variablen.
	\item Merke dir den Nenner des ersten Bruchs in einer zeiten Variablen.
	\item Verfahre ebenso mit dem zweiten Bruch. Bisher hast du vier Variablen.
	\item Multipliziere die beiden Zähler-Variablen und merke dir das Ergebnis in einer Variablen.
	\item Multipliziere die beiden Nenner-Variablen.
	\item Erzeuge einen Bruch aus den Produkten der Zähler und der Nenner und merke dir das Ergebnis in einer Variablen. 
	\item Wichtig! Kürze den soeben erstellten Bruch.
	\item Gib den neuen, gekürzten Bruch zurück.
\end{itemize}

\subsection*{Die Funktion zum zum Multiplizieren von Brüchen}

\begin{codePython}
def multipliziereBrueche(erster_faktor: Bruch, zweiter_faktor: Bruch) -> Bruch:
	erster_zaehler = erster_faktor.zaehler
	erster_nenner = erster_faktor.nenner
	zweiter_zaehler = zweiter_faktor.zaehler
	zweiter_nenner = zweiter_faktor.nenner
	produkt_zaehler = erster_zaehler * zweiter_zaehler
	produkt_nenner = erster_nenner * zweiter_nenner
	produkt = Bruch(produkt_zaehler, produkt_nenner)
	produkt = kuerzeBruch(produkt)
	return produkt
\end{codePython}

\subsection*{Überprüfung der Funktion}

Ergänze in der Datei \texttt{uebungsplatz.py}:

\begin{codePython}
	erster_faktor = Bruch(3,4)
	zweiter_faktor = Bruch(2,3)
	produkt = mathehelfer.multipliziereBrueche(erster_faktor, zweiter_faktor)
	print("Produkt: " + str(produkt))
\end{codePython}

Führe \texttt{uebungsplatz.py} als Python-Skript aus. Die Ausgabe muss folgende Zeile enthalten:
\begin{quote}
	Produkt: Bruch(zaehler=1, nenner=2)
\end{quote}

\subsection*{Der Algorithmus zur Darstellung der Multiplikationsformel}

\begin{itemize}
	\item Nimm zwei Brüche entgegen.
	\item Berechne das Produkt der beiden Brüche mit der Funktion \texttt{multipliziereBrueche}.
	\item Erstelle für den ersten Bruch die MathML-Darstellung.
	\item Erstelle für den zweiten Bruch die MathML-Darstellung.
	\item Erstelle für das Produkt die MathML-Darstellung.
	\item Setze jetzt den MathML-Text aus den drei MathML-Texten der Brüche zusammen.
	\item Zwischen dem ersten und dem zweiten Bruch steht noch das Multiplikationszeichen.
	\item Zwischen dem zweiten Bruch und dem Produkt steht das Gleichheitszeichen.
	\item Gib den gesamten MathML-Text zurück.
\end{itemize}

\subsection*{Die Funktion zur Darstellung der Multiplikationsformel}

\begin{codePython}
def schreibeMultiplikation(erster_faktor: Bruch, zweiter_faktor: Bruch) -> str:
	text_erster_faktor = schreibeBruch(erster_faktor)
	text_zweiter_faktor = schreibeBruch(zweiter_faktor)
	produkt = multipliziereBrueche(erster_faktor, zweiter_faktor)
	text_produkt = schreibeBruch(produkt)
	ergebnis = text_erster_faktor + "<mo>&middot;</mo>" + text_zweiter_faktor + "<mo>=</mo>" + text_produkt
	return ergebnis
\end{codePython}

\subsection*{Überprüfung der Darstellung}

Ergänze in der Funktion \texttt{mathehelfer.schreibeMathML()} die Zeilen für das Multiplizieren:

\begin{codePython}
erster_faktor = Bruch(3,4)
zweiter_faktor = Bruch(2,3)
inhalt = inhalt + "\n\t\t<p><math>"
inhalt = inhalt + schreibeMultiplikation(erster_faktor, zweiter_faktor)
inhalt = inhalt + "</math></p>"
\end{codePython}

Führe die Datei \texttt{htmlErzeugung.py} aus.

Prüfe im Firefox-Browser, wie die Datei \texttt{index.html} aussieht. Dort sollte eine Zeile stehen, die so aussieht:
\[
\frac{3}{4} \cdot \frac{2}{3} = \frac{1}{2}
\]
