\section{Zeitliche Planung}

Der zeitliche Rahmen besteht aus einer zweitägigen Veranstaltung mit jeweils vier Stunden Dauer.

Die TeilnehmerInnen arbeiten in Zweier-Teams an einem Rechner. Sie dürfen entscheiden, wie selbstständig sie arbeiten möchten. Ein möglichst eigenständiges Vorgehen ist wünschenswert.

Es sind ausreichend Pausen vorzusehen. Die TeilnehmerInnen dürfen Pausen selbst bestimmen, werden von den Betreuern daran erinnert.

\subsection*{Tag 1}

\begin{itemize}
	\item Begrüßung durch einen Inspirer der Hacker School (15 min, 15/240)
	\item Begrüßung durch das Veranstalterteam / die Inspirer des Veranstalters. Vorstellung der TeilnehmerInnen einschließlich Vorkenntnisse und Erwartungen (30 min, 45/240)
	\item MathML: Durchlesen der Unterlagen zu MathML, Lösen der Aufgabe (45 min, 90/240)
	\item Pause (15 min, 105/240)
	\item Algorithmen und Funktionen: Durchlesen der Unterlagen (20 min, 125/240)
	\item Nachvollziehen und Umsetzen des Beispiels Kürzen (20 min, 145/240)
	\item Pause (15 min, 160/240)
	\item Bearbeiten der Aufgaben zum Erweitern und der Multiplikation (40 min, 200/240)
	\item Vorstellen von Teamlösungen, Code-Reviews (30 min, 230/240)
	\item Feedback-Runde (10 min, 240/240)
\end{itemize}

\subsection*{Tag 2}

\begin{itemize}
	\item Ankommen, Feedback zu Vortag (15 min, 15/240)
	\item Optional: Fertigstellen der Aufgaben vom Vortag (30 min, 45/240)
	\item Zeichenketten: Durchlesen der Unterlagen, Nachvollziehen und Umsetzen des Beispiels \textit{Bruch in MathML} (30 min, 45/240)
	\item Nachvollziehen und Umsetzen des Beispiels \textit{Kürzen eines Bruchs in MathML} (60 min, 105/240)
	\item dazwischen: Pause (15 min, 120/240)
	\item Weitere Aufgaben: Erweitern oder Multiplizieren in MathML (60 min, 180/240)
	\item dazwischen: Pause (15 min, 195/240)
	\item Vorstellen von Teamlösungen, Code-Reviews (30 min, 225/240)
	\item Feedback-Runde und Abschluss (15 min, 240/240)	
\end{itemize}