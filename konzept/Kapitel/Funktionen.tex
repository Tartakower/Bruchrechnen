\section{Funktionen}

Eine Funktion ist ein kleines Teilprogramm, das eine spezifische Aufgabe löst. Eine Funktion arbeitet mit folgenden Schritten:
\begin{itemize}
	\item Die Funktion nimmt einen oder mehrere Eingabewerte entgegen.
	\item Die Funktion führt eine Berechnung oder andere Aktivität durch.
	\item Die Funktion gibt das Ergebnis der Berechnung zurück.
\end{itemize}

Für die gleichen Eingabewerte erhält man immer dasselbe Ergebnis. Die Eingabewerte heißen Parameter. Es kann einen oder mehrere Parameter geben, manchmal sogar gar keinen.

\subsection*{Der Aufbau einer Funktion in Python}

Der Aufbau einer Funktion orientiert sich an den drei Schritte, die wir eben kennen gelernt haben:
\begin{itemize}
	\item die Signaturzeile
	\item der Rumpf der Funktion 
	\item der Rückgabewert
\end{itemize}

\subsection*{Die Signaturzeile}

Die Signaturzeile nennt den Namen und beschreibt die Parameter.
\begin{codePython}
def kuerzeBruch(bruch):
\end{codePython}

\begin{itemize}
	\item Zuerst steht immer das Wort \texttt{def}.
	\item Dann folgt der Name der Funktion. Der Name sollte ausdrücken, welche Berechnung die Funktion durchführt.
	\item In den runden Klammer stehen die Eingabewerte für die Funktion. Hier ist es ein Parameter mit dem Namen \texttt{bruch}. Die Funktion kann den Wert des Parameters für die Berechnung nutzen.
\end{itemize}

\subsection*{Tipp: Datentypen angeben}

Wer möchte, kann die Typen der Parameter und des Ergebnisses angeben. Es hilft, Fehler zu vermeiden. Programmierprofis tun das.

Die Signaturzeile für die Funktion zum Kürzen lautet dann:

\lstset{style=syntaxPython}
\begin{lstlisting}
def kuerzeBruch(bruch: Bruch) -> Bruch:
\end{lstlisting}

Die Funktion nimmt also einen Parameter entgegen, dessen Name \texttt{bruch} lautet (klein geschrieben!) und dessen Datentyp ein \texttt{Bruch} ist (groß geschrieben!).

Die Funktion führt eine Berechnung durch, bei der ein Rückgabewert berechnet wird, der vom Datentyp \texttt{Bruch} ist. Das steht hier: \texttt{-> Bruch}.

\subsection*{Der Rumpf einer Funktion}

Der Rumpf der Funktion ist eingerückt. Wir benutzen dafür vier Leerzeichen.

Hier wird die Berechnung durchgeführt. Das erklären wir ausführlich im nächsten Beispiel.


\subsection*{Der Rückgabewert}

Die letzte Zeile im Rumpf gibt das Ergebnis zurück. Sie beginnt mit dem Wort
\begin{quote}
\texttt{return}
\end{quote}

