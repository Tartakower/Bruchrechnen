\section{Funktionen}

Eine Funktion ist ein kleines Teilprogramm, das eine spezifische Aufgabe löst. Eine Funktion arbeitet mit folgenden Schritten:
\begin{itemize}
	\item Die Funktion nimmt einen oder mehrere Eingabewerte entgegen.
	\item Die Funktion führt eine Berechnung oder andere Aktivität durch.
	\item Die Funktion gibt das Ergebnis der Berechnung zurück.
\end{itemize}

Für bestimmte Eingabewerte erhält man immer dasselbe Ergebnis. Die Eingabewerte heißen Parameter. Die Anzahl und der Typ der Parameter ist festgelegt.

\subsection{Der Aufbau einer Funktion in Python}

Der Aufbau einer Funktion orientiert sich an den drei Schritte, die wir eben kennen gelernt haben:
\begin{itemize}
	\item Die Signaturzeile nennt den Namen und beschreibt die Parameter.
	\item Der Rumpf der Funktion ist eingerückt. Hier wird die Berechnung durchgeführt.
	\item Die letzte Zeile im Rumpf gibt das Ergebnis zurück. Sie beginnt mit dem Wort \texttt{return}.
\end{itemize}

\subsection{Eine Beispielfunktion: Das Erweitern eines Bruchs}

In \ref{sec:Algorithmus} haben wir den Algorithmus zum Erweitern eines Bruchs formuliert:
\begin{itemize}
	\item Nimm einen Bruch und den Faktor, mit dem du den Bruch erweitern willst.
	\item Merke dir den Zähler und den Nenner des Bruchs jeweils in einer Variablen.
	\item Multipliziere die Variable für den Zähler mit dem Erweiterungsfaktor.
	\item Verfahre genauso mit dem Nenner.
	\item Erstelle einen neuen Bruch mit dem neuen Zähler und dem neuen Nenner.
\end{itemize}

In Python lautet die Funktion für das Erweitern:
\begin{codePython}
def erweitereBruch(bruch, faktor):
	alterZaehler = bruch.zaehler
	alterNenner = bruch.nenner
	neuerZaehler = alterZaehler * faktor
	neuerNenner = alterNenner * faktor
	neuerBruch = Bruch(neuerZaehler, neuerNenner)
	return neuerBruch
\end{codePython}

\subsection{Tipp: Datentypen angeben}

Wer möchte, kann die Typen der Parameter und des Ergebnisses angeben. Es hilft, Fehler zu vermeiden. Programmierprofis tun das.

Die Signaturzeile für die Funktion zum Erweitern lautet dann:
\begin{codePython}
def erweitereBruch(bruch: Bruch, faktor: int) -> Bruch:
\end{codePython}

Der Rumpf bleibt unverändert.


