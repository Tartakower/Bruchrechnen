\section{Aufgaben zu Funktionen}

\subsection{Die Berechnung des Kehrwerts}

Der Kehrwert eines Bruches ist ein anderer Bruch, dessen Zähler der Nenner des ursprünglichen Bruchs und dessen Nenner der ursprüngliche Zähler ist. Beispielsweise lautet der Kehrwert des Bruchs $\frac{2}{3}$ demzufolge $\frac{3}{2}$.

Es gibt eine bekannte Rechenregel zum Dividieren von Brüchen: Ein Bruch wird durch einen anderen Bruch dividiert, indem man den ersten Bruch mit dem Kehrwert des zweiten Bruchs multipliziert:
\[
\frac{2}{5} : \frac{1}{2} = \frac{2}{5} \cdot \frac{2}{1} = \frac{4}{5}
\]