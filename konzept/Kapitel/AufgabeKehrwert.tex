\section{Beispiel: Der Kehrwert eines Bruches}

Lies dieses Beispiel sorgfältig durch. Es dient als Vorlage für die Aufgaben, die du danach selbstständig lösen sollst.

\subsection{Der Kehrwert eines Bruches}

Der Kehrwert eines Bruches ist ein anderer Bruch, dessen Zähler der Nenner des ursprünglichen Bruchs und dessen Nenner der ursprüngliche Zähler ist. Beispielsweise lautet der Kehrwert des Bruchs $\frac{2}{3}$ demzufolge $\frac{3}{2}$.

Es gibt eine bekannte Rechenregel zum Dividieren von Brüchen: Ein Bruch wird durch einen anderen Bruch dividiert, indem man den ersten Bruch mit dem Kehrwert des zweiten Bruchs multipliziert:
\[
\frac{2}{5} : \frac{1}{2} = \frac{2}{5} \cdot \frac{2}{1} = \frac{4}{5}
\]

\subsection{Der Algorithmus}

Das Bilden des Kehrwerts funktioniert so:
\begin{itemize}
	\item Nimm einen Bruch.
	\item Benenne eine Variable mit dem Namen \texttt{neuerZaehler} und weise dieser Variablen den Nenner des übergebenen Bruchs zu.
	\item Verfahre genauso mit dem neuen Nenner.
	\item Erstelle einen neuen Bruch aus dem neuen Zähler und dem neuen Nenner.
\end{itemize}

\subsection{Die Funktion zur Berechung des Kehrwerts}

\begin{codePython}
\end{codePython}

\subsection{Die Funktion zur Darstellung der Rechnung zum Kürzen}

Der Algorithmus lautet:
\begin{itemize}
	\item Nimm einen Bruch.
	\item Erzeuge aus diesem Bruch einen gekürzten Bruch. Verwende dazu die Funktion, die du eben gesehen hast.
	\item Erzeuge aus dem ursprünglichen Bruch die Beschreibung in MathML. Verwende dazu die Funktion, die du im vorherigen Beispiel gesehen hast.
	\item Erzeuge aus dem gekürzten Bruch die Beschreibung in MathML. Verwende dazu ebenfalls die Funktion aus dem vorherigen Beispiel.
	\item Setze jetzt den MathML-Text aus den einzelnen Angaben zusammen.
	\item Die Zeichenkette beginnt mit <math>.
	\item Es folgt der MathML-Text für den ungekürzten Bruch.
	\item Dann kommt das Gleichheitszeichen.
	\item Es folge der MathML-Text für den gekürzten Bruch.
	\item Die Zeichenkette beginnt mit </math>
\end{itemize}

Die Funktion lautet dann:
\begin{codePython}{MathML für das Kürzen eines Bruchs}{code:schreibeKuerzen}
def schreibeKuerzen(bruch: Bruch) -> str:
	bruchGekuerzt = kuerzeBruch(bruch)
	textUngekuerzt = schreibeBruch(bruch)
	textGekuerzt = schreibeBruch(bruchGekuerzt)
	return "<math>" + textUngekuerzt + "<mo>=</mo>" + textGekuerzt + "</math>"
\end{codePython}

\subsection{Überprüfung des Ergebnisses}

Wir überprüfen das Ergebnis analog zum Vorgehen im Beispiel für die Darstellung eines Bruchs. Dazu ergänzen wir in der Funktion \texttt{schreibeMathML()} die
Zeilen

\begin{codePython}{Integration des Kürzens}
bruchUngekuerzt: Bruch = Bruch(6,8)
inhalt += "\n\t\t<p>"
inhalt += schreibeKuerzen(bruchUngekuerzt)
inhalt += "</p>"
\end{codePython}