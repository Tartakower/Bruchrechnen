\section{Lösung: Das Kürzen von Brüchen}

Lies dieses Beispiel sorgfältig durch. Es dient als Vorlage für die Aufgaben, die du danach selbstständig lösen sollst.

\subsection{Der Algorithmus}

Das Kürzen eines Bruches funktioniert so:
\begin{itemize}
	\item Nimm einen Bruch.
	\item Merke dir den Zähler und den Nenner des Bruchs.
	\item Berechne den größten gemeinsamen Teiler von Zähler und Bruch. (Es gibt dafür eine fertige Funktion.)
	\item Berechne den gekürzten Zähler.
	\item Berechne den gekürzten Nenner.
	\item Erstelle einen neuen Bruch aus dem gekürzten Zähler und dem gekürzten Nenner.
\end{itemize}

\subsection{Die Funktion zum Kürzen von Brüchen}

Die Funktion lautet dann:

\begin{codePython}{Funktion zum Kürzen von Brüchen}
def kuerzeBruch(bruch: Bruch) -> Bruch:
	alterZaehler = bruch.zaehler
	alterNenner = bruch.nenner
	kuerzungsFaktor = ggT(alterZaehler, alterNenner)
	neuerZaehler = alterZaehler // kuerzungsFaktor
	neuerNenner = alterNenner // kuerzungsFaktor
	neuerBruch = Bruch(neuerZaehler, neuerNenner)
	return neuerBruch
\end{codePython}

\subsection{Die Funktion zur Darstellung der Rechnung zum Kürzen}

Der Algorithmus lautet:
\begin{itemize}
	\item Nimm einen Bruch.
	\item Erzeuge aus diesem Bruch einen gekürzten Bruch. Verwende dazu die Funktion, die du eben gesehen hast.
	\item Erzeuge aus dem ursprünglichen Bruch die Beschreibung in MathML. Verwende dazu die Funktion, die du im vorherigen Beispiel gesehen hast.
	\item Erzeuge aus dem gekürzten Bruch die Beschreibung in MathML. Verwende dazu ebenfalls die Funktion aus dem vorherigen Beispiel.
	\item Setze jetzt den MathML-Text aus den einzelnen Angaben zusammen.
	\item Die Zeichenkette beginnt mit <math>.
	\item Es folgt der MathML-Text für den ungekürzten Bruch.
	\item Dann kommt das Gleichheitszeichen.
	\item Es folge der MathML-Text für den gekürzten Bruch.
	\item Die Zeichenkette beginnt mit </math>
\end{itemize}

Die Funktion lautet dann:
\begin{codePython}{MathML für das Kürzen eines Bruchs}{code:schreibeKuerzen}
def schreibeKuerzen(bruch: Bruch) -> str:
	bruchGekuerzt = kuerzeBruch(bruch)
	textUngekuerzt = schreibeBruch(bruch)
	textGekuerzt = schreibeBruch(bruchGekuerzt)
	return "<math>" + textUngekuerzt + "<mo>=</mo>" + textGekuerzt + "</math>"
\end{codePython}

\subsection{Überprüfung des Ergebnisses}

Wir überprüfen das Ergebnis analog zum Vorgehen im Beispiel für die Darstellung eines Bruchs. Dazu ergänzen wir in der Funktion \texttt{schreibeMathML()} die
Zeilen

\begin{codePython}{Integration des Kürzens}
bruchUngekuerzt: Bruch = Bruch(6,8)
inhalt += "\n\t\t<p>"
inhalt += schreibeKuerzen(bruchUngekuerzt)
inhalt += "</p>"
\end{codePython}