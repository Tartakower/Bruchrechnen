\section{Lösung: Das Kürzen eines Bruchs}

\subsection*{Der Algorithmus zum Kürzen eines Bruchs}

\begin{itemize}
	\item Nimm einen Bruch.
	\item Merke dir den Zähler des Bruchs in einer Variablen.
	\item Verfahre ebenso mit dem Nenner.
	\item Berechne den größten gemeinsamen Teiler (ggT) von Zähler und Nenner. (Es gibt dafür eine fertige Funktion.)
	\item Berechne den gekürzten Zähler, indem du den alten Zähler durch den ggT teilst. 
	\item Berechne ebenso den gekürzten Nenner.
	\item Erstelle einen neuen Bruch aus dem gekürzten Zähler und dem gekürzten Nenner.
	\item Gib den neuen Bruch zurück.
\end{itemize}

\subsection*{Die Funktion zum Kürzen von Brüchen}

Die Berechnung des ggT erfolgt mittels der Funktion \texttt{mathefunktionen.berechne\_ggT}. In der Funktion zum Kürzen wird also eine Zeile enthalten sein:

\begin{codePython}
ggT = mathefunktionen.berechne_ggT(alterZaehler, alterNenner)
\end{codePython}

Beachte, dass beim Teilen durch den ggT die Division \texttt{//} benutzt wird, damit der neue Zähler und der neue Nenner ganze Zahlen werden!

\begin{codePython}
def kuerzeBruch(bruch: Bruch) -> Bruch:
	alterZaehler = bruch.zaehler
	alterNenner = bruch.nenner
	ggT = mathefunktionen.berechne_ggT(alterZaehler, alterNenner)
	neuerZaehler = alterZaehler // ggT
	neuerNenner = alterNenner // ggT
	neuerBruch = Bruch(neuerZaehler, neuerNenner)
	return neuerBruch
\end{codePython}

\subsection*{Die Algorithums zur Darstellung der Formel zum Kürzen}

\begin{itemize}
	\item Nimm einen Bruch.
	\item Erzeuge aus diesem Bruch einen gekürzten Bruch. Verwende dazu die Funktion, die du gerade programmiert hast.
	\item Erzeuge aus dem ursprünglichen Bruch die Beschreibung in MathML. Verwende dazu die Funktion, die einen Bruch in MathML beschreibt.
	\item Erzeuge aus dem gekürzten Bruch ebenfalls die Beschreibung in MathML. Das geht genauso wie eben.
	\item Setze jetzt den MathML-Text aus den einzelnen Angaben zusammen.
	\item Die Zeichenkette beginnt mit dem MathML-Text für den ungekürzten Bruch.
	\item Dann kommt das Gleichheitszeichen.
	\item Es folge der MathML-Text für den gekürzten Bruch.
\end{itemize}

\subsection*{Die Funktion zur Darstellung der Formel zum Kürzen}

\begin{codePython}
def schreibeKuerzen(bruch: Bruch) -> str:
	bruchGekuerzt = kuerzeBruch(bruch)
	textUngekuerzt = schreibeBruch(bruch)
	textGekuerzt = schreibeBruch(bruchGekuerzt)
	ergebnis = textUngekuerzt + "<mo>=</mo>" + textGekuerzt
	return ergebnis
\end{codePython}

\subsection*{Überprüfung des Ergebnisses}

Überprüfe das Ergebnis in der Datei \texttt{uebungsplatz.py}. 

\begin{codePython}
bruch = Bruch(6,8)
formel = schreibeKuerzen(bruch)
print(formel)
\end{codePython}

Der Bruch $\frac{6}{8}$ muss gekürzt den Bruch $\frac{3}{4}$ ergeben. In einer Zeile steht:

\begin{codeHTML}
<mfrac><mn>6</mn><mn>8</mn></mfrac><mo>=</mo>
	<mfrac><mn>3</mn><mn>4</mn></mfrac>
\end{codeHTML}

Ergänze in der Funktion \texttt{schreibeMathML()} die Zeilen für Kürzen für den Bruch $\frac{6}{8}$.

\begin{codePython}
bruch = Bruch(6,8)
inhalt = inhalt + "\n\t\t<p><math>"
inhalt = inhalt + schreibeKuerzen(bruch)
inhalt = inhalt + "</math></p>"
\end{codePython}

Führe die Datei \texttt{htmlErzeugung.py} aus.

Prüfe im Firefox-Browser, wie die Datei \texttt{index.html} aussieht. Dort sollte eine Zeile stehen, die so aussieht:
\[
\frac{6}{8} = \frac{3}{4}
\]
