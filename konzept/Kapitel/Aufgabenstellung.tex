\section{Die Aufgabenstellung}

Der Mathehelfer Bruchrechnen kann in der Grundversion Brüche kürzen, erweitern und miteinander multiplizieren.

Die Darstellung der Umrechnungen erfolgt in einem Browser. Dabei werden die Brüche grafisch schön mit Zähler, Nenner und Bruchstrich dargestellt wie in einem Mathematikbuch.

\begin{align*}
& \frac{6}{8} = \frac{3}{4} & \\[2ex]
& \frac{3}{4} = \frac{9}{12} & \\[2ex]
& \frac{3}{4} \cdot \frac{2}{3} = \frac{1}{2} & 
\end{align*}


Dabei geben wir die linke Seite vor und berechnen die rechte Seite jeweils mit einem Programm. Wir schreiben jeweils eine eigene Funktion, also drei Funktionen, die

\begin{itemize}
	\item einen Bruch kürzt,
	\item einen Bruch mit einem beliebigen Faktor erweitert oder
	\item zwei Brüche miteinander multipliziert.
\end{itemize}

Als Programmiersprache verwenden wir Python.

Die Formelzeile tragen wir automatisch in eine kleine Web-Seite ein, die wir uns im Browser ansehen. Dabei beschreiben wir die Formel in einer speziellen HTML-Sprache für mathematische Formeln: MathML.

Die Sprache MathML ist sehr einfach. Allerdings muss man sehr viel Text eingeben, um eine Formel mit MathML zu beschreiben. Daher schreiben wir ein Programm, das die Formel automatisch in MathML ausdrückt.