\section{Aufgabe: Das Umwandeln eines Bruchs in eine Kommmazahl}

Diese Aufgabe ist einfach.

\subsection*{Überlege: Wie wird ein Bruch in eine Kommazahl umgewandelt?}

\subsection*{Formuliere den Algorithmus!}

\subsection*{Programmiere die Funktion zum Umwandeln.}

\subsection*{Zusatz: Eine Datenklasse für Kommazahlen}

Diese Zusatzaufgabe ist mittelschwer.

Grundsätzlich können wir mit dem Datentyp \texttt{float} arbeiten. Schöner ist es, eine eigene Datenklasse für Kommazahlen zu haben, ähnlich zur Datenklasse für Brüche.

Wie sieht eine Datenklasse für eine Kommazahl aus? Bitte nicht wundern - das Ergebnis ist sehr einfach.