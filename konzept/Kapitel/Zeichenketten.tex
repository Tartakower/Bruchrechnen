\section{Zeichenketten}

\subsection*{Was ist eine Zeichenkette?}

Eine Zeichenkette enthält ein oder mehrere beliebige Zeichen, also Buchstaben, Zahlen oder Sonderzeichen. Die Zeichen stehen hintereinander. Eine Zeichenkette wird eingeschlossen in Hochkommata (') oder hochgestellte Anführungszeichen (").

\begin{codePython}
eine_zeichenkette = "r2d2"
berg = '8000er-Gipfel'
ein_ganzer_satz = "Python ist toll!"
\end{codePython}

Häufig benutzen wir den englischen Begriff für Zeichenkette: String. Mit Strings kann man viele interessante Dinge anstellen.

\subsection*{Das Zusammenfügen von Zeichenketten}

In Python können zwei Strings mit dem Pluszeichen zu einem langen String zusammengefügt werden:

\begin{codePython}
gruss = "Hallo " + "Welt!"
\end{codePython}

In \texttt{gruss} steht jetzt der String \texttt{"Hallo Welt!"}.

\subsection*{Tipp: Das Pluszeichen setzt Zeichenketten zusammen!}

Das Pluszeichen bedeutet in diesem Fall nicht das Addieren von Zahlen, sondern das Zusammenfügen von Zeichenketten. Daran gewöhnt man sich schnell!

\subsection*{MathML als Zeichenkette}

Wir möchten einen String zusammenbauen, der in MathML eine Rechenformel beschreibt.

Die Formel
\[
2 + 3 = 5
\]
lautet in MathML:

\begin{codeHTML}
<math><mn>2</mn><mo>+</mo><mn>3</mn><mo>=</mo><mn>5</mn></math>
\end{codeHTML}

Diese Formel können wir in Python schrittweise aus kleinen Strings zu einem langen String zusammensetzen. 

\lstset{style=syntaxPython}
\begin{lstlisting}
formel = "<math>"
formel = formel + "<mn>2</mn>"
formel = formel + "<mo>+</mo>"
formel = formel + "<mn>3</mn>"
formel = formel + "<mo>=</mo>"
formel = formel + "<mn>5</mn>"
formel = formel + "</math>"
\end{lstlisting}

Dabei nehmen wir die bereits erstellte Formel, fügen den nächsten kurzen String an und merken uns das Ergebnis wieder in der Variablen \texttt{formel}.