\section{Zeichenketten}

\subsection{Was ist eine Zeichenkette?}

\subsection{Das Zusammenfügen von Zeichenketten}

\subsection{Der Algorithmus zur Darstellung eines Bruchs in MathML}

Einen Bruch in MathML zu schreiben, ist nicht schwierig, aber mühsam. Daher möchten wir eine Funktion ein Python programmieren, die das automatisch erledigt. Der Algorithmus lautet so:
\begin{itemize}
	\item Nimm einen Bruch entgegen.
	\item Wandle den Zähler des Bruchs in eine Zeichenkette und speichere diese Zeichenkette in einer Variablen.
	\item Verfahre genauso mit dem Nenner.
	\item Baue die Zeichenkette für das Ergebnis aus Einzelteilen zusammen, wie in den nächsten Schritten beschrieben.
	\item Beginne die Zeichenkette mit \texttt{<mfrac>}.
	\item Rahme die Zeichenkette, die den Zähler beschreibt, mit den Tags \texttt{<mn>} und \texttt{</mn>} ein.
	\item Verfahre ebenso mit der Zeichenkette für den Nenner.
	\item Beende die Zeichenkette mit \texttt{</mfrac>}.
	\item Gib die Zeichenkette zurück.
\end{itemize}

\subsection{Die Funktion zur Darstellung eines Bruchs in MathML}

\begin{codePython}
def schreibeBruch(bruch: Bruch) -> str:	
	zaehlerAlsString = str(bruch.zaehler)
	nennerAlsString = str(bruch.nenner)
	ergebnis = "<mfrac>"
	ergebnis = ergebnis + "<mn>" + zaehlerAlsString + "</mn>"
	ergebnis = ergebnis + "<mn>" + nennerAlsString + "</mn>"
	ergebnis = ergebnis + "</mfrac>"
	return ergebnis
\end{codePython}