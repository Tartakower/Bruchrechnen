\section{Lösung: Das Erweitern eines Bruchs}

\subsection*{Der Algorithmus zum Erweitern}

\begin{itemize}
	\item Nimm einen Bruch und den Faktor, mit dem du den Bruch erweitern willst.
	\item Merke dir den Zähler in einer Variablen.
	\item Merke dir den Nenner des Bruchs in einer zweiten Variablen.
	\item Multipliziere die Variable für den Zähler mit dem Erweiterungsfaktor. Merke dir das Ergebnis in einer Variablen.
	\item Verfahre genauso mit dem Nenner.
	\item Erstelle einen neuen Bruch mit dem neuen Zähler und dem neuen Nenner. Merke dir das Ergebnis in einer Variablen.
	\item Gib die Variable zurück, die den neuen Bruch enthält.
\end{itemize}

\subsection*{Die Funktion zum Erweitern}

\begin{codePython}
def erweitereBruch(bruch: Bruch, faktor: int) -> Bruch:
	alterZaehler = bruch.zaehler
	alterNenner = bruch.nenner
	neuerZaehler = alterZaehler * faktor
	neuerNenner = alterNenner * faktor
	neuerBruch = Bruch(neuerZaehler, neuerNenner)
	return neuerBruch
\end{codePython}

\subsection*{Der Algorithmus zur Darstellung der Erweiterungsformel in MathML}

\begin{itemize}
	\item Nimm einen Bruch und einen Faktor zur Erweiterung entgegen.
	\item Erzeuge aus diesem Bruch einen erweiterten Bruch. Verwende dazu die Funktion \texttt{erweitereBruch}. Merke das Ergebnis in einer Variablen.
	\item Erzeuge aus dem ursprünglichen Bruch die Beschreibung in MathML. Verwende dazu die Funktion \texttt{schreibeBruch}. Merke das Ergebnis in einer Variablen.
	\item Erzeuge aus dem erweiterten Bruch die Beschreibung in MathML. Verwende dazu ebenfalls die Funktion \texttt{schreibeBruch}. Merke das Ergebnis in einer Variablen.
	\item Setze jetzt den MathML-Text aus den einzelnen Angaben zusammen wie in den folgenden Schritten angegeben. Verknüpfe die einzelnen Strings mit dem Pluszeichen.
	\item Es beginnt mit dem MathML-Text für den ungekürzten Bruch.
	\item Dann kommt das Gleichheitszeichen.
	\item Es folgt der MathML-Text für den gekürzten Bruch.
\end{itemize}

\subsection*{Die Funktion zur Darstellung der Formel in MathML}

\lstset{style=syntaxPython}
\begin{lstlisting}
def schreibeErweitern(bruch: Bruch, faktor: int) -> str:
	bruchErweitert = erweitereBruch(bruch, faktor)
	textAlterBruch = schreibeBruch(bruch)
	textNeuerBruch = schreibeBruch(bruchErweitert)
	ergebnis = textAlterBruch + "<mo>=</mo>" + textNeuerBruch
	return ergebnis
\end{lstlisting}

\subsection*{Überprüfung des Ergebnisses}

Überprüfe das Ergebnis in der Datei \texttt{uebungsplatz.py}. Wir erweitern beispielsweise den Bruch $\frac{2}{3}$ mit dem Faktor 3.

\begin{codePython}
bruch = Bruch(2,3)
formel = schreibeErweitern(bruch, 3)
print(formel)
\end{codePython}

Die resultierende Formel lautet (allerdings in einer Zeile):

\begin{codeHTML}
<mfrac><mn>2</mn><mn>3</mn></mfrac><mo>=</mo>
									<mfrac><mn>6</mn><mn>9</mn></mfrac>
\end{codeHTML}

\subsection*{Darstellung der Formel im Browser}

Wir können die Formel automatisch in die HTML-Seite übertragen lassen und anschließend im Browser anschauen.

Ergänze dazu in der Funktion \texttt{schreibeMathML()} folgende Zeilen:

\begin{codePython}
inhalt = inhalt + "\n\t\t<p><math>"
inhalt = inhalt + schreibeErweitern(Bruch(2,3), 3)
inhalt = inhalt + "</math></p>"
\end{codePython}

Jetzt führen wir die Datei \texttt{htmlErzeugung.py} aus.

Prüfe im Firefox-Browser, wie die Datei \texttt{index.html} aussieht. Dort sollte eine Zeile stehen, die so aussieht:
\[
\frac{2}{3} = \frac{6}{9}
\]