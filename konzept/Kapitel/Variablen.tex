\section{Variablen}

\subsection*{Einfache Variable}

Eine Variable dient dazu, sich einen Wert, einen Text oder das Ergebnis einer Rechnung zu merken. Wir geben einer Variablen einen sinnvollen Namen.

\begin{codePython}
wichtige_zahl = 42
summe = 2 + 3
\end{codePython}

Mit Variablen können wir auch rechnen.

\begin{codePython}
summe = summe + 6
\end{codePython}

\subsection*{Strukturierte Variablen}

Wir können Variablen auch komplizierter aufbauen und einen Wert in einer Variablen merken, der sich aus mehreren Teilwerten zusammensetzt. Ein Beispiel ist der Bruch:

\begin{codePython}
class Bruch():
	zaehler: int
	nenner: int
\end{codePython}

Ein Bruch ist ein Wert, der sich aus zwei Teilwerten zusammensetzt, nämlich dem Zähler und dem Nenner. Den Bruchstrich müssen wir uns nicht merken, weil er immer da ist. Zähler und Nenner sind ganze Zahlen (\texttt{int}).
%Wir merken uns den Zähler und Nenner zusammen in einem Bruch. Diesen Bruch können wir dann als einen Wert in einer Variablen aufbewahren.

\lstset{style=syntaxPython}
\begin{lstlisting}
bruch = Bruch(3,4)
\end{lstlisting}

Jetzt können wir die Variable, die ja einen Bruch enthält, nach dem Zähler und dem Nenner fragen. Wir schreiben die beiden Werte in eigene Variablen.

\begin{lstlisting}
zaehler = bruch.zaehler
nenner = bruch.nenner
\end{lstlisting}

Wir können auch zwei Ganzzahlen jeweils in eine Variable schreiben und damit eine Bruchvariable erzeugen.

\begin{lstlisting}
neuer_zaehler = 7
neuer_nenner = 8
bruch = Bruch(neuer_zaehler, neuer_nenner)
\end{lstlisting}