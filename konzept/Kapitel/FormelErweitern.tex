\section{Aufgabe: Die Formel zum Erweitern eines Bruchs in MathML}

\subsection*{Erledigte Vorarbeiten}

Welche bereits erstellten Funktionen können wir sinnvoll nutzen?

\begin{itemize}[itemsep=2ex]
	\item 
	\item
\end{itemize}

\subsection*{Der Algorithmus zur Darstellung der Erweiterungsformel in MathML}

Wie lautet der Algorithmus?

\begin{itemize}[itemsep=5ex]
	\item Nimm einen Bruch und einen Faktor zur Erweiterung entgegen.
	\item 
	\item 
	\item 
	\item 
	\item 
	\item 
	\item 
\end{itemize}

\pagebreak

\subsection*{Die Funktion zur Darstellung der Formel in MathML}

Wie lautet die Funktion? Die Signatur sieht so aus:

\lstset{style=syntaxPython}
\begin{lstlisting}
def schreibeErweitern(bruch: Bruch, faktor: int) -> str:
\end{lstlisting}

Wie geht es weiter?

\subsection*{Überprüfung des Ergebnisses}

Überprüfe das Ergebnis in der Datei
\begin{quote}
	\texttt{uebungsplatz.py}.
\end{quote}
Wir erweitern beispielsweise den Bruch $\frac{3}{4}$ mit dem Faktor 3.

\begin{codePython}
bruch = Bruch(2,3)
faktor = 3
\end{codePython}

Wie geht es weiter? Wie sieht das Ergebnis aus?

\subsection*{Darstellung der Formel im Browser}

Wir können die Formel automatisch in die HTML-Seite übertragen lassen und anschließend im Browser anschauen.

Ergänze in der Funktion
\begin{quote}
	\texttt{schreibeMathML()}
\end{quote}
die entsprechenden Zeilen analog zum Beispiel Kürzen.


Dann führe die Datei
\begin{quote}
\texttt{htmlErzeugung.py}
\end{quote} als Python-Skript aus.

Prüfe im Firefox-Browser, wie die Datei \texttt{index.html} aussieht. Dort sollte eine Zeile stehen, die so aussieht:
\[
\frac{3}{4} = \frac{9}{12}
\]