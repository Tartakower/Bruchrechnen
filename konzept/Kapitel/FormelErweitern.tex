\section{Beispiel: Die Formel zum Erweitern eines Bruchs in MathML}

\subsection*{Erledigte Vorarbeiten}

\begin{itemize}
	\item Wir verfügen über die Funktion \texttt{erweitereBruch}, die zu einem übergebenen Bruch und einem Faktor den erweiterten Bruch berechnet. (siehe \ref{sec:FunktionErweitern})
	\item Außerdem haben wir die Funktion \texttt{schreibeBruch}, die einen Bruch in MathML darstellt. (siehe \ref{sec:MathMLBruch})
\end{itemize}

\subsection*{Der Algorithmus zur Darstellung der Erweiterungsformel in MathML}

\begin{itemize}
	\item Nimm einen Bruch und einen Faktor zur Erweiterung entgegen.
	\item Erzeuge aus diesem Bruch einen erweiterten Bruch. Verwende dazu die Funktion \texttt{erweitereBruch}. Merke das Ergebnis in einer Variablen.
	\item Erzeuge aus dem ursprünglichen Bruch die Beschreibung in MathML. Verwende dazu die Funktion \texttt{schreibeBruch}. Merke das Ergebnis in einer Variablen.
	\item Erzeuge aus dem erweiterten Bruch die Beschreibung in MathML. Verwende dazu ebenfalls die Funktion \texttt{schreibeBruch}. Merke das Ergebnis in einer Variablen.
	\item Setze jetzt den MathML-Text aus den einzelnen Angaben zusammen wie in den folgenden Schritten angegeben:
	\item Es beginnt mit dem MathML-Text für den ungekürzten Bruch.
	\item Dann kommt das Gleichheitszeichen.
	\item Es folgt der MathML-Text für den gekürzten Bruch.
\end{itemize}

\subsection*{Die Funktion zur Darstellung der Formel in MathML}

\begin{codePython}
def schreibeErweitern(bruch: Bruch, faktor: int) -> str:
	bruchErweitert = erweitereBruch(bruch, faktor)
	textAlterBruch = schreibeBruch(bruch)
	textNeuerBruch = schreibeBruch(bruchErweitert)
	ergebnis = textAlterBruch + "<mo>=</mo>" + textNeuerBruch
	return ergebnis
\end{codePython}

\subsection*{Überprüfung des Ergebnisses}

Überprüfe das Ergebnis in der Datei \texttt{spielwiese.py}. Wir erweitern beispielsweise den Bruch $\frac{2}{3}$ mit dem Faktor 3.

\begin{codePython}
bruch = Bruch(2,3)
formel = schreibeErweitern(bruch, 3)
print(formel)
\end{codePython}

Die resultierende Formel lautet (allerdings in einer Zeile):

\begin{codeHTML}
<mfrac><mn>2</mn><mn>3</mn></mfrac><mo>=</mo>
									<mfrac><mn>6</mn><mn>9</mn></mfrac>
\end{codeHTML}

\subsection*{Darstellung der Formel im Browser}

Wir können die Formel automatisch in die HTML-Seite übertragen lassen und anschließend im Browser anschauen.

Ergänze dazu in der Funktion \texttt{schreibeMathML()} folgende Zeilen:

\begin{codePython}
inhalt = inhalt + "\n\t\t<p><math>"
inhalt = inhalt + schreibeErweitern(Bruch(2,3), 3)
inhalt = inhalt + "</math></p>"
\end{codePython}

Jetzt führen wir die Datei \texttt{htmlErzeugung.py} aus.

Prüfe im Firefox-Browser, wie die Datei \texttt{index.html} aussieht. Dort sollte eine Zeile stehen, die so aussieht:
\[
\frac{2}{3} = \frac{6}{9}
\]