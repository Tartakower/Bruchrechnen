\section{Aufgabe: Umwandeln in eine gemischte Zahl}

Diese Aufgabe ist schwer.

Der Wert eines unechten Bruchs, also eines Bruchs, dessen Zähler größer als der Nenner ist, ist oft erst nach etwas Kopfrechnen zu erkennen. Daher schreiben wir unechte Brüche als gemischte Zahl, also als Summe einer ganzen Zahl und einem echten Bruch. Beispiel:
\[
\frac{70}{12} = 5 \frac{5}{6}
\]

Einen echten Bruch können wir ebenfalls als gemischte Zahl schreiben. Dann ist die ganze Zahl einfach gleich Null und wird gar nicht hingeschrieben.

\subsection*{Programmiere die Datenklasse für eine gemischte Zahl}

Tipp: Benutze dafür die Datenklasse \texttt{Bruch}.

\subsection*{Überlege: Wie wird ein Bruch in eine gemischte Zahl umgewandelt?}

\subsection*{Überlege: Wie wird eine gemischte Zahl in einen Bruch umgewandelt?}

\subsection*{Formuliere die beiden Algorithmen zur Umwandlung zwischen Bruch und gemischter Zahl.}

\subsection*{Programmiere die Funktionen zur Umwandlung.}

Sieht deine Darstellung einer gemischten Zahl in MathML schön aus?