\section{Aufgabe: Objektorientierte Programmierung}

Diese Aufgabe ist sehr schwierig. Es werden gute Kenntnisse der Programmierung in Python vorausgesetzt.

\subsection{Ziel der Aufgabe}

Wir möchten nur noch \textbf{eine} Funktion haben, die das Ergebnis einer Berechnung von zwei Zahlen liefert, unabhängig davon, ob es Brüche, Kommazahlen oder gemischte Zahlen sind, auch wild durcheinander. Das ist sehr anspruchsvoll!

Tipp: Intern rechnen wir alle Zahlen in Brüche um und rechen dann mit Brüchen.

\subsection{Die Datenklassen für Bruch, Kommazahl, gemischte Zahl}

Betrachte die Datenklassen für Bruch, Kommazahl, gemischte Zahl. Welche Funktionen, die wir bisher programmiert haben, können wir als Methoden in die Klassen ziehen?

\subsection{Der Operator}

Wie können wir den Operator als Klasse beschreiben? Tipp: Es gibt den Aufzählungstyp \texttt{enum}.

\subsection{Die Superklasse}

Programmiere eine Superklasse für Bruch, Kommazahl und gemischte Zahl. Diese Superklasse sollte eine abstrakte Methode deklarieren, die ein Objekt in einen Bruch umwandelt.

Implementiere in den drei Subklassen diese Methode.