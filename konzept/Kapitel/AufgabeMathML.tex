\section{Aufgabenblatt MathML}

\subsection{Aufgabe: Anzeige der HTML-Datei}

Schau dir im Firefox-Browser die Datei \texttt{manuell.html} an. Die Anzeige sollte etwa so aussehen:

\begin{equation*}
\frac{6}{8} = \frac{3}{4}
\end{equation*}

Hier kannst du die Formeln für das Umrechnen eines Bruchs in eine Dezimalzahl und umgekehrt einfügen.

\subsection{Aufgabe: Ansicht der HTML-Datei im Editor}

Schau dir jetzt dieselbe Datei im Editor der Entwicklungsumgebung an. Der Text sieht etwa so aus:

\begin{codeHTML}{Die Datei manuell.html}
<!doctype html>
<html lang="de">
	<head>
		<meta charset="utf-8">
		<meta name="viewport" content="width=device-width, initial-scale=1.0">
		<title>Bruchrechnen</title>
	</head>
	<body>
		<p>
			<math>
				<mfrac>
					<mi>6</mi>
					<mi>8</mi>
				</mfrac>
				<mo>=</mo>
				<mfrac>
					<mi>3</mi>
					<mi>4</mi>
				</mfrac>
			</math>
		</p>
		<p>
 			Hier kannst du die Formeln für das Umrechnen ...
		</p>
	</body>
</html>
\end{codeHTML}

Kannst du die Element entdecken, die für die Darstellung des Kürzens der beiden Brüche stehen?

\subsection{Aufgabe: Die Beschreibung der beiden Umrechungen}

Füge in die Datei \texttt{manuell.html} den Text in der MathML-Sprache ein, so dass die beiden Umrechungen dargestellt werden (zusätzlich zur Formel für das Kürzen). Der Text „Hier kannst du \dots“ darf überschrieben werden.

\begin{align*}
& \frac{6}{8} = \frac{3}{4} & \\[2ex]
& \frac{3}{4} = 0.75 & \\[2ex]
& 0.8 = \frac{4}{5} & 
\end{align*}

Überprüfe das Ergebnis im Firefox-Browser.