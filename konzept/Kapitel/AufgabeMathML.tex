\section{Aufgabenblatt MathML}

\subsection*{Aufgabe: Anzeige der HTML-Datei}

Schau dir im Firefox-Browser die Datei \texttt{manuell.html} an. Die Anzeige sollte etwa so aussehen:
\[
\frac{6}{8} = \frac{3}{4}
\]

„Hier kannst du die Formeln für das Umrechnen eines Bruchs in eine Dezimalzahl und umgekehrt einfügen.“

\subsection*{Aufgabe: Ansicht der HTML-Datei im Editor}

Schau dir jetzt dieselbe Datei im Editor der Entwicklungsumgebung an. Der Text sieht etwa so aus:

\begin{codeHTML}
<!doctype html>
<html lang="de">
	<head>
		<meta charset="utf-8">
		<meta name="viewport" content="width=device-width, initial-scale=1.0">
		<title>Bruchrechnen</title>
	</head>
	<body>
		<p>
			<math>
				<mfrac>
					<mn>6</mn>
					<mn>8</mn>
				</mfrac>
				<mo>=</mo>
				<mfrac>
					<mn>3</mn>
					<mn>4</mn>
				</mfrac>
			</math>
		</p>
		<p>
 			Hier kannst du die Formeln für das Umrechnen ...
		</p>
	</body>
</html>
\end{codeHTML}

Kannst du die Element entdecken, die für die Darstellung des Kürzens der beiden Brüche stehen?

\subsection*{Aufgabe: Das Erweitern eines Bruchs}

Füge in die Datei
\begin{quote}
\texttt{manuell.html}
\end{quote}
den Text in der MathML-Sprache ein, so dass das Erweitern eines Bruchs dargestellt wird. Der Text „Hier kannst du \dots“ darf überschrieben werden.

Die letzte Zeile soll so aussehen:
\[
\frac{3}{4} = \frac{9}{12}
\]

Überprüfe das Ergebnis im Firefox-Browser. Die Datei \texttt{manuell.html} sollte jetzt so dargestellt werden:
\begin{align*}
& \frac{6}{8} = \frac{3}{4} & \\[2ex]
& \frac{3}{4} = \frac{9}{12} &
\end{align*}

\subsection*{Aufgabe: Das Multiplizieren von zwei Brüchen}

Füge in die Datei 
\begin{quote}
	\texttt{manuell.html}
\end{quote}
den Text in der MathML-Sprache ein, so dass diese Zeile als letzte Zeile angezeigt wird:
\[
\frac{3}{4} \cdot \frac{2}{3} = \frac{1}{2}
\]

Überprüfe das Ergebnis im Firefox-Browser. Die Seite sollte jetzt ungefähr so aussehen:
\begin{align*}
& \frac{6}{8} = \frac{3}{4} & \\[2ex]
& \frac{3}{4} = \frac{9}{12} & \\[2ex]
& \frac{3}{4} \cdot \frac{2}{3} = \frac{1}{2} & 
\end{align*}



