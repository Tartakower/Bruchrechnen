\section{Aufgabe: Das Erweitern eines Bruchs}

\subsection*{Der Algorithmus zum Erweitern}

Wie lautet der Algorithmus zur Erweitern eines Bruchs mit einem vorgegebenen Faktor?


\begin{itemize}[itemsep=5ex]
	\item Nimm einen Bruch und den Faktor, mit dem du den Bruch erweitern willst.
	\item  
	\item  
	\item  
	\item 
	\item  
	\item
\end{itemize}

\pagebreak

\subsection*{Die Funktion zum Erweitern}

Programmiere die Funktion zum Erweitern eines Bruchs. Die Funktion erhält als Parameter einen Bruch sowie den Faktor zum Erweitern. Die Signatur lautet:

\begin{codePython}
def erweitereBruch(bruch: Bruch, faktor: int) -> Bruch:
\end{codePython}

Vergiss nicht die letzte Zeile mit dem \texttt{return}-Befehl!

\subsection*{Überprüfung des Ergebnisses}

Überprüfe das Ergebnis. Trage dazu die notwendige Befehle in die Datei
\begin{quote}
	\texttt{uebungsplatz.py}
\end{quote}
ein. Du kannst dich daran orientieren, wie wir es beim Kürzen gemacht haben. Hier möchten wir den Bruch $\frac{3}{4}$ mit dem Faktor 3 erweitern.

Es geht los mit

\lstset{style=syntaxPython}
\begin{lstlisting}
bruch = Bruch(3,4)
\end{lstlisting}

Wie geht es weiter?

Danach führen wir \texttt{uebungsplatz.py} als Python-Skript aus. Das Ergebnis sollte jetzt diese Zeile enthalten:
\begin{quote}
\texttt{Bruch(zaehler=9, nenner=12)}
\end{quote}

Sieht das Ergebnis bei dir genauso aus?
