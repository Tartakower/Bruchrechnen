\section{Aufgabe: Das Multiplizieren von Brüchen}

\subsection*{Der Algorithmus zum Multiplizieren von Brüchen}

Wie lautet der Algorithmus zum Multiplizieren von Brüchen?

\begin{itemize}[itemsep=2ex]
	\item Nimm zwei Brüche entgegen.
	\item 
	\item 
	\item 
	\item 
	\item 
	\item  
	\item 
	\item 
\end{itemize}

Hast du daran gedacht, den neuen Bruch zu kürzen?

\subsection*{Die Funktion zum Multiplizieren von Brüchen}

Wie lautet die Funktion?

\subsection*{Überprüfung der Funktion zum Multiplizieren}

Wir möchten in der Datei
\begin{quote}
	\texttt{uebungsplatz.py}
\end{quote}
die beiden Brüche $\frac{3}{4}$ und $\frac{2}{3}$ miteinander multiplizieren. Es beginnt also mit

\begin{codePython}
erster_faktor = Bruch(3,4)
zweiter_faktor = Bruch(2,3)
\end{codePython}

Wie geht es weiter?

Führe \texttt{uebungsplatz.py} als Python-Skript aus. Die Ausgabe muss folgende Zeile enthalten:
\begin{quote}
	Produkt: Bruch(zaehler=1, nenner=2)
\end{quote}