\section{Beispiel: Die Darstellung eines Bruches in MathML}

Lies dieses Beispiel sorgfältig durch. Es dient als Vorlage für die Aufgaben, die du danach selbstständig lösen sollst.

\subsection{Der Algorithmus}

Einen Bruch in MathML zu schreiben, ist nicht schwierig, aber mühsam. Daher möchten wir eine Funktion ein Python programmieren, die das automatisch erledigt. Der Algorithmus lautet so:
\begin{itemize}
	\item Nimm einen Bruch entgegen.
	\item Wandle den Zähler des Bruchs in eine Zeichenkette.
	\item Wandle den Nenner des Bruchs in eine Zeichenkette.
	\item Gib der Zeichenkette, die den Bruch in MathML beschreibt, einen Namen.
	\item Baue die Zeichenketteaus Einzelteilen zusammen.
	\item Beginne die Zeichenkette mit \texttt{<mfrac>}.
	\item Rahme die Zeichenkette, die den Zähler beschreibt, mit den Tags \texttt{<mi>} und \texttt{</mi>} ein.
	\item Verfahre ebenso mit der Zeichenkette für den Nenner.
	\item Beende die Zeichenkette mit \texttt{</mfrac>}.
	\item Gib die Zeichenkette zurück.
\end{itemize}

\subsection{Die Funktion}
