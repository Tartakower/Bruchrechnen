\section{Vorkenntnisse}

Das Thema setzt nur wenige Vorkenntnisse zwingend voraus. Allerdings wächst der Spaß mit steigendem Schwierigkeitsgrad der Aufgaben.

\subsection*{Notwendige Vorkenntnisse}

Unverzichtbar ist eine gewisse Fertigkeit im Umgang mit einem PC oder Laptop, u.a.
\begin{itemize}
	\item flüssige Bedienung von Maus und Tastatur,
	\item Aufruf und Nutzung des Dateimanagers,
	\item das Betrachten von HTML-Dateien im Browser,
	\item das Editieren von Text in einem Editor.
\end{itemize}

Unbedingt bekannt sollte das Bruchrechnen sein. Für das Formulieren der Algorithmen ist die Kenntnis der Rechenregeln und von einschlägigen Merksätzen wichtig.

\subsection*{Hilfreiche Vorkenntnisse}

Grundsätzlich wird in Form der Funktion ein einfaches Sprachelement von Python genutzt, das allerdings in Lehrtexte häufig sehr spät eingeführt wird. Daher erklärt der Kurs, wie Funktionen in Python aufgebaut sind. Vorerfahrung in der Programmierung mit Python ist natürlich hilfreich, weil dann die anspruchsvolleren Aufgaben bewältigt werden können.

Bei der Generierung von MathML sind Vorkenntnisse in HTML hilfreich, etwa der grundsätzliche Aufbau mit öffnenden und schließenden Tags.

\subsection*{Abfrage der Vorkenntnisse und Erwartungen}

Bei der Vorstellung der TeilnehmerInnen werden die Vorkenntnisse und Erwartungen besprochen. Mögliche Fragen sind:

\begin{itemize}
	\item Nutzt du regelmäßig den Computer zum Schreiben von Texten, beispielsweise für die Schule?
	\item Hast du schon einmal eine Web-Seite erstellt? Kennst du dich mit HTML aus?
	\item Hast du schon einmal mit Python programmiert? Was hast du dort kennengelernt?
	\item Was wünscht du dir von diesem Workshop?
\end{itemize}