\section{Technische Rahmenbedingungen}

\subsection*{Hardware und Betriebssystem}

Die Beispiele wurden auf einem Raspberry Pi, Version 4, mit 8 GB Ram entwickelt. Damit lässt sich stabil und flüssig arbeiten.

Vermutlich reicht ein Raspi 4 mit mindestens 2 GB Ram aus.

Ein Raspi 3 mit 1 GB Ram läuft nicht mehr performant mit einer anspruchsvollen IDE wie z.B. VS Code.

Als Betriebssystem wurden verschiedene Debian-/Ubuntu-Derivate für arm64-Prozessoren genutzt. Diese arbeiten stabil und flüssig. Unter Windows oder MacOS sollte alles genauso gut klappen.

\subsection*{Programmiersprache}

Der Mathehelfer Bruchrechnen wird in Python programmiert. Da die Klasse für Brüche als Dataclass definiert werden, ist mindestens Version 3.7 erforderlich. Jede aktuelle Linux-Distribution enthält diese oder eine neuere Python-Version.

\subsection*{IDE}

Für Python existieren viele verschiedene Entwicklungsumgebungen. Um eine sinnvolle Unterstützung durch Code-Vervollständigung zu erhalten, sind folgende IDEs geeignet:
\begin{itemize}
	\item Visual Studio Code (Python-Plugin, Pylance-Plugin von Microsoft)
	\item Pycharm
	\item eclipse mit dem Pydev-Plugin
\end{itemize}

Die Online-IDE repl.it funktioniert ebenfalls gut. Allerdings ist der Vorteil der fehlenden Installation gegenüber der Online-Abhängigkeit abzuwägen.

\subsection*{Browser}

MathML wird aktuell unmittelbar nur von Firefox dargestellt.

Die Chromium-/Chrome-Familie stellt MathML nicht (mehr)  dar.