\section{Aufgabe: Die Formel zum Kürzen von Brüchen in MathML}

\subsection*{Die Vorarbeiten}

Hast du die Funktion zum Kürzen eines Bruchs bereit? Wie heißt sie?

Außerdem benötigst du wieder die Funktion zur Darstellung eines Bruchs in MathML.

\subsection*{Der Algorithmus}

Wie lautet der Algorithmus für die Formeldarstellung des Kürzens?

{\Large
	\begin{itemize}
		\item Nimm einen Bruch entgegen. 
		\item  
		\item  
		\item  
		\item  
		\item  
	\end{itemize}
}


\subsection*{Die Funktion}

Die Signatur der Funktion lautet:

\begin{codePython}
def schreibeKuerzen(bruch: Bruch) -> str:
\end{codePython}

Wie geht es weiter?

\subsection*{Überprüfung des Ergebnisses}

Überprüfe das Ergebnis in der Datei \texttt{spielwiese.py}. Der Bruch $\frac{6}{8}$ muss gekürzt den Bruch $\frac{3}{4}$ ergeben.

\begin{codeHTML}
<mfrac><mn>6</mn><mn>8</mn></mfrac><mo>=</mo>
	<mfrac><mn>3</mn><mn>4</mn></mfrac>
\end{codeHTML}

\subsection*{Darstellung der Formel im Browser}

Ergänze in der Funktion \texttt{schreibeMathML()} die Zeilen für Kürzen für den Bruch $\frac{6}{8}$ analog zum Erweitern.

Führe die Datei \texttt{htmlErzeugung.py} aus.

Prüfe im Firefox-Browser, wie die Datei \texttt{index.html} aussieht. Dort sollte eine Zeile stehen, die so aussieht:
\[
\frac{6}{8} = \frac{3}{4}
\]