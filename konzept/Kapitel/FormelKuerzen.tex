\section{Beispiel: Die Formel zum Kürzen eines Bruchs in MathML}

\subsection*{Die Vorarbeiten}

\begin{itemize}
	\item Wir verfügen über die Funktion \texttt{kuerzeBruch}. (siehe \ref{sec:FunktionKuerzen})
	\item Außerdem haben wir die Funktion \texttt{schreibeBruch}, die einen Bruch in MathML darstellt. (siehe \ref{sec:MathMLBruch})
\end{itemize}

\subsection*{Der Algorithmus zur Darstellung der Formel zum Kürzen}

\begin{itemize}
	\item Nimm einen Bruch.
	\item Erzeuge aus diesem Bruch einen gekürzten Bruch. Verwende dazu die Funktion, die du gerade programmiert hast.
	\item Erzeuge aus dem ursprünglichen Bruch die Beschreibung in MathML. Verwende dazu die Funktion, die einen Bruch in MathML beschreibt.
	\item Erzeuge aus dem gekürzten Bruch ebenfalls die Beschreibung in MathML. Das geht genauso wie eben.
	\item Setze jetzt den MathML-Text aus den einzelnen Angaben zusammen.
	\item Die Zeichenkette beginnt mit dem MathML-Text für den ungekürzten Bruch.
	\item Dann kommt das Gleichheitszeichen.
	\item Es folge der MathML-Text für den gekürzten Bruch.
\end{itemize}

\subsection*{Die Funktion zur Darstellung der Formel zum Kürzen}

In der Datei \textbf{\texttt{mathehelfer.py}} programmieren wir folgende Funktion:

\begin{codePython}
def schreibeKuerzen(bruch: Bruch) -> str:
	bruchGekuerzt = kuerzeBruch(bruch)
	textUngekuerzt = schreibeBruch(bruch)
	textGekuerzt = schreibeBruch(bruchGekuerzt)
	ergebnis = textUngekuerzt + "<mo>=</mo>" + textGekuerzt
	return ergebnis
\end{codePython}

\subsection*{Überprüfung des Ergebnisses}

Überprüfe das Ergebnis in der Datei \textbf{\texttt{uebungsplatz.py}}.

\lstset{style=syntaxPython}
\begin{lstlisting}
bruch = Bruch(6,8)
formel = mathehelfer.schreibeKuerzen(bruch)
print(formel)
\end{lstlisting}

Der Bruch $\frac{6}{8}$ muss gekürzt den Bruch $\frac{3}{4}$ ergeben. In einer Zeile steht:

\begin{codeHTML}
<mfrac><mn>6</mn><mn>8</mn></mfrac><mo>=</mo>
								<mfrac><mn>3</mn><mn>4</mn></mfrac>
\end{codeHTML}

Jetzt arbeiten wir wieder in der Datei \textbf{\texttt{mathehelfer.py}}.

Ergänze in der Funktion \texttt{schreibeMathML()} die Zeilen für Kürzen für den Bruch $\frac{6}{8}$.

\begin{codePython}
bruch = Bruch(6,8)
inhalt = inhalt + "\n\t\t<p><math>"
inhalt = inhalt + schreibeKuerzen(bruch)
inhalt = inhalt + "</math></p>"
\end{codePython}

Führe die Datei \texttt{htmlErzeugung.py} aus.

Prüfe im Firefox-Browser, wie die Datei \texttt{index.html} aussieht. Dort sollte eine Zeile stehen, die so aussieht:
\[
\frac{6}{8} = \frac{3}{4}
\]
