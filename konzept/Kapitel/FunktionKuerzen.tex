\section{Beispiel: Das Kürzen eines Bruchs}

Im folgenden Beispiel lernst du das Programmieren einer Funktion in Python.

\subsection*{Aufgabe: Der Algorithmus zum Kürzen eines Bruchs}

Lies den Algorithmus zum Kürzen eines Bruchs durch:

\begin{itemize}
	\item Nimm einen Bruch.
	\item Merke dir den Zähler des Bruchs in einer Variablen.
	\item Merke dir den Nenner des Bruchs in einer anderen Variablen.
	\item Berechne den größten gemeinsamen Teiler (ggT) von Zähler und Nenner. Es gibt dafür eine fertige Funktion.
	\item Berechne den gekürzten Zähler, indem du den alten Zähler durch den ggT teilst. 
	\item Berechne ebenso den gekürzten Nenner.
	\item Erstelle einen neuen Bruch aus dem gekürzten Zähler und dem gekürzten Nenner.
	\item Gib den neuen Bruch zurück.
\end{itemize}

Verstehst du das Vorgehen? Andernfalls stelle deine Fragen dem Betreuer.

\subsection*{Der größte gemeinsame Teiler zweier Zahlen (ggT)}
\label{sec:FunktionGgT}

Die Berechnung des ggT erfolgt mittels der Funktion
\begin{quote}
\texttt{mathefunktionen.berechne\_ggT}.
\end{quote}

In der Funktion zum Kürzen wird also eine Zeile enthalten sein:

\begin{codePython}
ggT = mathefunktionen.berechne_ggT(alterZaehler, alterNenner)
\end{codePython}

\pagebreak

\subsection*{Die Funktion zum Kürzen von Brüchen}
\label{sec:FunktionKuerzen}

Beachte, dass beim Teilen durch den ggT die Division \texttt{//} benutzt wird, damit der neue Zähler und der neue Nenner ganze Zahlen werden!

\begin{codePython}
def kuerzeBruch(alterBruch: Bruch) -> Bruch:
	alterZaehler = alterBruch.zaehler
	alterNenner = alterBruch.nenner
	ggT = mathefunktionen.berechne_ggT(alterZaehler, alterNenner)
	neuerZaehler = alterZaehler // ggT
	neuerNenner = alterNenner // ggT
	neuerBruch = Bruch(neuerZaehler, neuerNenner)
	return neuerBruch
\end{codePython}

\subsection*{Aufgabe: Programmieren der Funktion}

Wir bearbeiten jetzt die Datei

\begin{quote}
\textbf{\texttt{mathehelfer.py}}
\end{quote}

hinter der Zeile

\begin{quote}
\textbf{\texttt{\# Ab hier dürft ihr Änderungen vornehmen.}}
\end{quote}

Gib die Funktion in den Editor der Entwicklungsumgebung (IDE) ein. Zeigt die IDE noch Fehler an? Vielleicht hast du dich vertippt? Falls du den Fehler nicht finden kannst, sprich den Betreuer an.

\subsection*{Überprüfung des Ergebnisses}

Wir überprüfen das Ergebnis mit Hilfe folgender Datei:

\begin{quote}
\textbf{\texttt{uebungsplatz.py}}
\end{quote}

Wir tragen dort die folgenden Zeilen ein:

\begin{codePython}
bruch = Bruch(6,8)
gekuerzter_bruch = mathehelfer.kuerzeBruch(bruch)
print(gekuerzter_bruch)
\end{codePython}

\subsection*{Ausführen eines Python-Programms}

Jetzt führen wir \texttt{uebungsplatz.py} als Python-Programm aus. Das geht in VS Code auf zwei verschiedene Weisen:

\begin{itemize}
	\item Im Terminal: \texttt{python3 uebungsplatz.py}
	\item Klicke mit der Maus rechts oben in der Ecke auf $\medtriangleright$ und wähle \textit{Führen Sie die Python-Datei aus}.
\end{itemize}


Das Ergebnis sollte jetzt diese Zeile enthalten:
\begin{quote}
\texttt{Bruch(zaehler=3, nenner=4)}
\end{quote}

Sieht das Ergebnis bei dir genauso aus?

